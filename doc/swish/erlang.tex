% Copyright 2018 Beckman Coulter, Inc.
%
% Permission is hereby granted, free of charge, to any person
% obtaining a copy of this software and associated documentation files
% (the "Software"), to deal in the Software without restriction,
% including without limitation the rights to use, copy, modify, merge,
% publish, distribute, sublicense, and/or sell copies of the Software,
% and to permit persons to whom the Software is furnished to do so,
% subject to the following conditions:
%
% The above copyright notice and this permission notice shall be
% included in all copies or substantial portions of the Software.
%
% THE SOFTWARE IS PROVIDED "AS IS", WITHOUT WARRANTY OF ANY KIND,
% EXPRESS OR IMPLIED, INCLUDING BUT NOT LIMITED TO THE WARRANTIES OF
% MERCHANTABILITY, FITNESS FOR A PARTICULAR PURPOSE AND
% NONINFRINGEMENT. IN NO EVENT SHALL THE AUTHORS OR COPYRIGHT HOLDERS
% BE LIABLE FOR ANY CLAIM, DAMAGES OR OTHER LIABILITY, WHETHER IN AN
% ACTION OF CONTRACT, TORT OR OTHERWISE, ARISING FROM, OUT OF OR IN
% CONNECTION WITH THE SOFTWARE OR THE USE OR OTHER DEALINGS IN THE
% SOFTWARE.

\chapter {Erlang Embedding}\label{chap:erlang}

\section {Introduction}

This chapter describes the design of the message-passing concurrency
model. It provides a Scheme embedding of a significant subset of the
Erlang programming
language~\cite{armstrong-thesis,programming-erlang}.\footnote{Tuples,
  denoted by \{$e_1$, \ldots, $e_n$\} in Erlang, are implemented as
  vectors: \code{\#(\var{e$_1$}~\etc~\var{e$_n$})}. Similarly
  records, defined as syntactic sugar over tuples in Erlang, are
  implemented as syntactic sugar over vectors.}  Record and pattern
matching macros provide succinct ways of composing and decomposing
data structures.

The basic unit of sequential computation is the \emph{process}.  Each
process has independent state and communicates with other processes by
message passing. Because processes share no mutable state, one process
cannot corrupt the state of another process---a problem that plagues
software using shared-state concurrency. \concern{System procedures that
  mutate data can cause state corruption.} \mitigation The code is
inspected for use of these procedures.

An uncaught exception in one process does not affect any other
process. A process can be monitored for termination, and it can be
linked to another process so that, when either process exits, the
other one receives an exit signal.  Processes are implemented with
one-shot continuations~\cite{one-shot}, and the concurrent system is
simulated by the single-threaded program using software
timer interrupts.  The operating system interface (see
Chapter~\ref{chap:osi}) provides asynchronous input/output (I/O) so
that processes waiting for I/O do not stop other processes from
executing.

For exceptions, we use Erlang's approach of encoding the information
in a machine-readable datum rather than a formatted string. Doing so
makes it possible to write code that matches particular exceptions
without having to parse strings, and the exception is human language
independent.

The rest of this chapter is organized as follows.
Section~\ref{sec:erlang-data-structures} introduces the main data structures,
Section~\ref{sec:erlang-theory} describes how the concurrency
model works, and Section~\ref{sec:erlang-api} gives the
programming interface.

\section {Data Structures}\label{sec:erlang-data-structures}

\paragraph {\code{q}}\index{q@\code{q}}
Queues are used in several key places: the inbox of messages for each
process, the list of processes ready to run, and the list of sleeping
processes. A \emph{queue} is a doubly-linked list with a sentinel
value, the queue's identity. Both the sentinel value and the elements
of the queue are instances of \code{q}, a Scheme record type with
mutable \code{prev} and \code{next} fields.  This representation
enables constant-time insertion and deletion operations.

\paragraph {\code{msg}}\index{msg@\code{msg}}
When a \emph{message} is sent to a process, its contents are wrapped
in an instance of \code{msg}, a Scheme record type that extends
\code{q} with an immutable \code{contents} field. This
\code{msg} is inserted into the process's inbox and removed when the
process receives it.

\paragraph {\code{pcb}}\index{pcb@\code{pcb}}
A \emph{process} is an instance of \code{pcb}, a Scheme record type
that extends \code{q} with an immutable \code{id} field, the process's
unique positive exact integer, an immutable \code{create-time} field,
the process's create time from \code{erlang:now}, and the following
mutable fields:
\begin{itemize}
\item \code{name}: registered name or \code{\#f}
\item \code{cont}: one-shot continuation if live and not currently
  running or \code{\#f} otherwise
\item \code{sic}: system interrupt count
\item \code{winders}: list of winders if live and not currently
  running or \code{()} otherwise
\item \code{exception-state}: exception state if live and not
  currently running, exit reason if dead, or \code{\#f} if currently
  running
\item \code{inbox}: queue of \code{msg} if live or \code{\#f} if
  dead
\item \code{precedence}: wake time if sleeping or 0 if ready to run
\item \code{flags}: fixnum with bit 0 set when sleeping and bit 1
  set when the process traps exits
\item \code{links}: list of linked processes
\item \code{monitors}: list of monitors
\item \code{src}: source location \code{\#(at \var{char-offset}
  \var{filename})} when available if waiting in a \code{receive}
  macro or \code{\#f} otherwise
\end{itemize}

\paragraph {\code{mon}}\index{mon@\code{mon}}
A \emph{monitor} is an instance of \code{mon}, a Scheme record type
with two immutable fields, \code{origin} and \code{target}, each
of which is a process.

\paragraph {\code{osi-port}}\index{osi-port@\code{osi-port}}
An \emph{osi-port} is an instance of \code{osi-port}, a Scheme
record type with an immutable \code{name} field and a mutable
\code{handle} field that wraps an operating system interface
port. The \code{handle} field is set to \code{\#f} when the
osi-port is closed.

\paragraph {\code{directory-watcher}}\index{directory-watcher@\code{directory-watcher}}
A \emph{directory watcher}\index{directory watcher} is an instance of
\code{directory-watcher}, a Scheme record type with a mutable
\code{handle} field and an immutable \code{path} field. The
\code{handle} field is set to \code{\#f} when the directory
watcher is closed.

\paragraph {\code{listener}}\index{listener@\code{listener}}
A \emph{TCP listener}\index{TCP listener} is an instance of
\code{listener}, a Scheme record type with immutable \code{address},
\code{port-number}, and \code{create-time} fields and a mutable
\code{handle} field. The \code{handle} field is set to \code{\#f} when
the listener is closed.

\section {Theory of Operation}\label{sec:erlang-theory}

The system uses a \emph{scheduler}\index{scheduler} to execute one
process at a time. Each process holds its own system interrupt count
(updated by \code{enable-interrupts} and \code{disable-interrupts}),
list of winders (maintained by \code{dynamic-wind} and the system
primitive \code{\$current-winders}), and exception state (maintained
by \code{current-exception-state}). The scheduler captures the
one-shot continuation for a process with an empty list of winders so
that, when it invokes the continuation of another process, it does not
run any winders. \concern{Using a system procedure that relies on the
  global winders list may lead to incorrect behavior.} \mitigation
System procedures that rely on the global winders list are called from
only one process at a time using the
\emph{gatekeeper}\index{gatekeeper} described in
Chapter~\ref{chap:gatekeeper}. The gatekeeper hooks the \code{\$cp0},
\code{\$np-compile}, \code{pretty-print}, and \code{sc-expand} system
primitives.

Spawning a new process is not as simple as capturing a one-shot
continuation and creating a \code{pcb} record, because the
continuation's stack link~\cite{representing-control} would be the
continuation of the caller, and its list of winders would be the
caller's. Thus, the scheduler remembers the current list of winders
and then sets it to the empty list before capturing a one-shot
continuation.  This return continuation is stored in a mutable
variable so that it is not closed over by the new process.  Next, a
full continuation is captured to create the initial exception state
that will terminate the new process when an uncaught exception is
raised. So that this full continuation does not refer to the caller's
continuation, the current stack link is set to the null continuation
before capturing it. After capturing the full continuation, a one-shot
continuation for the new process is captured and returned to the
caller via the return continuation.

Each process runs until it waits in a \code{receive} macro, is
preempted by the \code{timer-interrupt-handler}, or exits.  The
operating system interface (see Chapter~\ref{chap:osi}) provides
asynchronous I/O operations so that the scheduler can execute other
processes while the system is performing I/O.  The timer interrupt
handler runs every 1000 procedure calls.\footnote{1000 was chosen
  because Chez Scheme performs its internal interrupt checks every
  1000 ticks.} The scheduler uses \code{osi::SetTick} and
\code{osi::IsTickOver} to determine when the time quantum for a
process has elapsed.

When process \var{p} exits with reason \var{r}, the message
\code{\#(DOWN \var{m} \var{p} \var{r})} is sent to each of its
monitor \var{m}'s \code{origin} processes.  The message
\code{\#(EXIT \var{p} \var{r})} is sent to each linked process that
traps exits. If \var{r} is not \code{normal}, each linked process
that does not trap exits is killed with reason \var{r}.

A process can be registered with a global name, a symbol. This name
can be used instead of the process record itself to send it messages.
A global \emph{registrar}\index{registrar} maintains an eq-hashtable
mapping names to processes. The reverse mapping is maintained in the
\code{pcb} record through the \code{name} field.

There are two system processes: the
\emph{event-loop}\index{event-loop} and the
\emph{finalizer}\index{finalizer}.

The event-loop process calls \code{osi::GetCompletionPacket} to
retrieve completion packets from the operating system interface.  It
executes each callback with interrupts disabled.  Event-loop callbacks
are designed to execute quickly without failing or causing new
completion packets to be enqueued. Typical callbacks register objects
that wrap operating system interface handles with a guardian and send
messages to a process. If the event-loop process exits with reason
$r$, the system logs the event
\code{\#(event-loop-process-terminated $r$)} with
\code{console-event-handler} and calls \code{osi::ExitProcess}
with exit code 80.

The scheduler maintains the \emph{run queue}\index{run queue}, a queue
of ready-to-run processes, and the \emph{sleep queue}\index{sleep
  queue}, a queue of sleeping processes. Both are ordered by
increasing precedence and preserve the order of insertion for
processes with the same precedence. For the run queue, each process
has precedence 0 in order to implement round-robin scheduling. For the
sleep queue, each process uses its wake time as the precedence.

When the run queue is empty, the event-loop process calls
\code{osi::GetCompletionPacket} with a non-zero timeout based on the
first entry in the sleep queue to avoid busy waiting. When the
event-loop process finishes processing all completion packets, it
places itself at the end of the run queue.

\concern{Some process may starve another process.}  \mitigation The
run queue is managed with round-robin scheduling to prevent
starvation. The event-loop process does not starve other processes
because it drains the completion queue without causing new completion
packets to be enqueued.

The finalizer process runs the finalizers registered via
\code{add-finalizer}. These finalizers typically close operating
system interface handles to objects that are no longer
accessible. \concern{Ill-behaved finalizers may cause memory and handle
  leaks.}  \mitigation Finalizers are designed to execute quickly
without failing. Typical finalizers guard against errors when closing
handles. If the finalizer process exits with reason $r$, the
system logs the event
\code{\#(finalizer-process-terminated $r$)} with
\code{console-event-handler} and calls \code{osi::ExitProcess}
with exit code 80.

Once the finalizer process runs all the finalizers, it waits until
another garbage collection has occurred before running again. The
system hooks the \code{collect} procedure so that
it sends a wake-up message to the finalizer process every time a
garbage collection occurs.  When the finalizer receives the wake-up
message, it pumps all other wake-up messages from its inbox, since
there may have been more than one garbage collection since it last
ran.

Asynchronous I/O operations for COM ports, named pipes, external
operating system processes, files, console input, and TCP connections
are implemented with custom binary ports so that they have the same
interface as the system I/O procedures. The system I/O procedures are
not used because they perform synchronous I/O.  The custom port buffer
size is set to 1024\footnote{1024 was chosen because Chez Scheme uses
  1024 for the buffer size of buffered transcoded ports.} with
\code{custom-port-buffer-size}.  The custom binary port read and
write procedures call \code{osi::ReadPort} and
\code{osi::WritePort} with callbacks that send a message to the
calling process, which waits until it receives the message.

\concern{Using a port from more than one process at the same time may
  cause errors including buffer corruption.}  \mitigation The code is
inspected for concurrent use of ports. Port visibility is typically
limited to a single process.

For two-way communication ports, we use two custom ports: one
exclusively for input, and one exclusively for output. We do not use
custom input/output ports for two reasons.  First, textual
input/output ports created with \code{transcoded-port} are not safe
to use from two concurrent processes because one transcoding buffer is
used for both reading and writing.  Second, the input side of a port
is commonly used only by a reader process, and the output side of a
port is commonly used only by a writer process.  Keeping the input and
output sides separate prevents concurrent use. The underlying handle
is closed when the output port is closed.

\concern{Failing to close a handle from the operating system interface
  that is no longer used causes resource leaks.}  \mitigation An
\emph{osi-port guardian}\index{osi-port guardian} and associated
finalizer are used to identify and close inaccessible osi-ports using
\code{osi::ClosePort}. A \emph{directory-watcher
  guardian}\index{directory-watcher guardian} and associated finalizer
are used to identify and close inaccessible directory
watchers\index{directory watcher}.  A \emph{listener
  guardian}\index{listener guardian} and associated finalizer are used
to identify and close inaccessible TCP listeners\index{TCP
  listener}. In all cases, interrupts are disabled around code that
wraps handles and registers objects with guardians in order to prevent
the current process from being killed during this critical time.

\section {Programming Interface}\label{sec:erlang-api}

% ----------------------------------------------------------------------------
\subsection {Process Creation}

% ----------------------------------------------------------------------------
\defineentry{spawn}
\begin{procedure}
  \code{(spawn \var{thunk})}
\end{procedure}
\returns{} a process

The \code{spawn} procedure creates and returns a new process that
executes \var{thunk}, a procedure of no arguments. The new process
starts with \code{name} = \code{\#f}, \code{sic} = 0 (interrupts
enabled), \code{winders} = \code{()}, an \code{exception-state}
that terminates the process on an unhandled exception, an empty
\code{inbox}, \code{precedence} = 0, \code{flags} = 0 (the
process is not sleeping and does not trap exits), \code{links} =
\code{()}, \code{monitors} = \code{()}, and \code{src} =
\code{\#f}.

% ----------------------------------------------------------------------------
\defineentry{spawn\&link}
\begin{procedure}
  \code{(spawn\&link \var{thunk})}
\end{procedure}
\returns{} a process

Like \code{spawn}, the \code{spawn\&link} procedure creates and
returns a new process that executes \var{thunk}. In addition, it links
the new process to the calling process.

% ----------------------------------------------------------------------------
\subsection {Process Registration}

% ----------------------------------------------------------------------------
\defineentry{register}
\begin{procedure}
  \code{(register \var{name} \var{process})}
\end{procedure}
\returns{} \code{\#t}

The \code{register} procedure adds \var{name} $\rightarrow$
\var{process} to the registrar and sets \var{process}.\code{name} =
\var{name}. When a registered process exits, its registration is
removed. If \var{name} is not a symbol, exception \code{\#(bad-arg
  register \var{name})} is raised. If \var{process} is not a process,
exception \code{\#(bad-arg register \var{process})} is raised. If
\var{process} is dead, exception \code{\#(process-dead \var{process})}
is raised. If \var{process} is already registered to name $n$,
exception \code{\#(process-already-registered $n$)} is raised. If
\var{name} is already registered to process $p$, exception
\code{\#(name-already-registered $p$)} is raised.

% ----------------------------------------------------------------------------
\defineentry{unregister}
\begin{procedure}
  \code{(unregister \var{name})}
\end{procedure}
\returns{} \code{\#t}

The \code{unregister} procedure removes \var{name} $\rightarrow$
\var{process} from the registrar and sets \var{process}.\code{name}
= \code{\#f}.  If \var{name} is not registered, exception
\code{\#(bad-arg unregister \var{name})} is raised.

% ----------------------------------------------------------------------------
\defineentry{whereis}
\begin{procedure}
  \code{(whereis \var{name})}
\end{procedure}
\returns{} a process \alt{} \code{\#f}

The \code{whereis} procedure returns the process associated with
\var{name} or \code{\#f} if \var{name} is not registered.  If
\var{name} is not a symbol, exception \code{\#(bad-arg whereis
  \var{name})} is raised.

% ----------------------------------------------------------------------------
\defineentry{get-registered}
\begin{procedure}
  \code{(get-registered)}
\end{procedure}
\returns{} a list of registered process names

The \code{get-registered} procedure returns a list of currently
registered process names from the registrar.

% ----------------------------------------------------------------------------
\subsection {Process Termination, Links, and Monitors}

% ----------------------------------------------------------------------------
\defineentry{exit}
\begin{procedure}
  \code{(exit \var{reason})}
\end{procedure}
\returns{} never

The \code{exit} procedure is the system \code{raise} procedure,
which raises an exception of \var{reason}. If the calling process does
not catch the exception, the process will exit with \var{reason}.

The system \code{exit} procedure is exported as
\code{scheme-exit}\index{scheme-exit@\code{scheme-exit}}.

% ----------------------------------------------------------------------------
\defineentry{catch}
\begin{syntax}
  \code{(catch \var{e1} \var{e2} \etc{})}
\end{syntax}
\expandsto{}\begin{alltt}\antipar
(call/1cc
 (lambda (return)
   (with-exception-handler
    (lambda (reason) (return `#(EXIT ,reason)))
    (lambda () \var{e1} \var{e2} \etc{}))))\end{alltt}

The \code{catch} macro evaluates expressions \var{e1} \var{e2}
\etc{} in a dynamic context that traps exceptions.  If no exception is
raised, the return value is the value of the last expression. If
exception \var{reason} is raised, \code{\#(EXIT \var{reason})} is
returned.

% ----------------------------------------------------------------------------
\defineentry{kill}
\begin{procedure}
  \code{(kill \var{process} \var{reason})}
\end{procedure}
\returns{} \code{\#t}

The \code{kill} procedure is used to terminate a process.
\begin{enumerate}
\item If \var{process} is not a process, exception \code{\#(bad-arg kill
  \var{process})} is raised.
\item If \var{process} has already exited, nothing happens.
\item If \var{reason} is \code{kill}, \var{process} is terminated
  with reason \code{killed}, even if it traps exits.
\item If \var{process} traps exits, message \code{\#(EXIT \var{self}
  \var{reason})} is sent to \var{process}, where \var{self} is the
  calling process.
\item If \var{process} does not trap exits and \var{reason} is
  \code{normal}, nothing happens.
\item Otherwise, \var{process} is terminated with \var{reason}.
\end{enumerate}

% ----------------------------------------------------------------------------
\defineentry{link}
\begin{procedure}
  \code{(link \var{process})}
\end{procedure}
\returns{} \code{\#t}

The \code{link} procedure creates a bi-directional link between the
calling process (\var{self}) and \var{process}. No more than one link
can exist between two processes, but it is not an error to call
\code{link} more than once on the same two processes.

\begin{enumerate}
\item If \var{process} is not a process, exception \code{\#(bad-arg link
  \var{process})} is raised.
\item If \var{process} is \var{self}, nothing happens.
\item If \var{process} has not exited, then if the two processes are
  already linked, nothing happens; otherwise, \var{self} is added to
  \var{process}.\code{links}, and \var{process} is added to
  \var{self}.\code{links}.
\item Otherwise, \var{process} has exited with reason $r$ =
  \var{process}.\code{exception-state}.
  \begin{enumerate}
  \item If \var{self} traps exits, message \code{\#(EXIT \var{process} $r$)}
    is sent to \var{self}.
  \item If \var{self} does not trap exits and \var{reason} is \code{normal},
    nothing happens.
  \item Otherwise, \var{self} is terminated with reason $r$.
  \end{enumerate}
\end{enumerate}

% ----------------------------------------------------------------------------
\defineentry{unlink}
\begin{procedure}
  \code{(unlink \var{process})}
\end{procedure}
\returns{} \code{\#t}

The \code{unlink} procedure removes the bi-directional link if
present between the calling process (\var{self}) and \var{process} by
removing \var{self} from \var{process}.\code{links} and
\var{process} from \var{self}.\code{links}.  If \var{process} is not
a process, exception \code{\#(bad-arg unlink \var{process})} is
raised.

% ----------------------------------------------------------------------------
\defineentry{monitor}
\begin{procedure}
  \code{(monitor \var{process})}
\end{procedure}
\returns{} a monitor

The \code{monitor} procedure creates and returns a new monitor $m$
with \code{origin} = the calling process (\var{self}) and
\code{target} = \var{process}.  Unlike \code{link},
\code{monitor} can create more than one connection between the same
processes.  It adds $m$ to \var{self}.\code{monitors} and
\var{process}.\code{monitors}.  When \var{process} exits or has
already exited with reason $r$, the message \code{\#(DOWN $m$
  \var{process} $r$)} is sent to \var{self}.  If \var{process} is not
a process, exception \code{\#(bad-arg monitor \var{process})} is
raised.

% ----------------------------------------------------------------------------
\defineentry{demonitor}
\begin{procedure}
  \code{(demonitor \var{monitor})}
\end{procedure}
\returns{} \code{\#t}

The \code{demonitor} procedure removes a \var{monitor} created by
the calling process (\var{self}) from \var{self}.\code{monitors} and
\var{monitor}.\code{target}.\code{monitors} if present.  If
\var{monitor} is not a monitor with \code{origin} = \var{self},
exception \code{\#(bad-arg demonitor \var{monitor})} is raised.

% ----------------------------------------------------------------------------
\defineentry{demonitor\&flush}
\begin{procedure}
  \code{(demonitor\&flush \var{monitor})}
\end{procedure}
\returns{} \code{\#t}

The \code{demonitor\&flush} procedure provides a convenient way to
demonitor and flush any remaining \code{DOWN} message from the
calling process's \code{inbox}. It performs the following operations:
\antipar\begin{alltt}
(demonitor \var{monitor})
(receive (until 0 #t)
  [#(DOWN ,@\var{monitor} ,_ ,_) #t])
\end{alltt}

% ----------------------------------------------------------------------------
\defineentry{monitor?}
\begin{procedure}
  \code{(monitor? \var{x})}
\end{procedure}
\returns{} a boolean

The \code{monitor?} procedure determines whether or not the datum
\var{x} is a monitor.

% ----------------------------------------------------------------------------
\subsection {Messages and Pattern Matching}

The pattern matching syntax of Figure~\ref{fig:erlang-pattern-grammar}
provides a concise and expressive way to match data structures and
bind variables to parts. The \code{receive}, \code{match}, and
\code{match-let*} macros use this pattern language. The
implementation uses two structurally recursive passes over the
pattern. The first pass checks the pattern syntax including record
types and field names and accumulates the list of pattern
variables. This list enables it to check for duplicate pattern
variables.  The second pass emits code that matches the input against
the pattern left to right.

% ----------------------------------------------------------------------------
\begin{figure}
\begin{tabular}{lp{3.6in}}
  pattern & matches \\ \hline

  \var{symbol} & itself \\
  \var{number} & itself \\
  \var{boolean} & itself \\
  \var{character} & itself \\
  \var{string} & itself \\
  \var{bytevector} & itself \\

  \code{()} & itself \\
  \code{($p_1$ . $p_2$)} & a pair whose car matches $p_1$ and cdr
  matches $p_2$ \\

  \code{\#($p_1$ \etc{} $p_n$)} & a vector of $n$ elements whose
  elements match $p_1$ \etc{} $p_n$ \\

  \code{,\_} & any datum \\
  \code{,\var{variable}} & any datum and binds a fresh \var{variable} to it \\
  \code{,@\var{variable}} & any datum \code{equal?} to the bound
  \var{variable} \\

  \code{,(\var{variable} <= \var{pattern})} & any datum that
  matches \var{pattern} and binds a fresh \var{variable} to it \\

  \code{`(\var{type} \set{,\var{field}\alt{}[\var{field} \var{pattern}]} \etc{})} &
  an instance of the record \var{type}, each \var{field} of which is
  bound to fresh variable \var{field} or matches the corresponding
  \var{pattern} \\

  \hline
\end{tabular}
\caption{Pattern Grammar\label{fig:erlang-pattern-grammar}}
\end{figure}

% ----------------------------------------------------------------------------
\defineentry{send}
\begin{procedure}
  \code{(send \var{destination} \var{message})}
\end{procedure}
\returns{} \var{message}

The \code{send} procedure sends \var{message} to a process or
registered name, \var{destination}.  If \var{destination} is not a
process or registered name, exception \code{\#(bad-arg send
  \var{destination})} is raised.  If \var{destination} has exited,
nothing else happens.  Otherwise, \var{message} is added to the end of
\var{destination}.\code{inbox}.  If \var{destination} is sleeping,
it is awakened.  If \var{destination} is not on the run
queue\index{run queue}, it is placed on the run queue with precedence
0.

% ----------------------------------------------------------------------------
\defineentry{receive}
\begin{syntax}
  \begin{alltt}
(receive
  \opt{(after \var{timeout} \var{t1} \var{t2} \etc{})\alt{}(until \var{time} \var{t1} \var{t2} \etc{})}
  (\nt{pattern} \opt{(guard \var{g})} \var{b1} \var{b2} \etc{})
  \etc{})\strut\end{alltt}
\end{syntax}
\returns{} the value of the last evaluated expression

The \code{receive} macro examines each message $m$ in the calling
process's \code{inbox} by testing it against each pattern and
optional guard. Each guard expression \var{g} is evaluated in the
scope of its associated pattern variables.  When \var{g} returns
\code{\#f}, $m$ fails to match that clause.  For the first pattern
and guard that matches $m$, $m$ is removed from \code{inbox}, and
the expressions \var{b1} \var{b2} \etc{} are evaluated in the scope of
its pattern variables.  If $m$ fails to match all patterns, the
examination continues with the next message in \code{inbox}. When
all messages have been examined, the calling process waits with its
\code{src} field set to the source location of the \code{receive}
macro if available. The process awakens when a new message or the time
specified by the optional \code{after} or \code{until} clause
arrives. If a new message arrives before the timeout, the examination
process continues as before. Otherwise, the timeout expressions
\var{t1} \var{t2} \etc{} are evaluated.

The optional \code{after} clause specifies a \var{timeout} in
milliseconds from the time at which control enters the
\code{receive} macro.  Similarly, the optional \code{until}
clause specifies a clock \var{time} in milliseconds as measured by
\code{erlang:now}.  In addition, \var{timeout} and \var{time} can be
\code{infinity} to indicate no timeout. If $t$ = \var{timeout} or
\var{time} is not a non-negative exact integer or \code{infinity},
exception \code{\#(timeout-value $t$ \var{src})} is returned, where
\var{src} is the source location of the \code{receive} macro if
available.

See Figure~\ref{fig:erlang-pattern-grammar} for the pattern grammar.

% ----------------------------------------------------------------------------
\defineentry{match}
\begin{syntax}\begin{alltt}
(match \var{exp}
  (\nt{pattern} \opt{(guard \var{g})} \var{b1} \var{b2} \etc{})
  \etc{})\strut\end{alltt}
\end{syntax}
\returns{} the value of the last expression \var{b1} \var{b2} \etc{}
for the matched pattern

The \code{match} macro evaluates \var{exp} once and tests its value
$v$ against each pattern and optional guard.  Each guard expression
\var{g} is evaluated in the scope of its associated pattern
variables. When \var{g} returns \code{\#f}, $v$ fails to match that
clause.  For the first pattern and guard that matches $v$, the
expressions \var{b1} \var{b2} \etc{} are evaluated in the scope of its
pattern variables. If $v$ fails to match all patterns, exception
\code{\#(bad-match $v$ \var{src})} is raised, where \var{src} is the
source location of the \code{match} clause if available.

See Figure~\ref{fig:erlang-pattern-grammar} for the pattern grammar.

% ----------------------------------------------------------------------------
\defineentry{match-let*}
\begin{syntax}\begin{alltt}
(match-let* ([\nt{pattern} \opt{(guard \var{g})} \var{exp}]
             \etc{})
  \var{b1} \var{b2} \etc{})\strut\end{alltt}
\end{syntax}
\returns{} the value of the last expression \var{b1} \var{b2} \etc{}

The \code{match-let*} macro evaluates each \var{exp} in the order
specified and matches its value against its pattern and guard.  The
pattern variables of each clause extend the scope of its guard
expression \var{g} and all subsequent pattern clauses and body
expressions \var{b1} \var{b2} \etc{}.  The \code{match-let*} macro
returns the value of the last body expression.  If any pattern fails
to match or any \var{g} returns \code{\#f}, exception
\code{\#(bad-match $v$ \var{src})} is raised, where $v$ is the datum
that failed to match the pattern or guard at source location \var{src}
if available.

See Figure~\ref{fig:erlang-pattern-grammar} for the pattern grammar.

% ----------------------------------------------------------------------------
\subsection {Process Properties}

% ----------------------------------------------------------------------------
\defineentry{self}
\begin{syntax}
  \code{self}
\end{syntax}
\returns{} the current process

The \code{self} macro uses \code{identifier-syntax} to expand into
code that retrieves the global self variable's top-level value. The
global variable cannot be used directly because library bindings are
immutable.

% ----------------------------------------------------------------------------
\defineentry{process?}
\begin{procedure}
  \code{(process? \var{x})}
\end{procedure}
\returns{} a boolean

The \code{process?} procedure determines whether or not the datum
\var{x} is a process.

% ----------------------------------------------------------------------------
\defineentry{process-id}
\begin{procedure}
  \code{(process-id \opt{\var{process}})}
\end{procedure}
\returns{} the process id

The \code{process-id} procedure returns \var{process}.\code{id},
where \var{process} defaults to \code{self}. If \var{process} is not
a process, exception \code{\#(bad-arg process-id \var{process})} is
raised.

% ----------------------------------------------------------------------------
\defineentry{process-trap-exit}
\begin{parameter}
  \code{process-trap-exit}
\end{parameter}
\hasvalue{} boolean

The \code{process-trap-exit} parameter specifies whether or not the
calling process traps exit signals as messages. Processes start with
this parameter set to \code{\#f}.

% ----------------------------------------------------------------------------
\defineentry{pps}
\begin{procedure}
  \code{(pps \opt{\var{op}})}
\end{procedure}
\returns{} unspecified

The \code{pps} procedure prints information about all processes to
\var{op}, which defaults to the current output port. If \var{op} is
not an output port, exception \code{\#(bad-arg pps \var{op})} is
raised.

% ----------------------------------------------------------------------------
\subsection{Miscellaneous}

% ----------------------------------------------------------------------------
\defineentry{add-finalizer}
\begin{procedure}
  \code{(add-finalizer \var{finalizer})}
\end{procedure}
\returns{} unspecified

The \code{add-finalizer}\index{finalizer} procedure adds
\var{finalizer} to the global list of finalizers. \var{finalizer} is a
procedure of no arguments that runs in the finalizer process after
garbage collections. If \var{finalizer} is not a procedure, exception
\code{\#(bad-arg add-finalizer \var{finalizer})} is raised.

% ----------------------------------------------------------------------------
\defineentry{bad-arg}
\begin{procedure}
  \code{(bad-arg \var{who} \var{arg})}
\end{procedure}
\returns{} never

The \code{bad-arg} procedure raises exception \code{\#(bad-arg
  \var{who} \var{arg})}.

% ----------------------------------------------------------------------------
\defineentry{console-event-handler}
\begin{procedure}
  \code{(console-event-handler \var{event})}
\end{procedure}
\returns{} unspecified

The \code{console-event-handler} procedure prints an \var{event} to
the console. It is used when the event manager is not available. It
disables interrupts so that it can be called from multiple processes
safely.  The output is designed to be machine readable. The output
looks like this:

\begin{alltt}
Date: Fri Aug 06 11:54:59 2010
Timestamp: 1281110099144
Event: \var{event}
\end{alltt}

The date is the local time from the \code{date-and-time} procedure,
the timestamp is the clock time from \code{erlang:now}, and
\var{event} is printed as with \code{write}.

% ----------------------------------------------------------------------------
\defineentry{dbg}
\begin{procedure}
  \code{(dbg)}\\
  \code{(dbg \var{id})}\\
  \code{(dbg \var{base} \var{proc})}\strut
\end{procedure}
\returns{} see below

The \code{dbg} procedure is used to debug processes that exit with a
continuation condition.

\code{(dbg)} prints to the current output port the process id and
exception message for each process that exited with a continuation
condition.

\code{(dbg \var{id})} enters the interactive debugger using the
exception state of process \var{id}. If process \var{id} does not
exist or did not exit with a continuation condition, the following
message is printed: ``Nothing to debug.''

\code{(dbg \var{base} \var{proc})} folds over the processes that
exited with a continuation condition and calls \var{proc} with the
process id, process exception state, and the accumulator value
(initially \var{base}). It returns the final accumulator value.

% ----------------------------------------------------------------------------
\defineentry{erlang:now}
\begin{procedure}
  \code{(erlang:now)}
\end{procedure}
\returns{} the current clock time in milliseconds

The \code{erlang:now} procedure calls \code{osi::GetTickCount} to
return the number of milliseconds in UTC since the UNIX epoch January
1, 1970.

% -------------------------------------------------------------------------
\defineentry{make-process-parameter}
\begin{procedure}
  \code{(make-process-parameter \var{initial} \opt{\var{filter}})}
\end{procedure}
\returns{} a process parameter procedure

The \code{make-process-parameter} procedure creates a parameter
procedure $p$ that provides per-process, mutable storage using a weak
eq-hashtable mapping processes to values. Calling $p$ with no
arguments returns the current value of the parameter for the calling
process, and calling $p$ with one argument sets the value of the
parameter for the calling process. The \var{filter}, if present, is a
procedure of one argument that is applied to the \var{initial} and all
subsequent values. If \var{filter} is not a procedure, exception
\code{\#(bad-arg make-process-parameter \var{filter})} is raised.

Because the parameters used by the system formatting and printing
  routines are not process safe, they are redefined to use
  \code{make-process-parameter}. These system parameters are
\code{pretty-initial-indent},
\code{pretty-line-length},
\code{pretty-maximum-lines},
\code{pretty-one-line-limit},
\code{pretty-standard-indent},
\code{print-brackets},
\code{print-char-name},
\code{print-gensym},
\code{print-graph},
\code{print-length},
\code{print-level},
\code{print-precision},
\code{print-radix},
\code{print-record},
\code{print-unicode}, and
\code{print-vector-length}.
The \code{custom-port-buffer-size} parameter is also redefined to be
process safe.

% -------------------------------------------------------------------------
\defineentry{on-exit}
\begin{syntax}
  \code{(on-exit \var{finally} \var{b1} \var{b2} \etc{})}
\end{syntax}
\expandsto{}\begin{alltt}\antipar
(dynamic-wind
  void
  (lambda () \var{b1} \var{b2} \etc{})
  (lambda () \var{finally}))\end{alltt}

The \code{on-exit} macro executes the body expressions \var{b1}
\var{b2} \etc{} in a dynamic context that executes the \var{finally}
expression whenever control leaves the body.

% ----------------------------------------------------------------------------
\defineentry{profile-me}
\begin{procedure}
  \code{(profile-me)}
\end{procedure}
\returns{} unspecified

The \code{profile-me} procedure does nothing but provide a
place-holder for the system profiler to count the call site.  When
profiling is turned off, \code{(profile-me)} expands to
\code{(void)}, and the system optimizer eliminates it.

% ----------------------------------------------------------------------------
\subsection {Records}

For users of the concurrency model, a \emph{record}\index{record} is a
container of named, immutable fields implemented as a vector whose
first element is the record name and remaining elements are the
fields. Each record definition is a macro that provides all record
operations using field names only, not field indices. The macro makes
it easy to copy a record without having to specify the fields that
don't change.  We decided not to use the Scheme record facility
because it does not provide name-based constructors, copy operators,
or convenient serialization.

% ----------------------------------------------------------------------------
\index{record!define-record@\code{define-record}}
\defineentry{define-record}
\begin{syntax}
  \code{(define-record \var{name} \var{field} \etc{})}
\end{syntax}
\expandsto{} a macro definition of \var{name} described below

The \code{define-record} macro defines a macro for creating,
copying, identifying, and accessing record type \var{name}. \var{name}
and \var{field} \etc{} must be identifiers. No two field names can be
the same. The following field names are reserved: \code{make},
\code{copy}, \code{copy*}, and \code{is?}.

% ----------------------------------------------------------------------------
\index{record!make@\code{make}}
\begin{syntax}
  \code{(\var{name} make [\var{field} \var{value}] \etc{})}
\end{syntax}
\returns{} a new instance of record type \var{name} with \var{field} =
\var{value} \etc{}

The \code{make} form creates a new instance of the record type
\var{name}. \var{field} bindings may appear in any order. All fields
from the record definition must be specified.

% ----------------------------------------------------------------------------
\index{record!field accessor}
\begin{syntax}
  \code{(\var{name} \var{field} \var{instance})}
\end{syntax}
\returns{} \var{instance}.\var{field}

The field accessor form retrieves the value of the specified
\var{field} of \var{instance}. If $r$ = \var{instance} is not a record
of type \var{name}, exception \code{\#(bad-record \var{name} $r$
  \var{src})} is raised, where \var{src} is the source location of the
field accessor form if available.

\index{record!field accessor}
\begin{syntax}
  \code{(\var{name} \var{field})}
\end{syntax}
\returns{} a procedure that, given \var{instance}, returns \var{instance}.\var{field}

The \code{(\var{name} \var{field})} form expands to \code{(lambda (instance) (\var{name} \var{field} instance))}.

% ----------------------------------------------------------------------------
\index{record!open@\code{open}}
\begin{syntax}
  \code{(\var{name} open \var{instance} \opt{\var{prefix}} (\var{field} \etc{}))}
\end{syntax}
\expandsto{} definitions for \var{field} \etc{} or \var{prefix}\var{field} \etc{} described below

The \code{open} form defines identifier syntax for each specified \var{field}
so that a reference to \var{field} expands to \code{(\var{name} \var{field} $r$)}
where $r$ is the value of \var{instance}.
If $r$ is not a record of type \var{name}, exception
\code{\#(bad-record \var{name} $r$ \var{src})} is raised, where
\var{src} is the source location of the \code{open} form if avalable.
The \code{open} form is equivalent to the following, except that it
checks the record type only once:
\antipar\codebegin
(begin
  (define instance \var{instance})
  (define-syntax \var{field} (identifier-syntax (\var{name} \var{field} instance)))
  \etc{})
\codeend\antipar
The \code{open} form introduces definitions only for fields listed explicitly in \code{(\var{field} \etc{})}.
If the optional \var{prefix} identifier is supplied, \code{open} produces
a definition for \code{\var{prefix}\var{field}} rather than \var{field} for
each \var{field} specified.

% ----------------------------------------------------------------------------
\index{record!copy@\code{copy}}
\begin{syntax}
  \code{(\var{name} copy \var{instance} [\var{field} \var{value}] \etc{})}
\end{syntax}
\returns{} a new instance of record type \var{name} with \var{field} =
\var{value} \etc{} and remaining fields copied from \var{instance}

The \code{copy} form creates a copy of \var{instance} except that
each specified \var{field} is set to the associated \var{value}. If
$r$ = \var{instance} is not a record of type \var{name}, exception
\code{\#(bad-record \var{name} $r$ \var{src})} is raised, where
\var{src} is the source location of the \code{copy} form if
avalable.  \var{field} bindings may appear in any order.

% ----------------------------------------------------------------------------
\index{record!copy*@\code{copy*}}
\begin{syntax}
  \code{(\var{name} copy* \var{instance} [\var{field} \var{value}] \etc{})}
\end{syntax}
\returns{} a new instance of record type \var{name} with \var{field} =
\var{value} \etc{} and remaining fields copied from \var{instance}

The \code{copy*} form is like \code{copy} except that, within the \var{value} expressions,
each specified \var{field} is bound to an identifier macro that returns the value of \var{instance.field}.
If $r$ = \var{instance} is not a record of type \var{name}, exception
\code{\#(bad-record \var{name} $r$ \var{src})} is raised, where
\var{src} is the source location of the \code{copy*} form if avalable.
The \code{copy*} form is equivalent to the following, except that it
checks the record type only once:
\antipar\codebegin
(let ([instance \var{instance}])
  (\var{name} open instance (\var{field} \etc{}))
  (\var{name} copy instance [\var{field} \var{value}] \etc{}))
\codeend\antipar

% ----------------------------------------------------------------------------
\index{record!is?@\code{is?}}
\begin{syntax}
  \code{(\var{name} is? \var{x})}
\end{syntax}
\returns{} a boolean

The \code{is?} form determines whether or not the datum \var{x} is
an instance of record type \var{name}.

% ----------------------------------------------------------------------------
\subsection {I/O}

% ----------------------------------------------------------------------------
\defineentry{make-utf8-transcoder}
\begin{procedure}
  \code{(make-utf8-transcoder)}
\end{procedure}
\returns{} a UTF-8 transcoder

The \code{make-utf8-transcoder} procedure creates a UTF-8 transcoder
with end-of-line style \code{none} and error-handling mode
\code{raise}.

% ----------------------------------------------------------------------------
\defineentry{binary->utf8}
\begin{procedure}
  \code{(binary->utf8 \var{bp})}
\end{procedure}
\returns{} a transcoded textual port wrapping \var{bp}

The \code{binary->utf8} procedure takes a binary port \var{bp} and
returns a textual port wrapping \var{bp} using
\code{transcoded-port} and \code{(make-utf8-transcoder)}. The
original port \var{bp} is marked closed so that it cannot be used
except through the associated textual port.

% ----------------------------------------------------------------------------
\defineentry{force-close-output-port}
\begin{procedure}
  \code{(force-close-output-port \var{op})}
\end{procedure}
\returns{} unspecified

The \code{force-close-output-port} procedure is used to close a
custom output port, even if it has unflushed output that would
otherwise cause it to fail to close.  If \var{op} is not already
closed, \code{force-close-output-port} tries to close it with
\code{(close-output-port \var{op})}. If it fails with exception
\code{\#(io-error \var{name} WritePort \var{errno})}, the output
buffer is cleared with \code{(clear-output-port \var{op})}, and
\code{(close-output-port \var{op})} is called again.

% ----------------------------------------------------------------------------
\defineentry{io-error}
\begin{procedure}
  \code{(io-error \var{name} \var{who} \var{errno})}
\end{procedure}
\returns{} never

The \code{io-error} procedure raises exception \code{\#(io-error
  \var{name} \var{who} \var{errno})}.  The string \var{name}
identifies the file, pipe, COM port, process, console input, or TCP
port.  The symbol \var{who} specifies the procedure that raised an
error, and the number \var{errno} specifies the error code.  The
\code{read-osi-port} procedure raises this exception with
\var{who}=\code{ReadPort}, and the \code{write-osi-port} procedure
raises it with \var{who}=\code{WritePort}.

% ----------------------------------------------------------------------------
\defineentry{connect-usb}
\begin{procedure}
  \code{(connect-usb \var{device-name} \var{read-address} \var{write-address})}
\end{procedure}
\returns{} two values: a binary input port and a binary output port

The \code{connect-usb} procedure calls \code{osi::ConnectWinUSB} to
establish a USB connection with the given \var{device-name}. It
returns a custom binary input port that reads from the
\var{read-address} endpoint and a custom binary output port that
writes to the \var{write-address} endpoint. These ports do not track
or report position. The operating system handle is closed when the
output port is closed, not when the input port is closed, and the
underlying osi-port is registered with the osi-port
guardian\index{osi-port guardian}.

% ----------------------------------------------------------------------------
\defineentry{create-server-pipe}
\begin{procedure}
  \code{(create-server-pipe \var{name} \var{callback})}
\end{procedure}
\returns{} two values: a binary input port and a binary output port

The \code{create-server-pipe} procedure calls
\code{osi::CreateServerPipe(\var{full-name}, \var{callback})} to
create a server pipe with
\var{full-name}=\verb|\\.\pipe\|\var{name}. It returns a custom binary
input port that reads from the pipe and a custom binary output port
that writes to the pipe. These ports do not track or report
position. The operating system handle is closed when the output port
is closed, not when the input port is closed, and the underlying
osi-port is registered with the osi-port guardian\index{osi-port
  guardian}.

The \var{callback} procedure is called with two arguments, \var{count}
and \var{errno}, from the event loop after a client pipe has
connected. Before this occurs, I/O operations on the server pipe will
fail with error code 536.

If \code{osi::CreateServerPipe} returns error pair
\code{(\var{who} . \var{errno})}, exception \code{\#(io-error
  \var{full-name} \var{who} \var{errno})} is raised.

% ----------------------------------------------------------------------------
\defineentry{create-client-pipe}
\begin{procedure}
  \code{(create-client-pipe \var{name})}
\end{procedure}
\returns{} two values: a binary input port and a binary output port

The \code{create-client-pipe} procedure calls
\code{osi::CreateClientPipe(\var{full-name})} to create a client
pipe connection to \var{full-name}=\verb|\\.\pipe\|\var{name}. It
returns a custom binary input port that reads from the pipe and a
custom binary output port that writes to the pipe. These ports do not
track or report position. The operating system handle is closed when
the output port is closed, not when the input port is closed, and the
underlying osi-port is registered with the osi-port
guardian\index{osi-port guardian}.

If \code{osi::CreateClientPipe} returns error pair
\code{(\var{who} . \var{errno})}, exception \code{\#(io-error
  \var{full-name} \var{who} \var{errno})} is raised.

% ----------------------------------------------------------------------------
\defineentry{create-watched-process}
\begin{procedure}
  \code{(create-watched-process \var{command-line} \var{callback})}
\end{procedure}
\returns{} three values: a process handle, a binary input port, and a
binary output port

The \code{create-watched-process} procedure calls
\code{osi::CreateWatchedProcess(\var{command-line}, \var{callback})}
to spawn an operating system process. It returns a process handle that
can be used with \code{osi::TerminateProcess}, a custom binary input
port that reads from the standard output and standard error of the
process, and a custom binary output port that writes to the standard
input of the process. These ports do not track or report position. The
operating system handle for the input port is closed when the input
port is closed, and likewise for the output port. Both underlying
osi-ports are registered with the osi-port guardian\index{osi-port
  guardian}.

The \var{callback} procedure is called with two arguments, the process
handle and the exit code, from the event loop after the operating
system process exits.

If \code{osi::CreateWatchedProcess} returns error pair
\code{(\var{who} . \var{errno})}, exception
\code{\#(create-watched-process-failed \var{command-line} \var{who}
  \var{errno})} is raised.

% ----------------------------------------------------------------------------
\defineentry{absolute-path}
\begin{procedure}
  \code{(absolute-path \var{path} \var{root})}
\end{procedure}
\returns{} the full, absolute path of \var{path} with respect to \var{root}

If \var{path} is an absolute path, the \code{absolute-path}
procedure returns the full path of \var{path} using
\code{osi::GetFullPath}. Otherwise, it returns the full path of
\code{(path-combine \var{root} \var{path})}.

% ----------------------------------------------------------------------------
\defineentry{create-directory-path}
\begin{procedure}
  \code{(create-directory-path \var{path})}
\end{procedure}
\returns{} \var{path}

The \code{create-directory-path} procedure creates directories as
needed for the file \var{path}. It calls
\code{osi::GetFullPath(\var{path})} so that relative specifiers such
as \code{..} work properly. It does not raise an exception if
\code{osi::CreateDirectory} fails.

% ----------------------------------------------------------------------------
\defineentry{create-file-port}
\begin{procedure}
  \code{(create-file-port \var{name} \var{desired-access}
    \var{share-mode} \var{creation-disposition})}
\end{procedure}
\returns{} an osi-port

The \code{create-file-port} procedure creates an osi-port by calling
\code{osi::CreateFile(\var{name}, \var{desired-access},
  \var{share-mode}, \var{creation-disposition})}. The osi-port is
registered with the osi-port guardian\index{osi-port guardian}.

If \code{osi::CreateFile} returns error pair \code{(\var{who}
  . \var{errno})}, exception \code{\#(io-error \var{name} \var{who}
  \var{errno})} is raised.

The constants \code{GENERIC\_READ} and \code{GENERIC\_WRITE} can
be used for \var{desired-access}.

The constants \code{FILE\_SHARE\_NONE}, \code{FILE\_SHARE\_READ},
\code{FILE\_SHARE\_WRITE}, and \code{FILE\_SHARE\_DELETE} can be
used for \var{share-mode}.

The constants \code{CREATE\_ALWAYS}, \code{CREATE\_NEW},
\code{OPEN\_ALWAYS}, \code{OPEN\_EXISTING}, and
\code{TRUNCATE\_EXISTING} can be used for
\var{creation-disposition}.

% ----------------------------------------------------------------------------
\defineentry{read-osi-port}
\begin{procedure}
  \code{(read-osi-port \var{port} \var{bv} \var{start} \var{n} \var{fp})}
\end{procedure}
\returns{} the number of bytes read

The \code{read-osi-port} procedure calls \code{osi::ReadPort} with
the handle from the given osi-port \var{port}, bytevector buffer
\var{bv}, starting 0-based buffer index \var{start}, maximum number of
bytes to read \var{n}, starting 0-based file position \var{fp}, and a
callback that fires in the event loop. The callback sends a message to
the calling process, which waits until it receives it. If the read
fails, exception \code{\#(io-error \var{name} ReadPort \var{errno})}
is raised, where \var{name} is the name of \var{port} and \var{errno}
is the error number. Otherwise, the number of bytes read is returned.

Error codes 38 (end of file), 109 (pipe closed), 995 (thread exit),
and 10053 (connection aborted) are treated as end of file and cause a
0 to be returned.

% ----------------------------------------------------------------------------
\defineentry{write-osi-port}
\begin{procedure}
  \code{(write-osi-port \var{port} \var{bv} \var{start} \var{n} \var{fp})}
\end{procedure}
\returns{} the number of bytes written

The \code{write-osi-port} procedure calls \code{osi::WritePort}
with the handle from the given osi-port \var{port}, bytevector buffer
\var{bv}, starting 0-based buffer index \var{start}, maximum number of
bytes to write \var{n}, starting 0-based file position \var{fp}, and a
callback that fires in the event loop. The callback sends a message to
the calling process, which waits until it receives it. If the write
fails, exception \code{\#(io-error \var{name} WritePort
  \var{errno})} is raised, where \var{name} is the name of \var{port}
and \var{errno} is the error number. Otherwise, the number of bytes
written is returned.

% ----------------------------------------------------------------------------
\defineentry{close-osi-port}
\begin{procedure}
  \code{(close-osi-port \var{port})}
\end{procedure}
\returns{} unspecified

The \code{close-osi-port} procedure closes osi-port \var{port} using
\code{osi::ClosePort}. If \var{port} has already been closed,
\code{close-osi-port} does not raise an exception.

% ----------------------------------------------------------------------------
\defineentry{get-file-size}
\begin{procedure}
  \code{(get-file-size \var{port})}
\end{procedure}
\returns{} the number of bytes in the file associated with osi-port \var{port}

The \code{get-file-size} procedure calls \code{osi::GetFileSize}
to return the number of bytes in the file associated with osi-port
\var{port}.

If \code{osi::GetFileSize} returns error pair \code{(\var{who}
  . \var{errno})}, exception \code{\#(io-error \var{filename}
  \var{who} \var{errno})} is raised.

% ----------------------------------------------------------------------------
\defineentry{create-file}
\begin{procedure}
  \code{(create-file \var{name} \var{desired-access} \var{share-mode}
    \var{creation-disposition} \var{type})}
\end{procedure}
\returns{} a custom file port

The \code{create-file} procedure creates a custom file port by
calling \code{(create-file-port \var{name} \var{desired-access}
  \var{share-mode} \var{creation-disposition})}. The custom port
supports both getting and setting the file position. The particular
type of custom port returned is determined by \var{type}:
\begin{itemize}
  \item \code{binary-input}: a binary input port
  \item \code{binary-output}: a binary output port
  \item \code{input}: a textual input port wrapping a binary input
    port with \code{binary->utf8}
  \item \code{output}: a textual output port wrapping a binary
    output port with \code{binary->utf8}
  \item \code{append}: a textual output port wrapping a binary
    output port with \code{binary->utf8} and initially positioned at
    the end of the file (determined by \code{osi::GetFileSize})
\end{itemize}

If \var{type} is any other value, exception \code{\#(bad-arg
  create-file \var{type})} is raised.

% ----------------------------------------------------------------------------
\defineentry{find-files}
\begin{procedure}
  \code{(find-files \var{spec})}
\end{procedure}
\returns{} a list of files and directories that match \var{spec}

The \code{find-files} procedure calls \code{osi::FindFiles} with a
callback that fires in the event loop.  The callback sends a message
to the calling process, which waits until it receives the message and
returns the list of files and directories that match \var{spec}. If
\code{osi::FindFiles} or the message indicates error pair
\code{(\var{who} . \var{errno})}, exception
\code{\#(find-files-failed \var{spec} \var{who} \var{errno})} is
raised.

% ----------------------------------------------------------------------------
\defineentry{hook-console-input}
\begin{procedure}
  \code{(hook-console-input)}
\end{procedure}
\returns{} unspecified

The \code{hook-console-input} procedure replaces the system console
input port, which uses synchronous I/O, with a custom textual input
port that uses asynchronous I/O. It builds a custom binary input port
with \code{osi::OpenConsole}, wraps it with \code{binary->utf8},
and sets the result as the \code{console-input-port},
\code{current-input-port}, and the system internal
\code{\$console-input-port}. It does nothing after it has been
called once.

% ----------------------------------------------------------------------------
\defineentry{move-file}
\begin{procedure}
  \code{(move-file \var{old-pathname} \var{new-pathname} \opt{\var{option}})}
\end{procedure}
\returns{} \code{\#t}

The \code{move-file} procedure calls
\code{osi::MoveFile}. \var{option} may be either \code{error} or
\code{replace}. A value of \code{replace} causes
\var{new-pathname} to be replaced if it exists. The default is
\code{error}.

If \var{old-pathname} is not a string, exception \code{\#(bad-arg
  move-file \var{old-pathname})} is raised.  If \var{new-pathname} is
not a string, exception \code{\#(bad-arg move-file
  \var{new-pathname})} is raised.  If \var{option} is not
\code{error} or \code{replace}, exception \code{\#(bad-arg
  move-file \var{option})} is raised.

% ----------------------------------------------------------------------------
\defineentry{open-file-to-append}
\begin{procedure}
  \code{(open-file-to-append \var{name})}
\end{procedure}
\returns{} a custom file port

The \code{open-file-to-append} procedure calls \code{(create-file
  \var{name} GENERIC\_WRITE FILE\_SHARE\_READ OPEN\_ALWAYS 'append)}.

% ----------------------------------------------------------------------------
\defineentry{open-file-to-read}
\begin{procedure}
  \code{(open-file-to-read \var{name})}
\end{procedure}
\returns{} a custom file port

The \code{open-file-to-read} procedure calls \code{(create-file
  \var{name} GENERIC\_READ FILE\_SHARE\_READ OPEN\_EXISTING 'input)}.

% ----------------------------------------------------------------------------
\defineentry{open-file-to-replace}
\begin{procedure}
  \code{(open-file-to-replace \var{name})}
\end{procedure}
\returns{} a custom file port

The \code{open-file-to-replace} procedure calls \code{(create-file
  \var{name} GENERIC\_WRITE FILE\_SHARE\_READ CREATE\_ALWAYS 'output)}.

% ----------------------------------------------------------------------------
\defineentry{open-file-to-write}
\begin{procedure}
  \code{(open-file-to-write \var{name})}
\end{procedure}
\returns{} a custom file port

The \code{open-file-to-write} procedure calls \code{(create-file
  \var{name} GENERIC\_WRITE FILE\_SHARE\_READ CREATE\_NEW 'output)}.

% ----------------------------------------------------------------------------
\defineentry{open-utf8-bytevector}
\begin{procedure}
  \code{(open-utf8-bytevector \var{bv})}
\end{procedure}
\returns{} a transcoded textual input port wrapping \var{bv}

The \code{open-utf8-bytevector} procedure calls
\code{(binary->utf8 (open-bytevector-input-port \var{bv}))}.

% ----------------------------------------------------------------------------
\defineentry{path-combine}
\begin{procedure}
  \code{(path-combine \var{path$_1$} \var{path$_2$} \etc{})}
\end{procedure}
\returns{} the string combining the paths

The \code{path-combine} procedure appends one or more paths,
inserting the \verb|\| character between each pair of paths when the
first path of the pair does not end with it.

% ----------------------------------------------------------------------------
\defineentry{read-bytevector}
\begin{procedure}
  \code{(read-bytevector \var{name} \var{contents})}
\end{procedure}
\returns{} a list of annotations

The \code{read-bytevector} procedure takes a filename \var{name} and
\var{contents} bytevector and returns a list of annotations read using
\code{get-datum/annotations} from the \var{contents} bytevector
transcoded with the UTF-8 transcoder described in
\code{binary->utf8}.

% ----------------------------------------------------------------------------
\defineentry{read-file}
\begin{procedure}
  \code{(read-file \var{name})}
\end{procedure}
\returns{} a bytevector with the contents of \var{name}

The \code{read-file} procedure calls \code{(create-file-port
  \var{name} GENERIC\_READ FILE\_SHARE\_READ OPEN\_EXISTING)} to open
the file \var{name} and returns the contents as a bytevector.

% ----------------------------------------------------------------------------
\defineentry{watch-directory}
\begin{procedure}
  \code{(watch-directory \var{path} \var{subtree?} \var{callback})}
\end{procedure}
\returns{} a directory watcher\index{directory watcher}

The \code{watch-directory} procedure calls
\code{osi::WatchDirectory} to track changes in the directory
\var{path} and returns a directory watcher that is registered with the
directory-watcher guardian\index{directory-watcher guardian}. When
\var{subtree?} is true, the directory tree rooted at \var{path} is
watched. The change filter is configured to detect creating, renaming,
and deleting files and directories as well as changes to the size and
last-write time of files. Changes to file attributes, last-access and
creation times, and security attributes are not included.

Every time changes are detected, \code{(\var{callback}
  ((\var{action} . \var{filename}) \etc))} is executed in the event
loop. The \var{action} number is defined as follows:

\begin{center}\begin{tabular}{ll}
  \var{action} & Meaning\\
  \hline
  1 & \var{filename} was added to the directory.\\
  2 & \var{filename} was removed from the directory.\\
  3 & \var{filename} was modified.\\
  4 & \var{filename} was renamed, and \var{filename} is the old
  name.\\
  5 & \var{filename} was renamed, and \var{filename} is the new
  name.\\
\end{tabular}\end{center}

When the directory watcher stops, it executes \code(\var{callback}
\var{errno}) in the event loop, and no more callbacks from this
watcher will be executed. An \var{errno} of zero indicates a normal
close from \code{close-directory-watcher}. A non-zero \var{errno}
indicates an error from \code{ReadDirectoryChangesW}.

If \code{osi::WatchDirectory} returns error pair \code{(\var{who}
  . \var{errno})}, exception \code{\#(watch-directory-failed
  \var{path} \var{who} \var{errno})} is raised.

% ----------------------------------------------------------------------------
\defineentry{close-directory-watcher}
\begin{procedure}
  \code{(close-directory-watcher \var{watcher})}
\end{procedure}
\returns{} unspecified

The \code{close-directory-watcher} procedure uses
\code{osi::CloseDirectoryWatcher} to close the given directory
\var{watcher}\index{directory watcher}. If the watcher has not already
stopped, the associated callback will be executed at least one more
time. If \var{watcher} is not a directory watcher, exception
\code{\#(bad-arg close-directory-watcher \var{watcher})} is
raised. If \var{watcher} has already been closed,
\code{close-directory-watcher} does not raise an exception.

% ----------------------------------------------------------------------------
\defineentry{directory-watcher-path}
\begin{procedure}
  \code{(directory-watcher-path \var{watcher})}
\end{procedure}
\returns{} the \code{path} field of \var{watcher}

The \code{directory-watcher-path} procedure returns the
\code{path} of the given directory \var{watcher}\index{directory
  watcher}.

% ----------------------------------------------------------------------------
\defineentry{listen-tcp}
\begin{procedure}
  \code{(listen-tcp \var{address} \var{port-number} \var{process})}
\end{procedure}
\returns{} a TCP listener\index{TCP listener}

The \code{listen-tcp} procedure calls \code{osi\_listen\_tcp} to
create a TCP listener on the given \var{address} and \var{port-number}
and returns a TCP listener that is registered with the listener
guardian\index{listener guardian}.

For each accepted connection, the message \code{\#(accept-tcp
  \var{listener} \var{ip} \var{op})} is sent to \var{process}, where
\var{ip} is the input port and \var{op} is the output port.

For each failed connection, the message \code{\#(accept-tcp-failed
  \var{listener} \var{who} \var{errno})} is sent to \var{process},
where \var{who} and \var{errno} specify the error.

The \var{address} is a dotted quad IPv4 address or an IPv6
address. Use \code{"[::]"} to listen on all IPv4 and IPv6 interfaces.
If \var{address} is not a string, exception \code{\#(bad-arg
  listen-tcp \var{address})} is raised.

If \var{port-number} is zero, the operating system will choose an
available port number, which can be queried with
\code{listener-port-number}. If \var{port-number} is not a fixnum
between 0 and 65535 inclusive, exception \code{\#(bad-arg listen-tcp
  \var{port-number})} is raised.

If \code{osi\_listen\_tcp} returns error pair \code{(\var{who}
  . \var{errno})}, exception \code{\#(listen-tcp-failed \var{address}
  \var{port-number} \var{who} \var{errno})} is raised.

% ----------------------------------------------------------------------------
\defineentry{close-tcp-listener}
\begin{procedure}
  \code{(close-tcp-listener \var{listener})}
\end{procedure}
\returns{} unspecified

The \code{close-tcp-listener} procedure closes a TCP
\var{listener}\index{TCP listener} using
\code{osi::CloseTCPListener}.  If \var{listener} is not a TCP
listener, exception \code{\#(bad-arg close-tcp-listener
  \var{listener})} is raised. If \var{listener} has already been
closed, \code{close-tcp-listener} does not raise an exception.

% ----------------------------------------------------------------------------
\defineentry{listener-address}
\begin{procedure}
  \code{(listener-address \var{listener})}
\end{procedure}
\returns{} the \code{address} field of \var{listener}

The \code{listener-address} procedure returns the \code{address} of
the given TCP \var{listener}\index{TCP listener}.

% ----------------------------------------------------------------------------
\defineentry{listener-port-number}
\begin{procedure}
  \code{(listener-port-number \var{listener})}
\end{procedure}
\returns{} the \code{port-number} field of \var{listener}

The \code{listener-port-number} procedure returns the
\code{port-number} of the given TCP \var{listener}\index{TCP
  listener}.

% ----------------------------------------------------------------------------
\defineentry{connect-tcp}
\begin{procedure}
  \code{(connect-tcp \var{hostname} \var{port-spec} \opt{\var{process}})}
\end{procedure}
\returns{} see below

The \code{connect-tcp} procedure calls \code{osi::ConnectTCP} with
a callback that fires in the event loop when the TCP connection to
\var{hostname} on \var{port-spec} is established or fails to be
established.  The \var{port-spec} may be a fixnum port number or a
string service name such as ``http''. The callback creates a custom
binary input port that reads from the new connection and a custom
binary output port that writes to the new connection. These ports do
not track or report position. The operating system handle is closed
when the output port is closed, not when the input port is closed, and
the underlying osi-port is registered with the osi-port
guardian\index{osi-port guardian}.  The callback sends one of the
following messages to \var{process}, which defaults to \code{self}:

\begin{itemize}
\item \code{\#(connect-tcp \var{hostname} \var{port-spec} \var{ip}
  \var{op})}, where \var{ip} is the input port and \var{op} is the
  output port
\item \code{\#(connect-tcp-failed \var{hostname} \var{port-spec}
  \var{who} \var{errno})}, where \var{who} and \var{errno} specify the
  error from the callback to \code{osi::ConnectTCP}
\end{itemize}

When \var{process} is present, \code{connect-tcp} returns
\code{\#t} without waiting for the callback.

When \var{process} is absent, \code{connect-tcp} waits to receive a
message from the callback. If it receives the \code{connect-tcp}
message, it returns two values, \var{ip} and \var{op}. If it receives
the \code{connect-tcp-failed} message, it raises exception
\code{\#(connect-tcp-failed \var{hostname} \var{port-spec} \var{who}
  \var{errno})}.

If \var{hostname} is not a string, exception \code{\#(bad-arg
  connect-tcp \var{hostname})} is raised. If \var{port-spec} is not a
fixnum between 0 and 65535 inclusive or a string, exception
\code{\#(bad-arg connect-tcp \var{port-spec})} is raised.

% ----------------------------------------------------------------------------
\subsection {Queues}

A queue is represented as a pair of lists,
\code{(\var{in}~.~\var{out})}. The \var{out} list contains the first
elements of the queue, and the \var{in} list contains the last
elements of the queue in reverse. This representation allows for O(1)
amortized insertion and removal times. The implementation is based on
the Erlang queue module~\cite{queue-ref}.

% ----------------------------------------------------------------------------
\defineentry{queue:add}
\begin{procedure}
  \code{(queue:add \var{x} \var{q})}
\end{procedure}
\returns{} a queue that adds \var{x} to the rear of \var{q}

% ----------------------------------------------------------------------------
\defineentry{queue:add-front}
\begin{procedure}
  \code{(queue:add-front \var{x} \var{q})}
\end{procedure}
\returns{} a queue that adds \var{x} to the front of \var{q}

% ----------------------------------------------------------------------------
\defineentry{queue:drop}
\begin{procedure}
  \code{(queue:drop \var{q})}
\end{procedure}
\returns{} a queue without the first element of \var{q}

% ----------------------------------------------------------------------------
\defineentry{queue:empty}
\begin{syntax}
  \code{queue:empty}
\end{syntax}
\returns{} the empty queue

% ----------------------------------------------------------------------------
\defineentry{queue:empty?}
\begin{procedure}
  \code{(queue:empty? \var{q})}
\end{procedure}
\returns{} \code{\#t} if \var{q} is a queue, \code{\#f} otherwise

% ----------------------------------------------------------------------------
\defineentry{queue:get}
\begin{procedure}
  \code{(queue:get \var{q})}
\end{procedure}
\returns{} the first element of \var{q}

% ----------------------------------------------------------------------------
\subsection{Hash Tables}

The implementation of functional hash tables is based on the Erlang
dict module~\cite{dict-ref,dynamic-hashing}.

% ----------------------------------------------------------------------------
\defineentry{ht:delete}
\begin{procedure}
  \code{(ht:delete \var{ht} \var{key})}
\end{procedure}
\returns{} a hash table formed by dropping any association of
\var{key} from \var{ht}

% ----------------------------------------------------------------------------
\defineentry{ht:fold}
\begin{procedure}
  \code{(ht:fold \var{ht} \var{f} \var{init})}
\end{procedure}
\returns{} see below

The \code{ht:fold} procedure accumulates a value by applying \var{f}
to each key/value association in \var{ht} and the accumulator, which
is initially \var{init}. It can be defined recursively as follows,
where $n$ is the size of \var{ht}, and the result of \code{ht:fold}
is $F_n$:

$$\begin{array}{r@{\;=\;}l}
F_0 & \var{init}\\
F_i & \code{(\var{f} \var{key$_i$} \var{val$_i$} $F_{i-1}$)}
\hbox{ for $1 \le i \le n$}\\
\end{array}$$

% ----------------------------------------------------------------------------
\defineentry{ht:is?}
\begin{procedure}
  \code{(ht:is? \var{x})}
\end{procedure}
\returns{} \code{\#t} if \var{x} is a hash table, \code{\#f}
otherwise

% ----------------------------------------------------------------------------
\defineentry{ht:keys}
\begin{procedure}
  \code{(ht:keys \var{ht})}
\end{procedure}
\returns{} a list of the keys of \var{ht}

% ----------------------------------------------------------------------------
\defineentry{ht:make}
\begin{procedure}
  \code{(ht:make \var{hash-key} \var{equal-key?} \var{valid-key?})}
\end{procedure}
\returns{} an empty hash table

The \code{ht:make} procedure returns an empty hash table.

The \var{hash-key} procedure takes a key and returns an exact
integer. It must return the same integer for equivalent keys.

The \var{equal-key?} procedure takes two keys and returns a true value
if they are equivalent and \code{\#f} otherwise.

The \var{valid-key?} procedure takes a datum and returns a true value
if it a valid key and \code{\#f} otherwise.

% ----------------------------------------------------------------------------
\defineentry{ht:ref}
\begin{procedure}
  \code{(ht:ref \var{ht} \var{key} \var{default})}
\end{procedure}
\returns{} the value associated with \var{key} in \var{ht},
\var{default} if none

% ----------------------------------------------------------------------------
\defineentry{ht:set}
\begin{procedure}
  \code{(ht:set \var{ht} \var{key} \var{val})}
\end{procedure}
\returns{} a hash table formed by associating \var{key} with \var{val}
in \var{ht}

% ----------------------------------------------------------------------------
\defineentry{ht:size}
\begin{procedure}
  \code{(ht:size \var{ht})}
\end{procedure}
\returns{} the number of entries in \var{ht}

% ----------------------------------------------------------------------------
\subsection{Error Strings}

% ----------------------------------------------------------------------------
\defineentry{current-exit-reason->english}
\begin{parameter}
  \code{current-exit-reason->english}
\end{parameter}
\hasvalue{} a procedure of one argument that returns an English string

The \code{current-exit-reason->english} parameter specifies the
conversion procedure used by \code{exit-reason->english}. It defaults
to \code{swish-exit-reason->english}.

% ----------------------------------------------------------------------------
\defineentry{exit-reason->english}
\begin{procedure}
  \code{(exit-reason->english \var{x})}
\end{procedure}
\returns{} a string in U.S. English

The \code{exit-reason->english} procedure converts an exit reason into
an English string using the procedure stored in parameter
\code{current-exit-reason->english}.

% ----------------------------------------------------------------------------
\defineentry{swish-exit-reason->english}
\begin{procedure}
  \code{(swish-exit-reason->english \var{x})}
\end{procedure}
\returns{} a string in U.S. English

The \code{swish-exit-reason->english} procedure converts an exit
reason from Swish into an English string.

% ----------------------------------------------------------------------------
\subsection{String Utilities}

The string utilities below are found in the \code{(swish
  string-utils)} library. For regular expression support, see the
\code{(swish pregexp)} library described in~\cite{pregexp}.

% ----------------------------------------------------------------------------
\defineentry{ends-with?}
\begin{procedure}
  \code{(ends-with? \var{s} \var{p})}
\end{procedure}
\returns{} a boolean

The \code{ends-with?} procedure determines whether or not the string
\var{s} ends with string \var{p} using case-sensitive comparisons.

% ----------------------------------------------------------------------------
\defineentry{ends-with-ci?}
\begin{procedure}
  \code{(ends-with-ci? \var{s} \var{p})}
\end{procedure}
\returns{} a boolean

The \code{ends-with-ci?} procedure determines whether or not the
string \var{s} ends with string \var{p} using case-insensitive
comparisons.

% ----------------------------------------------------------------------------
\defineentry{format-rfc2822}
\begin{procedure}
  \code{(format-rfc2822 \var{d})}
\end{procedure}
\returns{} a string like ``Thu, 28 Jul 2016 17:20:11 -0400''

The \code{format-rfc2822} procedure returns a string representation of
the date object \var{d} in the form specified in Section 3.3 of RFC
2822~\cite{RFC2822}.

% ----------------------------------------------------------------------------
\defineentry{join}
\begin{procedure}
  \code{(join \var{ls} \var{separator} \opt{\var{last-separator}})}
\end{procedure}
\returns{} a string

The \code{join} procedure returns the string formed by displaying each
of the elements of list \var{ls} separated by displaying
\var{separator}. When \var{last-separator} is specified, it is used as
the last separator.

% ----------------------------------------------------------------------------
\defineentry{split}
\begin{procedure}
  \code{(split \var{str} \var{separator})}
\end{procedure}
\returns{} a list of strings

The \code{split} procedure divides the \var{str} string by the
\var{separator} character into a list of strings, none of which
contain \var{separator}.

% ----------------------------------------------------------------------------
\defineentry{split-n}
\begin{procedure}
  \code{(split-n \var{str} \var{separator} \var{n})}
\end{procedure}
\returns{} a list of no more than \var{n} strings

The \code{split-n} procedure divides the \var{str} string by the
\var{separator} character into a list of at most \var{n} strings. The
last string may contain \var{separator}.

% ----------------------------------------------------------------------------
\defineentry{starts-with?}
\begin{procedure}
  \code{(starts-with? \var{s} \var{p})}
\end{procedure}
\returns{} a boolean

The \code{starts-with?} procedure determines whether or not the string
\var{s} starts with string \var{p} using case-sensitive comparisons.

% ----------------------------------------------------------------------------
\defineentry{starts-with-ci?}
\begin{procedure}
  \code{(starts-with-ci? \var{s} \var{p})}
\end{procedure}
\returns{} a boolean

The \code{starts-with-ci?} procedure determines whether or not the
string \var{s} starts with string \var{p} using case-insensitive
comparisons.

% ----------------------------------------------------------------------------
\defineentry{symbol-append}
\begin{procedure}
  \code{(symbol-append . \var{ls})}
\end{procedure}
\returns{} a symbol

The \code{symbol-append} procedure returns the symbol formed by
appending the symbols passed as arguments.
