% Copyright 2018 Beckman Coulter, Inc.
%
% Permission is hereby granted, free of charge, to any person
% obtaining a copy of this software and associated documentation files
% (the "Software"), to deal in the Software without restriction,
% including without limitation the rights to use, copy, modify, merge,
% publish, distribute, sublicense, and/or sell copies of the Software,
% and to permit persons to whom the Software is furnished to do so,
% subject to the following conditions:
%
% The above copyright notice and this permission notice shall be
% included in all copies or substantial portions of the Software.
%
% THE SOFTWARE IS PROVIDED "AS IS", WITHOUT WARRANTY OF ANY KIND,
% EXPRESS OR IMPLIED, INCLUDING BUT NOT LIMITED TO THE WARRANTIES OF
% MERCHANTABILITY, FITNESS FOR A PARTICULAR PURPOSE AND
% NONINFRINGEMENT. IN NO EVENT SHALL THE AUTHORS OR COPYRIGHT HOLDERS
% BE LIABLE FOR ANY CLAIM, DAMAGES OR OTHER LIABILITY, WHETHER IN AN
% ACTION OF CONTRACT, TORT OR OTHERWISE, ARISING FROM, OUT OF OR IN
% CONNECTION WITH THE SOFTWARE OR THE USE OR OTHER DEALINGS IN THE
% SOFTWARE.

\chapter {Command Line Interface}\label{chap:cli}

% For long options we want to explicitly see two hyphen characters
% rather than an en-dash. For normal text use sopt and lopt. Inside
% codebegin/end blocks Emacs may convert the characters but LaTeX will
% still render them properly.

\newcommand{\sopt}[1]{\code{{-}#1}}
\newcommand{\lopt}[1]{\code{{-}{-}#1}}
\newcommand{\str}[1]{\code{"#1"}}

\section {Introduction}

The command-line interface (\code{cli}) provides parsing of
command-line arguments as well as consistent usage of common options
and display of help.

\section {Theory of Operation}

Many programs parse command-line arguments and perform actions based
on them.  The \code{cli} library helps to make programs that process
arguments and display help simple and consistent. Command-line
arguments are parsed left to right in a single pass. Command-line
interface specifications, or \code{cli-specs}, are used for parsing
and error checking a command line, displaying one-line usage, and
displaying a full help summary.

Arguments may be preceded by a single dash (\sopt{}), a double
dash (\lopt{}), or no dash at all.
A single dash precedes short, single character arguments. The API does
not allow numbers as they could be mistaken as a negative numerical
value supplied to another argument. A double dash precedes longer,
more descriptive arguments, \lopt{repl} for example.
Positional arguments are not preceded by any dashes.  As arguments
with dashes are consumed, the remaining arguments are matched against
the positional specifications in order.

Argument specifications include a type such as: \code{bool},
\code{count}, \code{string}, and \code{list}.  A set of \code{bool}
and \code{count} arguments can be specified together (\sopt{abc} is
equivalent to \sopt{a} \sopt{b} \sopt{c}). Arguments of type
\code{list} collect values in left to right order.

The API does not directly support sub-commands and alternate usage
help text. These can be implemented using the primitives provided. The
implementations of \code{swish-build} and \code{swish-test} provide
examples of advanced command-line handling.

In the following REPL transcript, we define \code{example-cli} using
\code{cli-specs}. We then set the \code{command-line-arguments}
parameter as they would be for an application. Calling
\code{parse-command-line-arguments} returns a procedure, \code{opt},
which we can use to access the parsed command-line values. Finally, we use
\code{display-help} to display the automatically generated help.

\codebegin
> (define example-cli
    (cli-specs
     default-help
     ["verbose" -v count "indicates verbosity level"]
     ["output" -o (string "<output>") "print output to an <output> file"]
     ["repl" --repl bool "start a repl"]
     ["files" (list "<file>" ...) "a list of input files"]))
> (command-line-arguments '("-vvv" "-o" "file.out" "file.in"))
> (define opt (parse-command-line-arguments example-cli))
> (opt "verbose")
3
> (opt "output")
"file.out"
> (opt "files")
("file.in")
> (display-help "sample" example-cli)
Usage: sample [-hv] [-o <output>] [--repl] <file> ...

  -h, --help        display this help and exit
  -v                indicates verbosity level
  -o <output>       print output to an <output> file
  --repl            start a repl
  <file> ...        a list of input files
\codeend

Putting the parts together into \code{sample.ss}, we have a working
example albeit incomplete.

\codebegin
#!/usr/bin/env swish

(define example-cli
  (cli-specs
   default-help
   ["verbose" -v count "indicates verbosity level"]
   ["output" -o (string "<output>") "print output to an <output> file"]
   ["repl" --repl bool "start a repl"]
   ["files" (list "<file>" ...) "a list of input files"]))

(let ([opt (parse-command-line-arguments example-cli)])
  (when (opt "help")
    (display-help "sample" example-cli)
    (exit 0))
  (let ([verbosity (or (opt "verbose") 0)])
    (when (> verbosity 0)
      (printf "showing verbosity level: ~a~n" verbosity)))
  (when (opt "repl")
    (new-cafe)))
\codeend

\section {Programming Interface}

\defineentry{cli-specs}
\begin{syntax}
  \begin{alltt}
(cli-specs
  \opt{default-help}
  (\var{name} \opt{\var{short}} \opt{\var{long}} \var{type} \var{help}
    \opt{(conflicts \var{conflicts})}
    \opt{(requires \var{requires})}
    \opt{(usage \opt{\var{visibility}} \opt{\var{how}})})
  \etc{})\strut\end{alltt}
\end{syntax}
\expandsto{}\begin{alltt}\antipar
a list of \code{<arg-spec>} tuples
\end{alltt}

The \code{cli-specs} macro simplifies the creation of the
\code{<arg-spec>} tuples. The \code{<arg-spec>} \code{name} field
uniquely identifies a specification, and is used to retrieve parsed
argument values and check constraints.

\begin{argtbl}
  \argrow{name}{a string to identify the argument}
  \argrow{short}{a symbol of the form \sopt{x}, where \var{x} is a
    single character, see below}
  \argrow{long}{a symbol of the form \lopt{x}, where \var{x} is a string}
  \argrow{type}{see Figure~\ref{fig:cli-types}}
  \argrow{help}{a string or list of strings that describes the
    argument}
  \argrow{conflicts}{a list of \code{<arg-spec>} names}
  \argrow{requires}{a list of \code{<arg-spec>} names}
\end{argtbl}

To specify \sopt{i} or \sopt{I} for \var{short}, use |\sopt{i}| and
|\sopt{I}| respectively to prevent Chez Scheme from reading them as
the complex number $0-1i$.

\begin{figure}[H]
\begin{tabular}{lp{3.6in}}
  Type & Result \\ \hline
  \code{bool} & \code{\#t} \\
  \code{count} & a positive integer \\
  \code{(string \var{x})} & a string \\
  \code{(list \var{x})} & a list of one item \\
  \code{(list \var{x} ...)} & a list of one or more items up to the
  next argument \\
  \code{(list . \var{x})} & a list of the rest of the arguments \\
  \hline
\end{tabular}\par\medskip
For each type where \var{x} is specified, \var{x} is a string that is
used in the help display.
\caption{Command-line argument types\label{fig:cli-types}}
\end{figure}

The \code{list} types can support multiple \var{x} arguments, for
instance \code{(list "i1" "i2")} would specify a list of two
arguments.

\begin{grammar}
  \prodrow{\var{visibility}}{\code{show}}
  \altrow{\code{hide}}
  \altrow{\code{fit}}
\end{grammar}

When printing the help usage line, a \var{visiblility} of \code{show}
means the argument must be displayed. \code{hide} forces the argument
to be hidden. \code{fit} displays the argument if it fits on the
line.

\begin{grammar}
  \prodrow{\var{how}}{\code{short}}
  \altrow{\code{long}}
  \altrow{\code{opt}}
  \altrow{\code{req}}
\end{grammar}

The \var{how} expands into input of the \code{format-spec} procedure
according to Figure~\ref{fig:cli-usage-keywords}.

\begin{figure}[H]
  \begin{tabular}{r l}
    Keyword & Expands into: \\
    \code{short} & \code{(opt (and short args))} \\
    \code{long} & \code{(opt (and long args))} \\
    \code{opt} & \code{(opt (and (or short long) args))} \\
    \code{req} & \code{(req (and (or short long) args))}
  \end{tabular}
  \caption{\code{cli-specs} \var{how} field\label{fig:cli-usage-keywords}}
\end{figure}

For options with \var{short} or \var{long} specified, \code{fit} and
\code{opt} are the defaults. For other options, \code{show} and
\code{req} are the defaults.

\var{conflicts} is a list of specification names that prevent this
argument from processing correctly. When multiple command-line
arguments are specified that are in conflict, an exception is raised.

\var{requires} is a list of other specification names that are
necessary for this argument to be processed correctly. Unless all the
required command-line arguments are specified, an exception is raised.

The \code{conflicts}, \code{requires}, and \code{usage} clauses may be
specified in any order.

\defineentry{display-help}
\begin{procedure}
  \code{(display-help \var{exe-name} \var{specs} \opt{\var{args}} \opt{\var{op}})}
\end{procedure}
\returns{} unspecified

The \code{display-help} procedure is equivalent to calling
\code{display-usage} with a prefix of \str{Usage:} followed by
\code{display-options}.

\defineentry{display-options}
\begin{procedure}
  \code{(display-options \var{specs} \opt{\var{args}} \opt{\var{op}})}
\end{procedure}
\returns{} unspecified

For each specification in \var{specs}, the \code{display-options}
procedure renders two columns of output to \var{op}, which defaults to
the current output port. The first column renders the short and long
form of the argument with its additional inputs. The second column
renders the \code{<arg-spec>} \code{help} field and will automatically
wrap if the \code{help-wrap-width} is exceeded.

If an \var{args} hash table is specified, the specified value appended
to the second column. This is useful for displaying default or current
values.

\defineentry{display-usage}
\begin{procedure}
  \code{(display-usage \var{prefix} \var{exe-name} \var{specs} \opt{\var{width}} \opt{\var{op}})}
\end{procedure}
\returns{} unspecified

The \code{display-usage} procedure displays the first line of help
output to \var{op}, which defaults to the current output port. It
starts with \var{prefix} and \var{exe-name} then attempts to fit
\var{specs} onto the line using \code{format-spec}. When the line will
exceed \var{width} characters, some arguments may collapse to
\code{[options]}.

A \var{width} of \code{\#f} defaults the line width to
\code{help-wrap-width}.

\defineentry{format-spec}
\begin{procedure}
  \code{(format-spec \var{spec} \opt{\var{how}})}
\end{procedure}
\returns{} a string

The \code{format-spec} procedure is responsible rendering \var{spec}
as a string as specified by \var{how}.  \code{format-spec} can display
dashes in front of arguments, ellipses on list types, and brackets
around optional arguments. A \var{how} of \code{\#f} defaults to the
\code{<arg-spec>} \code{usage} field.

\begin{tabular}{lp{3.6in}}
  \var{how} & Return value: \\
  \hline
  \code{short} & \str{\sopt{x}} if \var{spec} \var{short} is \code{x},
  else \code{\#f}\\

  \code{long} & \str{\lopt{x}} if \var{spec} \var{long} is \code{x},
  else \code{\#f}\\

  \code{args} & The \var{spec} \var{type} is evaluated as follows: \\
  &
  \begin{minipage}{.75\textwidth}
  \begin{tabular}{l l}
   % \code{<arg-spec>} \code{type}
    \var{type} & Return value:\\
    \hline
    \code{bool} & \code{\#f}\\
    \code{count} & \code{\#f}\\
    \code{(string \var{x})} & \str{x}\\
    \code{(list \var{x})} & \str{x}\\
    \code{(list \var{x} ...)} & \str{x ...}\\
    \code{(list . \var{x})} & \str{x ...}\\
    \hline
  \end{tabular}
  \end{minipage}\\

  \code{(or \var{how} \etc{})} & Recur and use the first
  non-\code{\#f}\\

  \code{(and \var{how} \etc{})} & Recur and concatenate all
  non-\code{\#f} values\\

  \code{(opt \var{how})} & Recur and surround the result with square
  brackets \code{[]} if non-\code{\#f}\\

  \code{(req \var{how})} & Recur and use the result\\
  \hline
\end{tabular}

\defineentry{help-wrap-width}
\begin{parameter}
  \code{help-wrap-width}
\end{parameter}
\hasvalue{} a positive fixnum

The \code{help-wrap-width} parameter specifies the default width for
\code{display-usage} and \code{display-options}.

\defineentry{parse-command-line-arguments}
\begin{procedure}
  \code{(parse-command-line-arguments \var{specs} \opt{\var{ls}} \opt{\var{fail}})}
\end{procedure}
\returns{} a procedure

The \code{parse-command-line-arguments} procedure processes the elements
of \var{ls} from left to right in a single pass.
As it scans each \var{x} in \var{ls}, the parser must find a suitable \var{s}
within \var{specs}.
If a suitable \var{s} cannot be found, the parser reports an error by
calling \var{fail}.
Based on the type of \var{s}, the parser may consume additional elements
following \var{x}.
The type of \var{s} determines what data the parser records for that
argument.
% When \var{s} is satisfied, the parser resumes its scan of \var{ls}.
When \var{s} is satisfied, the parser continues scanning the
remaining elements of \var{ls}.

% say more about how it finds ``suitable'' spec
%  - looks for - or --
%  - explodes -abc to -a -b -c on demand

The parser returns a procedure \var{p} that accepts zero or one
argument. When called with no arguments, \var{p} returns a hash table
that maps the name of each \var{s} found while processing \var{ls}
to the data recorded for that argument.
When called with the name of an element \var{s} in \var{specs},
\var{p} returns the data, if any, recorded for that name in the hash table
or else \code{\#f}.
If a particular \var{s} was not found while processing \var{ls},
the internal hash table has no entry for the name of \var{s}
and \var{p} returns \code{\#f} when given that name.
If called with a name that is not in \var{specs}, \var{p}
raises an exception.

The following table summarizes the parser's behavior.

\begin{tabular}{lll}
  \code{<arg-spec> type} & extra arguments consumed / recorded & return value of \code{(\var{p} \var{name})}\\ \hline
  \code{bool}
  & none
  & \code{\#t} \\
  \code{count}
  & none
  & an exact positive integer \\
  \code{(string \var{x})}
  & one
  & a string \\
  \code{(list \var{$x_0$} \etc{} \var{$x_n$} ...)}
  & $n$ or more, up to the next option
  & a list of strings \\
  \code{(list \var{$x_0$} \etc{} \var{$x_n$} . \var{rest})}
  & at least $n$ and all remaining
  & a list of strings \\
\end{tabular}

By default \var{ls} is the value of \code{(command-line-arguments)}
and \var{fail} is a procedure that applies \code{errorf} to its
arguments. Providing a \var{fail} procedure allows a developer to
accumulate parsing errors without necessarily generating exceptions.

\begin{tupledef}{<arg-spec>}
  \argrow{name}{a string to use as the key of the output hash table}
  \argrow{type}{see Figure~\ref{fig:cli-types}}
  \argrow{short}{\code{\#f} $|$ a character}
  \argrow{long}{\code{\#f} $|$ a string }
  \argrow{help}{a string or list of strings describing argument}
  \argrow{conflicts}{a list of \code{<arg-spec>} names}
  \argrow{requires}{a list of \code{<arg-spec>} names}
  \argrow{usage}{a list containing one \var{visibility} symbol and a
    \code{format-spec} \var{how} expression}
\end{tupledef}
