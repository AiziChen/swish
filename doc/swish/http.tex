% Copyright 2018 Beckman Coulter, Inc.
%
% Permission is hereby granted, free of charge, to any person
% obtaining a copy of this software and associated documentation files
% (the "Software"), to deal in the Software without restriction,
% including without limitation the rights to use, copy, modify, merge,
% publish, distribute, sublicense, and/or sell copies of the Software,
% and to permit persons to whom the Software is furnished to do so,
% subject to the following conditions:
%
% The above copyright notice and this permission notice shall be
% included in all copies or substantial portions of the Software.
%
% THE SOFTWARE IS PROVIDED "AS IS", WITHOUT WARRANTY OF ANY KIND,
% EXPRESS OR IMPLIED, INCLUDING BUT NOT LIMITED TO THE WARRANTIES OF
% MERCHANTABILITY, FITNESS FOR A PARTICULAR PURPOSE AND
% NONINFRINGEMENT. IN NO EVENT SHALL THE AUTHORS OR COPYRIGHT HOLDERS
% BE LIABLE FOR ANY CLAIM, DAMAGES OR OTHER LIABILITY, WHETHER IN AN
% ACTION OF CONTRACT, TORT OR OTHERWISE, ARISING FROM, OUT OF OR IN
% CONNECTION WITH THE SOFTWARE OR THE USE OR OTHER DEALINGS IN THE
% SOFTWARE.

\chapter {HTTP Interface}\label{chap:http}

\section {Introduction}

The HTTP interface provides a basic implementation of the Hypertext
Transfer Protocol~\cite{RFC7230}. The programming interface includes
procedures for the HyperText Markup Language (HTML) version
5~\cite{html5} and JavaScript Object Notation (JSON)~\cite{RFC7159}.

\section {Theory of Operation}

The HTTP interface provides a supervisor, \code{http-sup}, to manage
the \code{http-listener} gen-server, the \code{http-cache}
gen-server, and new connection processes. This structure is
illustrated in Figure~\ref{fig:http-tree}.

\begin{figure}
  \center\includegraphics{swish/http-tree.pdf}
  \caption{\label{fig:http-tree}HTTP tree}
\end{figure}

The \code{http-listener} is a gen-server that creates a TCP listener
using \code{listen-tcp} and accepts new connections using
\code{accept-tcp}. For each connection, the http-listener uses its
supervisor to spawn and link a handler process.

Each handler reads from its input port until a CR LF
occurs. Well-formed input is converted to a \code{<request>} tuple,
and the HTTP request header and any content parameters are read.

When \code{Content-Length} appears in the header, the content bytes
are read. If the \code{Content-Type} is \code{multipart/form-data}
or \code{application/x-www-form-urlencoded}, the content is
converted to an association list. Otherwise, parameter
\code{"unhandled-content"} is included with the value of the raw
bytevector of data. Each uploaded file is stored in
\code{(tmp-path)}, and the association list value is
\code{\#(<file> \var{filename})}.

\code{http:file-handler} is then called combining the
\code{<request>} query parameters and content
parameters. \code{http:file-handler} logs the specific request,
validates that the requested path does not include ``..'', retrieves a
page handler from the \code{http-cache}, and invokes it. A
\emph{page handler} is a procedure which responds to a particular HTTP
request. The output port is flushed after the page handler returns.

After a request is processed, all uploaded files are deleted, and the
current process and connection can be reused. The system reads another
request from the input port.

The \code{http-cache} is a gen-server that stores page handlers and
provides a mapping from file extension to content type. It creates a
directory watcher using \code{watch-directory} to invalidate the
cache when anything in the \code{(web-dir)} tree changes.

The \code{http-cache} considers a path that ends in ``.ss'' a
dynamic page loaded from \code{(web-dir)}. Other paths are
considered static and are sent directly over the connection using
\code{http:respond-file}.

\section {Security}

The HTTP interface is written in Scheme, and therefore buffer overrun
exploits cannot be used against the system.

User input should be carefully checked before calling \code{eval} or
invoking a database query.

A URL which directs the system away from \code{(web-dir)} using
``..'' could allow access to system files. \code{http:file-handler}
explicitly checks for relative paths.

The HTTP interface limits incoming data to protect against large
memory allocation which may crash the system. URL requests are limited
to 4,096 bytes. Headers are limited to 1,048,576 bytes. Posted content
is limited to 4,194,304 bytes, not including uploaded files.

The HTTP interface does not limit incoming file uploads. If the disk
runs out of space, the handler process will exit with an I/O error.

\section {Dynamic Pages}

A dynamic page is a sequence of definitions followed by a sequence of
expressions stored in ``.ss'' files in \code{(web-dir)}.  The
definitions and expressions are placed in a \code{lambda} expression
that is evaluated by the \code{interpret} system procedure.  The
page is responsible for sending the HTTP response. The output port is
flushed after the page handler returns.

\section {Dynamic Page Constructs}

The evaluated \code{lambda} expression exposes the following
variables to a dynamic page:

\begin{argtbl}
  \argrow{ip}{binary input port}
  \argrow{op}{binary output port}
  \argrow{request}{a \code{<request>} tuple}
  \argrow{header}{an association list}
  \argrow{params}{an association list}
\end{argtbl}

\defineentry{find-param}
\begin{syntax}
  \code{(find-param \var{key})}
\end{syntax}\antipar
\implementation{The \code{find-param} macro expands to
  \code{(http:find-param \var{key} \var{params})}.}

\defineentry{get-param}
\begin{syntax}
  \code{(get-param \var{key})}
\end{syntax}\antipar
\implementation{The \code{get-param} macro expands to
  \code{(http:get-param \var{key} \var{params})}.}

\defineentry{http:include}
\begin{syntax}
  \code{(http:include "\var{filename}")}
\end{syntax}\antipar

The \code{http:include} construct includes the definitions from
\var{filename}, a path relative to \code{(web-dir)} if
\var{filename} begins with a forward slash, else relative to the
directory of the current file.

\implementation{The \code{http:include} macro calls
  \code{read-file} and \code{read-bytevector} to retrieve a list
  of expressions that are spliced in at the same scope as the use of
  \code{http:include}. The splicing is done with \code{let-syntax}
  so that any nested \code{http:include} expressions are processed
  relative to the directory of \var{filename}.}

\section {Programming Interface}

\begin{tupledef}{<request>}
  \argrow{method}{a symbol}
  \argrow{path}{a decoded string}
  \argrow{query}{a decoded association list}
\end{tupledef}

\defineentry{http-port-number}
\begin{parameter}
  \code{http-port-number}
\end{parameter}
\hasvalue{} \code{\#f} or a fixnum $0 \le port \le 65535$

The \code{http-port-number} parameter specifies whether or not
\code{http-sup:start\&link} should start the HTTP server and,
if started, on what port that server should listen for connections.

\defineentry{http-sup:start\&link}
\begin{procedure}
  \code{(http-sup:start\&link)}
\end{procedure}
\returns{}
\code{\#(ok \var{pid})} \alt{} \code{\#(error \var{error})}

If \code{(http-port-number)} is not \code{\#f},
the \code{http-sup:start\&link} procedure creates a supervisor named
\code{http-sup} using \code{supervisor:start\&link} configured
one-for-one with up to 10 restarts every 10 seconds. The supervisor
starts the \code{http-cache} and \code{http-listener} gen-servers.

\defineentry{http:get-port-number}
\begin{procedure}
  \code{(http:get-port-number)}
\end{procedure}
\returns{} see below

If the \code{(http-port-number)} is configured to be zero, the
operating system will choose an available port
number. \code{(http:get-port-number)} uses
\code{listener-port-number} to retrieve the actual port number that
the server is listening on.

\defineentry{http:find-header}
\begin{procedure}
  \code{(http:find-header \var{name} \var{header})}
\end{procedure}
\returns{} a string \alt{} \code{\#f}

The \code{http:find-header} procedure returns the value associated
with \var{name} in \var{header}. Header comparisons are
case-insensitive. If \var{name} is not a string, exception
\code{\#(bad-arg http:find-header \var{name})} is raised.

\defineentry{http:get-header}
\begin{procedure}
  \code{(http:get-header \var{name} \var{header})}
\end{procedure}
\returns{} a string

The \code{http:get-header} procedure returns the value assocated
with \var{name} in \var{header} or exits with reason
\code{\#(invalid-header \var{name} \var{header})}. Header
comparisons are case-insensitive. If \var{name} is not a string,
exception \code{\#(bad-arg http:get-header \var{name})} is raised.

\defineentry{http:find-param}
\begin{procedure}
  \code{(http:find-param \var{name} \var{params})}
\end{procedure}
\returns{} a string \alt{} \code{\#f}

The \code{http:find-param} procedure returns the value associated
with \var{name} in \var{params}. Parameter comparisons are
case-sensitive. If \var{name} is not a string, exception
\code{\#(bad-arg http:find-param \var{name})} is raised.

\defineentry{http:get-param}
\begin{procedure}
  \code{(http:get-param \var{name} \var{params})}
\end{procedure}
\returns{} a string

The \code{http:get-param} procedure returns the value associated
with \var{name} in \var{params} or exits with reason
\code{\#(invalid-param \var{name} \var{params})}. Parameter
comparisons are case-sensitive. If \var{name} is not a string,
exception \code{\#(bad-arg http:get-param \var{name})} is raised.

\defineentry{http:read-header}
\begin{procedure}
  \code{(http:read-header \var{ip} \var{limit})}
\end{procedure}
\returns{} an association list

The \code{http:read-header} procedure reads from the binary input
port \var{ip} until a blank line is read.

An association list is created by making a string from the characters
before the first colon as the key. Non-linear white space is skipped,
and the remaining characters are converted to a string value.

Reading beyond \var{limit} will result in exiting with reason
\code{input-limit-exceeded}.

Failure to find a colon on any given line will result in exiting with
reason \code{invalid-header}.

\defineentry{http:read-status}
\begin{procedure}
  \code{(http:read-status \var{ip} \var{limit})}
\end{procedure}
\returns{} number \alt{} \code{\#f}

The \code{http:read-status} procedure reads the HTTP response status
line from the binary input port \var{op} and returns the number if
well formed and \code{\#f} otherwise. Reading beyond \var{limit}
will result in exiting with reason \code{input-limit-exceeded}.

\defineentry{http:write-status}
\begin{procedure}
  \code{(http:write-status \var{op} \var{status})}
\end{procedure}
\returns{} unspecified

The \code{http:write-status} procedure writes the HTTP response
status line to the binary output port \var{op}.

Unless \var{status} is a fixnum and $100 \leq \var{status} \leq 599$, the
exception \code{\#(bad-arg http:write-status \var{status})} is
raised.

According to HTTP~\cite{RFC7230} the status line includes a human
readable reason phrase. The grammar shows that it can in fact be 0
characters long; therefore, the reason phrase is not included in this
implementation.

\defineentry{http:write-header}
\begin{procedure}
  \code{(http:write-header \var{op} \var{header})}
\end{procedure}
\returns{} unspecified

The \code{http:write-header} procedure writes the HTTP \var{header},
and trailing CR LF to the binary output port \var{op}.

\var{header} is an association list. If \var{header}'s keys are not
strings, exception \code{\#(bad-arg http:write-header \var{header})}
is raised.

\defineentry{http:respond}
\begin{procedure}
  \code{(http:respond \var{op} \var{status} \var{header} \var{content})}
\end{procedure}
\returns{} unspecified

The \code{http:respond} procedure writes the HTTP \var{status} and
\var{header} to binary output port \var{op} using
\code{http:write-status} and \code{http:write-header}, adding
\code{Content-Length} to the \var{header}. When
\code{Cache-Control} is not present in \var{header}, it is added
with value \code{no-cache}. The \var{content} is then written, and
the output port is flushed.

\var{content} is a bytevector.

\defineentry{http:respond-file}
\begin{procedure}
  \code{(http:respond-file \var{op} \var{status} \var{header} \var{filename})}
\end{procedure}
\returns{} unspecified

The \code{http:respond-file} procedure writes the HTTP \var{status}
and \var{header} to binary output port \var{op} using
\code{http:write-status} and \code{http:write-header}, adding
\code{Content-Length} to \var{header}.  The \code{Cache-Control}
header is added, if it is not already present, with value
\code{max-age=3600}. The \code{Content-Type} header is added if
it is not already present and the extension of \var{filename} matches
(case insensitively) an extension in the \code{mime-types} file of
\code{(web-dir)}. Each line of \code{mime-types} has the form
\code{("\var{extension}"~.~"\var{Content-Type}")}. The content of
the file is streamed to the output port so that the file does not need
to be loaded into memory. The output port is flushed.

\defineentry{http:percent-encode}
\begin{procedure}
  \code{(http:percent-encode \var{s})}
\end{procedure}
\returns{} an encoded string

The \code{http:percent-encode} procedure writes the characters
\code{A}--\code{Z}, \code{a}--\code{z},
\code{0}--\code{9}, hyphen, underscore, period, and
\code{\~}. Other characters are converted to a \code{\%} prefix
and two digit hexadecimal representation.

\defineentry{html:encode}
\begin{procedure}
  \code{(html:encode \var{s})} \\
  \code{(html:encode \var{op} \var{s})}\strut
\end{procedure}
\returns{} see below

The \code{html:encode} procedure converts special character entities
in string \var{s}.

\begin{tabular}{ll}
  input & output \\ \hline
  \code{"} & \code{\&quot;} \\
  \code{\&} & \code{\&amp;} \\
  \code{\textless} & \code{\&lt;} \\
  \code{\textgreater} & \code{\&gt;} \\
  \hline
\end{tabular}

The single argument form of \code{html:encode} returns an encoded
string.

The two argument form of \code{html:encode} sends the encoded string
to the textual output port \var{op}.

\defineentry{html->string}
\begin{procedure}
  \code{(html->string \var{x})} \\
  \code{(html->string \var{op} \var{x})}\strut
\end{procedure}
\returns{} see below

The \code{html->string} procedure transforms an object into
HTML. The transformation, $H$, is described below:

\begin{tabular}{ll}
  \var{x} & $H(\var{x})$\\ \hline

  \code{()} & nothing\\
  \code{\#!void} & nothing\\
  \code{\var{string}} & $E(\var{string})$\\
  \code{\var{number}} & \var{number}\\
  \code{(begin \var{pattern} \ldots)} & $H(\var{pattern})$\ldots\\
  \code{(cdata \var{string} \ldots)} &
  \code{[!CDATA[\var{string}$\ldots$]]}\\
  \code{(html5 \opt{(@ \var{attr} \ldots)} \var{pattern} \ldots)} &
  \code{<!DOCTYPE html><html $A(\var{attr})$ $\ldots$>$H(\var{pattern})\ldots$</html>}\\
  \code{(raw \var{string} \ldots)} & \var{string}$\ldots$\\
  \code{(script \opt{(@ \var{attr} \ldots)} \var{string} \ldots)} &
  \code{<script $A(\var{attr})$ $\ldots$>\var{string}$\ldots$</script>}\\
  \code{(style \opt{(@ \var{attr} \ldots)} \var{string} \ldots)} &
  \code{<style $A(\var{attr})$ $\ldots$>\var{string}$\ldots$</style>}\\
  \code{(\var{tag} \opt{(@ \var{attr} \ldots)} \var{pattern} \ldots)} &
  \code{<\var{tag} $A(\var{attr})$ $\ldots$>$H(\var{pattern})\ldots$</\var{tag}>}\\
  \code{(\var{void-tag} \opt{(@ \var{attr} \ldots)})} &
  \code{<\var{void-tag} $A(\var{attr})$ $\ldots$>}\\

  \hline
\end{tabular}

$E$ denotes the \code{html:encode} function.

A \var{void-tag} is one of \code{area}, \code{base}, \code{br},
\code{col}, \code{embed}, \code{hr}, \code{img},
\code{input}, \code{keygen}, \code{link}, \code{menuitem},
\code{meta}, \code{param}, \code{source}, \code{track}, or
\code{wbr}. A \var{tag} is any other symbol.

The attribute transformation, $A$, is described below, where \var{key}
is a symbol:

\begin{tabular}{ll}
  \var{attr} & $A(\var{attr})$\\ \hline

  \code{\#!void} & nothing\\
  \code{(\var{key})} & \var{key}\\
  \code{(\var{key} \var{string})} & \code{\var{key}="$E(\var{string})$"}\\
  \code{(\var{key} \var{number})} & \code{\var{key}="\var{number}"}\\

  \hline
\end{tabular}

The single argument form of \code{html->string} returns an encoded
HTML string.

The two argument form of \code{html->string} sends the encoded HTML
string to the textual output port \var{op}.

Input that does not match the specification causes a
\code{\#(bad-arg html->string \var{x})} exception to be raised.

\defineentry{html->bytevector}
\begin{procedure}
  \code{(html->bytevector \var{x})}
\end{procedure}
\returns{} a bytevector

The \code{html->bytevector} procedure calls \code{html->string} on
\var{x} using a bytevector output port transcoded using
\code{(make-utf8-transcoder)} and returns the resulting bytevector.

\subsection {JavaScript Object Notation}

This implementation translates JavaScript types into the following
Scheme types:

\begin{tabular}{ll}
  JavaScript & Scheme \\ \hline

  \code{true} & \code{\#t} \\
  \code{false} & \code{\#f} \\
  \code{null} & \code{\#\textbackslash nul} \\
  \var{string} & \var{string} \\
  \var{number} & \var{number} \\
  \var{array} & \var{list} \\
  \var{object} & hashtable mapping case-sensitive strings to values \\

  \hline
\end{tabular}

This implementation does not range check values to ensure that a
JavaScript implementation can interpret the data.

Some JavaScript implementations allow inline expressions within JSON
objects. \code{json:write} enables this by supporting symbols as
values. Symbols are inlined into the output.

\defineentry{json:extend-object}
\begin{syntax}
  \code{(json:extend-object \var{ht} [\var{key} \var{value}] \etc)}
\end{syntax}

The \code{json:extend-object} construct adds the \var{key} /
\var{value} pairs to the hashtable \var{ht} using
\code{hashtable-set!}. The resulting expression returns \var{ht}.

\defineentry{json:make-object}
\begin{syntax}
  \code{(json:make-object [\var{key} \var{value}] \etc)}
\end{syntax}

The \code{json:make-object} construct expands into a call to
\code{json:extend-object} with a new hashtable.

\defineentry{json:read}
\begin{procedure}
  \code{(json:read \var{ip})}
\end{procedure}
\returns{} a Scheme object

The \code{json:read} procedure reads characters from the textual
input port \var{ip} and returns an appropriate Scheme object.

The following exceptions may be raised:
\begin{itemize}
\item \code{invalid-surrogate-pair}
\item \code{unexpected-eof}
\item \code{\#(unexpected-input \var{data} \var{input-position})}
\end{itemize}

\defineentry{json:write}
\begin{procedure}
  \code{(json:write \var{op} \var{x})}
\end{procedure}
\returns{} unspecified

The \code{json:write} procedure writes the object \var{x} to the
textual output port \var{op} in JSON format. JSON objects are sorted
by key using \code{string<?} to provide stable output. Scheme fixnums,
bignums, and finite flonums may be used as numbers.

If an object cannot be formatted, \code{\#(invalid-datum \var{x})}
is raised.

\defineentry{json:object->bytevector}
\begin{procedure}
  \code{(json:object->bytevector \var{x})}
\end{procedure}
\returns{} a bytevector

The \code{json:object->bytevector} procedure calls
\code{json:write} on \var{x} using a bytevector output port
transcoded using \code{(make-utf8-transcoder)} and returns the
resulting bytevector.

\defineentry{json:object->string}
\begin{procedure}
  \code{(json:object->string \var{x})}
\end{procedure}
\returns{} a JSON formatted string

The \code{json:object->string} procedure creates a string output
port, calls \code{json:write} on \var{x}, and returns the resulting
string.

\defineentry{json:string->object}
\begin{procedure}
  \code{(json:string->object \var{x})}
\end{procedure}
\returns{} a Scheme object

The \code{json:string->object} procedure creates a string input port
on \var{x}, calls \code{json:read}, and returns the resulting Scheme
object after making sure the rest of the string is only whitespace.

\section {Published Events}

\begin{event}\codeindex{<http-request>}\end{event}\antipar
\begin{argtbl}
  \argrow{timestamp}{timestamp from \code{erlang:now}}
  \argrow{pid}{handler process}
  \argrow{host}{the IP address of the client}
  \argrow{method}{\code{<request>} method}
  \argrow{path}{\code{<request>} path}
  \argrow{header}{an association list}
  \argrow{params}{an association list}
\end{argtbl}
