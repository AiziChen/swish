% Copyright 2018 Beckman Coulter, Inc.
%
% Permission is hereby granted, free of charge, to any person
% obtaining a copy of this software and associated documentation files
% (the "Software"), to deal in the Software without restriction,
% including without limitation the rights to use, copy, modify, merge,
% publish, distribute, sublicense, and/or sell copies of the Software,
% and to permit persons to whom the Software is furnished to do so,
% subject to the following conditions:
%
% The above copyright notice and this permission notice shall be
% included in all copies or substantial portions of the Software.
%
% THE SOFTWARE IS PROVIDED "AS IS", WITHOUT WARRANTY OF ANY KIND,
% EXPRESS OR IMPLIED, INCLUDING BUT NOT LIMITED TO THE WARRANTIES OF
% MERCHANTABILITY, FITNESS FOR A PARTICULAR PURPOSE AND
% NONINFRINGEMENT. IN NO EVENT SHALL THE AUTHORS OR COPYRIGHT HOLDERS
% BE LIABLE FOR ANY CLAIM, DAMAGES OR OTHER LIABILITY, WHETHER IN AN
% ACTION OF CONTRACT, TORT OR OTHERWISE, ARISING FROM, OUT OF OR IN
% CONNECTION WITH THE SOFTWARE OR THE USE OR OTHER DEALINGS IN THE
% SOFTWARE.

\chapter {Database Interface}\label{chap:db}

\section {Introduction}

The database (\code{db}) interface is a gen-server which provides a
basic transaction framework to retrieve and store data in a SQLite
database. It provides functions to use transactions (directly and
lazily).

The low-level SQLite interface can be found in the operating system
interface design (see Chapter~\ref{chap:osi}).

Other SQLite resources are available online~\cite{sqlite} or in The
Definitive Guide to SQLite~\cite{sqlite-guide}.

\section {Theory of Operation}

The \code{db} gen-server serializes internal requests to the
database.  For storage and retrieval of data, each transaction is
processed in turn by a separate linked process.  The gen-server does
not block waiting for this process to finish so that it can maintain
linear performance by keeping its inbox short. The return value of the
transaction is returned to the caller or an error is generated without
tearing down the gen-server.

To facilitate logging, the \code{db} gen-server can lazily open a
transaction. In order to allow other processes access to the database,
lazy transactions should be closed occasionally. To support this, it
tracks a count of entries in the current transaction. A transaction is
committed when the threshold of 10,000 is reached, the message queue
of the \code{db} is empty, or when a direct transaction is
requested.  Each database is created with write-ahead logging enabled
to prevent write operations from blocking on queries made from another
connection.

SQLite has three types of transactions: deferred, immediate, and
exclusive. This interface uses only immediate transactions to simplify
the handling of the \code{SQLITE\_BUSY} error.  Using immediate
transactions means that \code{SQLITE\_BUSY} will only occur during
\code{BEGIN IMMEDIATE}, \code{BEGIN TRANSACTION}, \code{COMMIT},
and \code{ROLLBACK}\footnote{Our testing showed that
  \code{ROLLBACK} returns \code{SQLITE\_BUSY} only when a
  \code{COMMIT} for the same transaction returned
  \code{SQLITE\_BUSY}. This framework never causes that situation to
  occur, but it guards against it anyway.} statements. For each of
these statements, when a \code{SQLITE\_BUSY} occurs, the code waits
for a brief time, then retries the statement. The wait times in
milliseconds follow the pattern \code{(2 3 6 11 16 21 26 26 26 51
  51 . \#0=(101 . \#0\#))}, and up to 500 retries are attempted before
exiting with \code{\#(db-retry-failed \var{sql} \var{count})}.  When
the retry count is positive, it is logged to the event manager along
with the total duration with a \code{<transaction-retry>} event.

\begin{pubevent}{<transaction-retry>}
  \argrow{timestamp}{timestamp from \code{erlang:now}}
  \argrow{database}{database filename}
  \argrow{duration}{duration in milliseconds}
  \argrow{count}{retry count}
  \argrow{sql}{query}
\end{pubevent}

The \code{db} gen-server uses the operating system interface to
interact with SQLite. To prevent memory leaks, each raw database and
statement handle is wrapped in a Scheme record and registered with a
guardian via \code{make-foreign-handle-guardian}.

\paragraph* {state}\index{db!state}
\code{(define-state-tuple <db-state> filename db cache queue worker)}
\begin{itemize}
\item \code{filename} is the database specified when the server was
  started.
\item \code{db} is the database record.
\item \code{cache} is a hash table mapping SQL strings to SQLite
  prepared statements.
\item \code{queue} is a queue of log and transaction requests.
\item \code{worker} is the pid of the active worker or \code{\#f}.
\end{itemize}

\paragraph* {dictionary parameters}\index{db!parameters}
\begin{itemize}

\item \code{current-database} stores a Scheme record:\newline
  \code{(define-record-type database (fields (mutable
    handle)))}\newline The \code{handle} is set to \code{\#f} when
  the database is closed.

\item \code{statement-cache} stores a Scheme record:\newline
  \code{(define-record-type cache (fields (immutable ht) (mutable
    waketime) (mutable lazy-statements)))}\newline
  The \code{waketime} is the next time the cache will attempt to
  remove dead entries.

  The hash table, \code{ht}, maps SQL strings to a Scheme
  record:\newline \code{(define-record-type entry (fields
    (immutable stmt) (mutable timestamp)))}\newline

  When a SQL string is not found in the cache,
  \code{osi\_prepare\_statement} is used with the
  \code{current-database} to make a SQLite statement. The raw
  statement handle is stored in a Scheme record:\newline
  \code{(define-record-type statement (fields (immutable handle)
    (immutable database)))}\newline
  The statement is finalized using
  \code{FinalizeStatement} when it is removed from the
  cache. \code{CloseDatabase} will finalize any remaining
  statements associated with the database.

  When a SQL string is found in the cache, the entry's
  \code{timestamp} is updated. Entries older than 5 minutes will
  be removed from the cache.

  Accessing the cache may exit with reason reason
  \code{\#(db-error prepare \var{error} \var{sql})}, where
  \var{error} is a SQLite error pair.

  The \code{lazy-statements} list contains \code{statement}
  records created by \code{lazy-execute}. These statements are
  finalized when a transaction completes.

\end{itemize}

\genserver{db}{init} The \code{init} procedure takes a filename and
mode symbol and attempts to open that database, setting
\code{journal\_mode} to ``wal'' if \var{mode} is
\code{create}. The handle returned from \code{osi\_open\_database} is
wrapped in a \var{database} record that is registered with a
foreign-handle guardian using the type name \code{databases}.
The foreign-handle guardian hooks
the garbage collector so that dead databases are
closed even if the \code{db} gen-server fails to close them for any
reason.

The gen-server traps exits so that it can close the database in its
\code{terminate} procedure.

\genserver{db}{terminate} The \code{terminate} flushes the queue and
closes the database.

\genserver{db}{handle-call} The \code{handle-call} procedure
processes the following messages:

\antipar\begin{itemize}

\item \code{\#(transaction \var{f})}: Add this transaction along
  with the \var{from} argument to \code{handle-call} to the queue.
  Process the queue.

\item \code{filename}: Return the database filename.

\item \code{stop}: Flush the queue and stop with reason
  \code{normal}, returning \code{stopped} to the caller.

\end{itemize}

\genserver{db}{handle-cast} The \code{handle-cast} procedure
processes the following message:

\antipar\begin{itemize}

\item \code{\#(log \var{sql} \var{bindings})}: Add this tuple to
  the queue. Process the queue.

\end{itemize}

\genserver{db}{handle-info} The \code{handle-info} procedure
processes the following messages:

\antipar\begin{itemize}

\item \code{timeout}: Remove old entries from the statement cache.

\item \code{\#(EXIT \var{worker-pid} normal)}: The worker finished
  the previous request successfully. Process the queue.

\item \code{\#(EXIT \var{worker-pid} \var{reason})}: The worker
  failed to process the previous request. Flush the queue and stop
  with \var{reason}.

\end{itemize}

\section {Design Decisions}

There is a one-to-one relationship between a SQLite database handle
and the \code{db} gen-server. For clarity, the database handle and a
SQLite statement cache are implemented in terms of Erlang process
dictionary parameters.

An alternate approach for logging was already explored where a
transaction was not lazily opened. Such an approach means that when a
third party tool tries to access the database, it will hang until the
transaction is complete.

A commit threshold of 10,000 was chosen because it was large enough to
minimize the cost of a transaction but small enough to execute simple
queries in less than one second.

\section {Programming Interface}

\defineentry{db:start\&link}
\begin{procedure}
  \code{(db:start\&link \var{name} \var{filename} \var{mode})}
\end{procedure}
\returns{}
\code{\#(ok \var{pid})} $|$
\code{\#(error \var{error})}

The \code{db:start\&link} procedure creates a new \code{db}
gen-server using \code{gen-server:start\&link}.

\var{name} is the registered name of the process. For an anonymous
server, \code{\#f} may be specified.

\var{filename} is the path to a SQLite database.

\var{mode} is one of the following symbols used to pass SQLite flags
to \code{osi\_open\_database}:

\antipar\begin{itemize}

\item \code{read-only} uses the SQLite flag
  \code{SQLITE\_OPEN\_READONLY}.

\item \code{open} uses the SQLite flag
  \code{SQLITE\_OPEN\_READWRITE}.

\item \code{create} combines the SQLite flags \code{(logor
  SQLITE\_OPEN\_READWRITE \code{SQLITE\_OPEN\_CREATE})}.
\end{itemize}

The SQLite constants can be found in \texttt{sqlite3.h} or
online~\cite{sqlite}.

This procedure may return an \var{error} of \code{\#(db-error open
  \var{error} \var{filename})}, where \var{error} is a SQLite error
pair.

\defineentry{db:stop}
\begin{procedure}
  \code{(db:stop \var{who})}
\end{procedure}
\returns{}
\code{stopped}

The \code{db:stop} procedure calls \code{(gen-server:call
  \var{who} stop infinity)}.

\defineentry{with-db}
\begin{syntax}
  \code{(with-db [\var{db} \var{filename} \var{flags}] \var{body\(\sb{1}\)} \var{body\(\sb{2}\)} \etc)}
\end{syntax}
\expandsto{} \antipar\begin{alltt}
(let ([\var{db} (sqlite:open \var{filename} \var{flags})])
  (on-exit (sqlite:close \var{db})
    \var{body\(\sb{1}\)} \var{body\(\sb{2}\)} ...))
\end{alltt}

The \code{with-db} macro opens the database in \var{filename},
executes the statements in the body, and closes the database before
exiting.  This is a suitable alternative to starting a
\code{gen-server} when you need to query a database using a separate
SQLite connection, and you do not need to cache prepared SQL
statements.

\defineentry{db:filename}
\begin{procedure}
  \code{(db:filename \var{who})}
\end{procedure}
\returns{} the database filename

The \code{db:filename} procedure calls \code{(gen-server:call
  \var{who} filename)}.

\defineentry{db:log}
\begin{procedure}
  \code{(db:log \var{who} \var{sql} . \var{bindings})}
\end{procedure}
\returns{}
\code{ok}

The \code{db:log} procedure calls \code{(gen-server:cast \var{who}
  \#(log \var{sql} \var{bindings}))}.  \var{sql} is a SQL string, and
\var{bindings} is a list of values to be bound in the query.
Because \code{db:log} does not wait for a reply from the server, any
error in processing the request will crash the server.

\defineentry{db:transaction}
\begin{procedure}
  \code{(db:transaction \var{who} \var{f})}
\end{procedure}
\returns{}
\code{\#(ok \var{result})} $|$
\code{\#(error \var{error})}

The \code{db:transaction} procedure calls \code{(gen-server:call
  \var{who} \#(transaction \var{f}) infinity)}.

\var{f} is a thunk which returns a single value,
\var{result}. \code{execute}, \code{lazy-execute}, and
\code{columns} can be used inside the procedure \var{f}.

\var{result} is the successful return value of \var{f}. Typically,
this is a list of rows as returned by a \code{SELECT} query.

\var{error} is the failure reason of \var{f}.

\defineentry{transaction}
\begin{syntax}
  \code{(transaction \var{db} \var{body} \etc)}
\end{syntax}
\expandsto{} \antipar\begin{alltt}
(match (db:transaction \var{db} (lambda () \var{body} \etc))
  [#(ok ,result) result]
  [#(error ,reason) (raise reason)])
\end{alltt}

The \code{transaction} macro runs the body in a transaction and
returns the result when successful and exits when unsuccessful.

\defineentry{execute}
\begin{procedure}
  \code{(execute \var{sql} . \var{bindings})}
\end{procedure}
\returns{}
a list of rows where each row is a vector of data in column order as
specified in the \var{sql} statement

\code{execute} should only be used from within a thunk \var{f}
provided to \code{db:transaction}.

\var{sql} is mapped to a SQLite statement using the
\code{statement-cache}. The \var{bindings} are then applied using
\code{BindStatement}. The statement is then executed using
\code{StepStatement}. The results are accumulated as a list, and the
statement is reset using \code{ResetStatement} to prevent the
statement from locking parts of the database.

This procedure may exit with reason \code{\#(db-error prepare
  \var{error} \var{sql})}, where \var{error} is a SQLite error pair.

\defineentry{lazy-execute}
\begin{procedure}
  \code{(lazy-execute \var{sql} . \var{bindings})}
\end{procedure}
\returns{}
a thunk

\code{lazy-execute} should only be used from within a thunk \var{f}
provided to \code{db:transaction}.

A new SQLite statement is created from \var{sql} using
\code{osi\_prepare\_statement} so that the statement won't interfere with
any other queries. The statement is added to the
\code{lazy-statements} list of the \code{statement-cache} and is
finalized when the transaction completes.  The \var{bindings} are then
applied using \code{BindStatement}. A thunk is returned which, when
called, executes the statement using \code{StepStatement}. The thunk
returns one row of data or \code{\#f}.

This procedure may exit with reason \code{\#(db-error prepare
  \var{error} \var{sql})}, where \var{error} is a SQLite error pair.

\defineentry{execute-sql}
\begin{procedure}
  \code{(execute-sql \var{db} \var{sql} . \var{bindings})}
\end{procedure}
\returns{}
a list of rows where each row is a vector of data in column order as
specified in the \var{sql} statement

\code{execute-sql} should only be used for statements that do not need to be inside a transaction, such as a one-time query.

\var{sql} is prepared into a SQLite statement for use with \var{db}, executed via \code{sqlite:execute} with the specified \var{bindings}, and finalized.

This procedure may exit with reason \code{\#(db-error prepare
  \var{error} \var{sql})}, where \var{error} is a SQLite error pair.

\defineentry{columns}
\begin{procedure}
  \code{(columns \var{sql})}
\end{procedure}
\returns{}
a vector of column names in order as specified in the \var{sql} statement

\code{columns} should only be used from within a thunk \var{f}
provided to \code{db:transaction}.

\var{sql} is mapped to a SQLite statement using the
\code{statement-cache}. The statement columns are then retrieved
using \code{GetStatementColumns}.

\defineentry{parse-sql}
\begin{procedure}\code{(parse-sql \var{x} \opt{\var{symbol->sql}})}\end{procedure}
\returns{} two values: a query string and a list of syntax objects for
the arguments

The \code{parse-sql} procedure is used by macro transformers to take
syntax object \var{x} and produce a query string and associated
arguments according to the patterns below.
When one of these patterns is matched, the \var{symbol->sql} procedure is
applied to the remaining symbols of the input before they are spliced into the
query string, as if by \code{\fixtilde(format "~a" (symbol->sql sym))}.
By default, \var{symbol->sql} is the identity function.

\begin{itemize}

\item \code{(insert \var{table} ([\var{column} \var{e\(\sb{1}\)}
    \var{e\(\sb{2}\)} \etc{}] \etc{})}

  The \code{insert} form generates a SQL insert statement. The
  \var{table} and \var{column} patterns are SQL identifiers. Any
  \var{e} expression that is \code{(unquote \var{exp})} is converted
  to \code{?} in the query, and \var{exp} is added to the list of
  arguments. All other expressions are spliced into the query string.

\item \code{(update \var{table} ([\var{column} \var{e\(\sb{1}\)}
    \var{e\(\sb{2}\)} \etc{}] \etc{}) \var{where} \etc{})}

  The \code{update} form generates a SQL update statement. The
  \var{table} and \var{column} patterns are SQL identifiers. Any
  \var{e} or \var{where} expression that is \code{(unquote
    \var{exp})} is converted to \code{?} in the query, and \var{exp}
  is added to the list of arguments. All other expressions are spliced
  into the query string.

\item \code{(delete \var{table} \var{where} \etc{})}

  The \code{delete} form generates a SQL delete statement. The
  \var{table} pattern is a SQL identifier. Any \var{where} expression
  that is \code{(unquote \var{exp})} is converted to \code{?} in
  the query, and \var{exp} is added to the list of arguments. All
  other expressions are spliced into the query string.

\end{itemize}

\defineentry{database?}
\begin{procedure}
  \code{(database? \var{x})}
\end{procedure}
\returns{} a boolean

The \code{database?} procedure determines whether or not the datum
\var{x} is a database record instance.

\defineentry{database-create-time}
\begin{procedure}
  \code{(database-create-time \var{db})}
\end{procedure}
\returns{} a clock time in milliseconds

The \code{database-create-time} procedure returns the clock time from
\code{erlang:now} when database record instance \var{db} was created.

\defineentry{database-filename}
\begin{procedure}
  \code{(database-filename \var{db})}
\end{procedure}
\returns{} a string

The \code{database-filename} procedure returns the filename of
database record instance \var{db}.

\defineentry{database-count}
\begin{procedure}
  \code{(database-count)}
\end{procedure}
\returns{} the number of open databases

The \code{database-count} procedure returns the number of open
databases.
This is the procedure returned by \code{(foreign-handle-count\ 'databases)}.

\defineentry{print-databases}
\begin{procedure}
  \code{(print-databases \opt{\var{op}})}
\end{procedure}
\returns{} unspecified

The \code{print-databases} procedure prints information about all open
databases to textual output port \var{op}, which defaults to the
current output port.
This is the procedure returned by \code{(foreign-handle-print\ 'databases)}.

\defineentry{statement?}
\begin{procedure}
  \code{(statement? \var{x})}
\end{procedure}
\returns{} a boolean

The \code{statement?} procedure determines whether or not the datum
\var{x} is a statement record instance.

\defineentry{statement-create-time}
\begin{procedure}
  \code{(statement-create-time \var{stmt})}
\end{procedure}
\returns{} a clock time in milliseconds

The \code{statement-create-time} procedure returns the clock time from
\code{erlang:now} when statement record instance \var{stmt} was
created.

\defineentry{statement-database}
\begin{procedure}
  \code{(statement-database \var{stmt})}
\end{procedure}
\returns{} a string

The \code{statement-database} procedure returns the database record
instance of the statement record instance \var{stmt}.

\defineentry{statement-sql}
\begin{procedure}
  \code{(statement-sql \var{stmt})}
\end{procedure}
\returns{} a string

The \code{statement-sql} procedure returns the SQL string of the
statement record instance \var{stmt}.

\defineentry{statement-count}
\begin{procedure}
  \code{(statement-count)}
\end{procedure}
\returns{} the number of unfinalized statements

The \code{statement-count} procedure returns the number of unfinalized
statements.
This is the procedure returned by \code{(foreign-handle-count\ 'statements)}.

\defineentry{print-statements}
\begin{procedure}
  \code{(print-statements \opt{\var{op}})}
\end{procedure}
\returns{} unspecified

The \code{print-statements} procedure prints information about all
unfinalized statements to textual output port \var{op}, which defaults
to the current output port.
This is the procedure returned by \code{(foreign-handle-print\ 'statements)}.

\defineentry{sqlite:bind}
\begin{procedure}
  \code{(sqlite:bind \var{stmt} \var{bindings})}
\end{procedure}
\returns{} unspecified

The \code{sqlite:bind} procedure binds the variables in statement
record instance \var{stmt} with the list of \var{bindings}. It resets
the statement before binding the variables.

\defineentry{sqlite:clear-bindings}
\begin{procedure}
  \code{(sqlite:clear-bindings \var{stmt})}
\end{procedure}
\returns{} unspecified

The \code{sqlite:clear-bindings} procedure clears the variable
bindings in statement record instance \var{stmt}.

\defineentry{sqlite:close}
\begin{procedure}
  \code{(sqlite:close \var{db})}
\end{procedure}
\returns{} unspecified

The \code{sqlite:close} procedure closes the database associated with
database record instance \var{db}.

\defineentry{sqlite:columns}
\begin{procedure}
  \code{(sqlite:columns \var{stmt})}
\end{procedure}
\returns{} a vector of column names

The \code{sqlite:columns} procedure returns a vector of column names
for the statement record instance \var{stmt}.

\defineentry{sqlite:execute}
\begin{procedure}
  \code{(sqlite:execute \var{stmt} \var{bindings})}
\end{procedure}
\returns{} a list of rows where each row is a vector of data in column order

The \code{sqlite:execute} procedure calls \code{(sqlite:bind
  \var{stmt} \var{bindings})} to bind any variables and then
iteratively calls \code{(sqlite:step \var{stmt})} to build the
resulting list of rows. It resets the statement when the procedure
exits.

\defineentry{sqlite:expanded-sql}
\begin{procedure}
  \code{(sqlite:expanded-sql \var{stmt})}
\end{procedure}
\returns{} a string

The \code{sqlite:expanded-sql} procedure returns the SQL string
expanded with the binding values for the statement record instance
\var{stmt}.

\defineentry{sqlite:finalize}
\begin{procedure}
  \code{(sqlite:finalize \var{stmt})}
\end{procedure}
\returns{} unspecified

The \code{sqlite:finalize} procedure finalizes the statement record
instance \var{stmt}.

\defineentry{sqlite:interrupt}
\begin{procedure}
  \code{(sqlite:interrupt \var{db})}
\end{procedure}
\returns{} unspecified

The \code{sqlite:interrupt} procedure interrupts any pending
operations on the database associated with database record instance
\var{db}.

\defineentry{sqlite:last-insert-rowid}
\begin{procedure}
  \code{(sqlite:last-insert-rowid \var{db})}
\end{procedure}
\returns{} unspecified

The \code{sqlite:last-insert-rowid} procedure returns the rowid of the
most recent successful insert into a rowid table or virtual table on
the database associated with database record instance \var{db}. It
returns 0 if no such insert has occurred.

\defineentry{sqlite:open}
\begin{procedure}
  \code{(sqlite:open \var{filename} \var{flags})}
\end{procedure}
\returns{} a database record instance

The \code{sqlite:open} procedure opens the SQLite database in file
\var{filename} with \var{flags} specified by
\code{sqlite3\_open\_v2}~\cite{sqlite}. The constants
\code{SQLITE\_OPEN\_CREATE}, \code{SQLITE\_OPEN\_READONLY}, and
\code{SQLITE\_OPEN\_READWRITE} are exported from the \code{(swish db)}
library.
The \code{sqlite:open} procedure registers the database record
with a foreign-handle guardian using the type name \code{databases}.\index{database guardian}

\defineentry{sqlite:prepare}
\begin{procedure}
  \code{(sqlite:prepare \var{db} \var{sql})}
\end{procedure}
\returns{} a statement record instance

The \code{sqlite:prepare} procedure returns a statement record
instance for the \var{sql} statement in the database record instance
\var{db}.
The \code{sqlite:prepare} procedure registers the statement record
with a foreign-handle guardian using the type name \code{statements}.\index{statement guardian}

\defineentry{sqlite:sql}
\begin{procedure}
  \code{(sqlite:sql \var{stmt})}
\end{procedure}
\returns{} a string

The \code{sqlite:sql} procedure returns the unexpanded SQL string for
the statement record instance \var{stmt}.

\defineentry{sqlite:step}
\begin{procedure}
  \code{(sqlite:step \var{stmt})}
\end{procedure}
\returns{} a vector of data in column order or \code{\#f}

The \code{sqlite:step} procedure steps the statement record instance
\var{stmt} and returns the next row vector in column order or
\code{\#f} if there are no more rows.
