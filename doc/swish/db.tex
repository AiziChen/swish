% Copyright 2018 Beckman Coulter, Inc.
%
% Permission is hereby granted, free of charge, to any person
% obtaining a copy of this software and associated documentation files
% (the "Software"), to deal in the Software without restriction,
% including without limitation the rights to use, copy, modify, merge,
% publish, distribute, sublicense, and/or sell copies of the Software,
% and to permit persons to whom the Software is furnished to do so,
% subject to the following conditions:
%
% The above copyright notice and this permission notice shall be
% included in all copies or substantial portions of the Software.
%
% THE SOFTWARE IS PROVIDED "AS IS", WITHOUT WARRANTY OF ANY KIND,
% EXPRESS OR IMPLIED, INCLUDING BUT NOT LIMITED TO THE WARRANTIES OF
% MERCHANTABILITY, FITNESS FOR A PARTICULAR PURPOSE AND
% NONINFRINGEMENT. IN NO EVENT SHALL THE AUTHORS OR COPYRIGHT HOLDERS
% BE LIABLE FOR ANY CLAIM, DAMAGES OR OTHER LIABILITY, WHETHER IN AN
% ACTION OF CONTRACT, TORT OR OTHERWISE, ARISING FROM, OUT OF OR IN
% CONNECTION WITH THE SOFTWARE OR THE USE OR OTHER DEALINGS IN THE
% SOFTWARE.

\chapter {Database Interface}\label{chap:db}

\section {Introduction}

The database (\code{db}) interface is a gen-server which provides a
basic transaction framework to retrieve and store data in a SQLite
database. It provides functions to use transactions (directly and
lazily).

The low-level SQLite interface can be found in the operating system
interface design (see Chapter~\ref{chap:osi}).

Other SQLite resources are available online~\cite{sqlite} or in The
Definitive Guide to SQLite~\cite{sqlite-guide}.

\section {Theory of Operation}

The \code{db} gen-server serializes internal requests to the
database.  For storage and retrieval of data, each transaction is
processed in turn by a separate monitored \var{worker} process.  The gen-server does
not block waiting for this process to finish so that it can maintain
linear performance by keeping its inbox short. The return value of the
transaction is returned to the caller or an error is generated without
tearing down the gen-server.

To facilitate logging, the \code{db} gen-server can execute SQL statements
asynchronously. It enqueues SQL statements submitted via \code{db:log} and
executes them by opening a transaction lazily when an explicit transaction is
enqueued, when a \var{worker} process exits normally, or when the \code{db}
message queue is empty and \code{commit-delay} has elapsed since enqueuing
a \code{db:log} request in an empty queue.
To maintain responsiveness, each lazy transaction commits at most \code{commit-limit} \code{db:log} requests.
See \code{db:options} on page~\pageref{db:options} for details.
By default, each database is created with write-ahead logging enabled
to prevent write operations from blocking on queries made from another
connection.

SQLite has three types of transactions: deferred, immediate, and
exclusive. This interface uses only immediate transactions to simplify
the handling of the \code{SQLITE\_BUSY} error.  Using immediate
transactions means that \code{SQLITE\_BUSY} will only occur during
\code{BEGIN IMMEDIATE}, \code{BEGIN TRANSACTION}, \code{COMMIT},
and \code{ROLLBACK}\footnote{Our testing showed that
  \code{ROLLBACK} returns \code{SQLITE\_BUSY} only when a
  \code{COMMIT} for the same transaction returned
  \code{SQLITE\_BUSY}. This framework never causes that situation to
  occur, but it guards against it anyway.} statements. For each of
these statements, when a \code{SQLITE\_BUSY} occurs, the code waits
for a brief time, then retries the statement. The wait times in
milliseconds follow the pattern \code{(2 3 6 11 16 21 26 26 26 51
  51 . \#0=(101 . \#0\#))}, and up to 500 retries are attempted before
exiting with \code{\#(db-retry-failed \var{sql} \var{count})}.  When
the retry count is positive, it is logged to the event manager along
with the total duration with a \code{<transaction-retry>} event.

\begin{pubevent}{<transaction-retry>}
  \argrow{timestamp}{timestamp from \code{erlang:now}}
  \argrow{database}{database filename}
  \argrow{duration}{duration in milliseconds}
  \argrow{count}{retry count}
  \argrow{sql}{query}
\end{pubevent}

The \code{db} gen-server uses the operating system interface to
interact with SQLite. To prevent memory leaks, each raw database and
statement handle is wrapped in a Scheme record and registered with a
guardian via \code{make-foreign-handle-guardian}.

\paragraph* {state}\index{db!state}
\code{(define-state-tuple <db-state> filename db cache queue worker)}
\begin{itemize}
\item \code{filename} is the database specified when the server was
  started.
\item \code{db} is the database record.
\item \code{cache} is a hash table mapping SQL strings to SQLite
  prepared statements.
\item \code{queue} is a queue of log and transaction requests.
\item \code{worker} is the pid of the active worker or \code{\#f}.
\end{itemize}

\paragraph* {dictionary parameters}\index{db!parameters}
\begin{itemize}

\item \code{current-database} stores a Scheme record:
  \begin{alltt}
(define-record-type database
  (fields
   (immutable filename)
   (immutable create-time)
   (mutable handle)))
  \end{alltt}\antipar
  The \code{handle} is set to \code{\#f} when the database is closed.

\item \code{statement-cache} stores a Scheme record:
  \begin{alltt}
(define-record-type cache
  (fields
   (immutable ht)
   (immutable expire-timeout)
   (mutable waketime)
   (mutable lazy-objects)))
  \end{alltt}\antipar
  The \code{expire-timeout} is the duration in milliseconds that
  entries live in the cache. This is configurable using the
  \code{cache-timeout} option.

  The \code{waketime} is the next time the cache will attempt to
  remove dead entries.

  The hash table, \code{ht}, maps SQL strings to a Scheme record:
  \begin{alltt}
(define-record-type entry
  (fields
   (immutable stmt)
   (mutable timestamp)))
  \end{alltt}\antipar

  When a SQL string is not found in the cache,
  \code{osi\_prepare\_statement} is used with the
  \code{current-database} to make a SQLite statement. The raw
  statement handle is stored in a Scheme record:
  \begin{alltt}
(define-record-type statement
  (fields
   (immutable database)
   (immutable sql)
   (immutable create-time)
   (mutable handle)))
  \end{alltt}\antipar
  The statement is finalized using \code{osi\_finalize\_statement}
  when it is removed from the cache. \code{osi\_close\_database} will
  finalize any remaining statements associated with the database.

  When a SQL string is found in the cache, the entry's
  \code{timestamp} is updated. Entries older than 5 minutes will
  be removed from the cache.

  Accessing the cache may exit with reason reason
  \code{\#(db-error prepare \var{error} \var{sql})}, where
  \var{error} is a SQLite error pair.

  The \code{lazy-objects} list contains \code{statement}
  and marshaled \code{bindings} records created by \code{lazy-execute}.
  These records are finalized when a transaction completes.

\end{itemize}

\genserver{db}{init} The \code{init} procedure takes a filename, mode
symbol, and an initialization procedure and attempts to open that
database and invoke the initialization procedure.  The handle returned
from \code{osi\_open\_database} is wrapped in a \var{database} record
that is registered with a foreign-handle guardian using the type name
\code{databases}.  The foreign-handle guardian hooks the garbage
collector so that dead databases are closed even if the \code{db}
gen-server fails to close them for any reason.

The gen-server traps exits so that it can close the database in its
\code{terminate} procedure.

\genserver{db}{terminate} The \code{terminate} flushes the queue and
closes the database.

\genserver{db}{handle-call} The \code{handle-call} procedure
processes the following messages:

\antipar\begin{itemize}

\item \code{\#(transaction \var{f})}: Add this transaction along
  with the \var{from} argument to \code{handle-call} to the queue.
  Process the queue.

\item \code{filename}: Return the database filename.

\item \code{stop}: Flush the queue and stop with reason
  \code{normal}, returning \code{stopped} to the caller.

\end{itemize}

\genserver{db}{handle-cast} The \code{handle-cast} procedure
processes the following message:

\antipar\begin{itemize}

\item \code{\#(<log> \var{sql} \var{mbindings})}: Add this tuple to the
  queue. Process the queue.

\end{itemize}

\genserver{db}{handle-info} The \code{handle-info} procedure
processes messages matching the following patterns:

\antipar\begin{itemize}

\item \code{timeout}: If the request queue is empty, remove old
  entries from the statement cache. Process the queue.

\item \code{`(DOWN \_ \var{worker-pid} \var{reason} \var{e})}: The
  worker finished the previous request. If successful, process the
  queue. Otherwise, flush the queue and stop with the fault \var{e}.

\item \code{`(DOWN \_ \_ \_)}: Ignore the unexpected \code{DOWN}
  message. Process the queue.

\item \code{`(EXIT \var{pid} \_ \var{e})}: If the \var{pid} is the
  worker, ignore the message. Do not update the state. A follow-up
  \code{DOWN} message will process the queue. Otherwise, flush the
  queue, and stop with the fault \var{e}.

\end{itemize}

\section {Design Decisions}

There is a one-to-one relationship between a SQLite database handle
and the \code{db} gen-server. For clarity, the database handle and a
SQLite statement cache are implemented in terms of Erlang process
dictionary parameters.

An alternate approach for logging was already explored where a
transaction was not lazily opened. Such an approach means that when a
third party tool tries to access the database, it will hang until the
transaction is complete.

A commit threshold of 10,000 was chosen because it was large enough to
minimize the cost of a transaction but small enough to execute simple
queries in less than one second.

Version 2.1.0 adds the concept of marshaled bindings. Bindings are
copied into the C heap. The resulting handle is wrapped in a Scheme
record and registered with a guardian via
\code{make-foreign-handle-guardian}. The database worker process uses
marshaled bindings to invoke \code{sqlite:bulk-execute} when processing log
messages.

\section {Programming Interface}

\defineentry{db:start\&link}
\begin{procedure}
  \code{(db:start\&link \var{name} \var{filename} \var{mode} \opt{\var{db-init}})}\\
  \code{(db:start\&link \var{name} \var{filename} \var{mode} \opt{\var{db-options}})}
\end{procedure}
\returns{}
\code{\#(ok \var{pid})} $|$
\code{\#(error \var{error})}

The \code{db:start\&link} procedure creates a new \code{db}
gen-server using \code{gen-server:start\&link}.

\var{name} is the registered name of the process. For an anonymous
server, \code{\#f} may be specified.

\var{filename} is the path to a SQLite database.

\var{mode} is one of the following symbols used to pass SQLite flags
to \code{osi\_open\_database}:

\antipar\begin{itemize}

\item \code{read-only} uses the SQLite flag
  \code{SQLITE\_OPEN\_READONLY}.

\item \code{open} uses the SQLite flag
  \code{SQLITE\_OPEN\_READWRITE}.

\item \code{create} combines the SQLite flags \code{(logor
  SQLITE\_OPEN\_READWRITE \code{SQLITE\_OPEN\_CREATE})}.
\end{itemize}

The SQLite constants can be found in \texttt{sqlite3.h} or
online~\cite{sqlite}.

\var{db-init} is a procedure that takes one argument, a database
record instance. The return value is ignored. When the \var{mode} is
\code{create} and \var{filename} is not a special SQLite filename, the
default procedure sets \code{journal\_mode} to ``wal''; otherwise, no
additional initialization occurs.

\var{db-options} can be defined using
\code{(db:options [\var{option} \var{value}] \etc)}.
The following options may be used:
\defineentry{db:options}
\phantomsection % make pageref go to correct page for this label
\label{db:options}

\begin{tabular}{lp{5em}p{.65\textwidth}}
  option & default & description \\ \hline

  \code{init}
  & see right
  & a procedure, \code{(lambda (filename mode db) \etc)},
  called when initializing the gen-server,
  where \var{db} is a database record instance;
  the default \code{init} procedure is equivalent to
  the default \var{db-init} procedure described above \\

  \code{cache-timeout}
  & 5 minutes
  & a nonnegative fixnum; the number of milliseconds before
  unreferenced statements expire from the statement cache \\

  \code{commit-delay}
  & 0
  & a nonnegative fixnum; the number of milliseconds
  to wait before opening a lazy transaction \\

  \code{commit-limit}
  & 10,000
  & a positive fixnum; the maximum number of \code{db:log}
  entries to include when opening a lazy transaction \\
\end{tabular}

The \code{db:start\&link} procedure may return an \var{error} of
\code{\#(db-error open
  \var{error} \var{filename})}, where \var{error} is a SQLite error
pair.

\defineentry{db:start}
\begin{procedure}
  \code{(db:start \var{name} \var{filename} \var{mode} \opt{\var{db-init}})}\\
  \code{(db:start \var{name} \var{filename} \var{mode} \opt{\var{db-options}})}
\end{procedure}
\returns{}
\code{\#(ok \var{pid})} $|$
\code{\#(error \var{error})}

\code{db:start} behaves the same as \code{db:start\&link} except that
it does not link to the calling process.

\defineentry{db:stop}
\begin{procedure}
  \code{(db:stop \var{who})}
\end{procedure}
\returns{}
\code{stopped}

The \code{db:stop} procedure calls \code{(gen-server:call
  \var{who} stop infinity)}.

\defineentry{with-db}
\begin{syntax}
  \code{(with-db [\var{db} \var{filename} \var{flags}] \var{body\(\sb{1}\)} \var{body\(\sb{2}\)} \etc)}
\end{syntax}
\expandsto{} \antipar\begin{alltt}
(let ([\var{db} (sqlite:open \var{filename} \var{flags})])
  (on-exit (sqlite:close \var{db})
    \var{body\(\sb{1}\)} \var{body\(\sb{2}\)} \etc))
\end{alltt}

The \code{with-db} macro opens the database in \var{filename},
executes the statements in the body, and closes the database before
exiting.  This is a suitable alternative to starting a
\code{gen-server} when you need to query a database using a separate
SQLite connection, and you do not need to cache prepared SQL
statements.

\defineentry{db:expire-cache}
\begin{procedure}
  \code{(db:expire-cache \var{who})}
\end{procedure}
\returns{} unspecified

The \code{db:expire-cache} procedure enqueues a request to remove
entries from the statement cache regardless of their expiration
time. \code{BEGIN IMMEDIATE} and \code{COMMIT} remain in the cache
because they are used frequently.

\defineentry{db:filename}
\begin{procedure}
  \code{(db:filename \var{who})}
\end{procedure}
\returns{} the database filename

The \code{db:filename} procedure calls \code{(gen-server:call
  \var{who} filename)}.

\defineentry{db:log}
\begin{procedure}
  \code{(db:log \var{who} \var{sql} . \var{bindings})}
\end{procedure}
\returns{}
\code{ok}

The \code{db:log} procedure calls \code{(gen-server:cast \var{who}
  \#(<log> \var{sql} \var{mbindings}))}, where \var{sql} is a SQL string,
\var{bindings} is a list of values to be bound in the query, and
\var{mbindings} is the result of \code{(sqlite:marshal-bindings \var{bindings})}.
Because \code{db:log} does not wait for a reply from the server, any
error in processing the request will crash the server.

\defineentry{db:transaction}
\begin{procedure}
  \code{(db:transaction \var{who} \var{f})}
\end{procedure}
\returns{}
\code{\#(ok \var{result})} $|$
\code{\#(error \var{error})}

The \code{db:transaction} procedure calls \code{(gen-server:call
  \var{who} \#(transaction \var{f}) infinity)}.

\var{f} is a thunk which returns a single value,
\var{result}. \code{execute}, \code{lazy-execute}, and
\code{columns} can be used inside the procedure \var{f}.

\var{result} is the successful return value of \var{f}. Typically,
this is a list of rows as returned by a \code{SELECT} query.

\var{error} is the failure reason of \var{f}.

\defineentry{transaction}
\begin{syntax}
  \code{(transaction \var{db} \var{body} \etc)}
\end{syntax}
\expandsto{} \antipar\begin{alltt}
(match (db:transaction \var{db} (lambda () \var{body} \etc))
  [#(ok ,result) result]
  [#(error ,reason) (throw reason)])
\end{alltt}

The \code{transaction} macro runs the body in a transaction and
returns the result when successful and exits when unsuccessful.

\defineentry{execute}
\begin{procedure}
  \code{(execute \var{sql} . \var{bindings})}
\end{procedure}
\returns{}
a list of rows where each row is a vector of data in column order as
specified in the \var{sql} statement

\code{execute} should only be used from within a thunk \var{f}
provided to \code{db:transaction}.

\var{sql} is mapped to a SQLite statement using the
\code{statement-cache}. The \var{bindings} are then applied using
\code{osi\_bind\_statement}. The statement is then executed using
\code{osi\_step\_statement}. The results are accumulated as a list, and the
statement is reset using \code{osi\_reset\_statement} to prevent the
statement from locking parts of the database.

This procedure may exit with reason \code{\#(db-error prepare
  \var{error} \var{sql})}, where \var{error} is a SQLite error pair.

\defineentry{lazy-execute}
\begin{procedure}
  \code{(lazy-execute \var{sql} . \var{bindings})}
\end{procedure}
\returns{}
a thunk

\code{lazy-execute} should only be used from within a thunk \var{f}
provided to \code{db:transaction}.

A new SQLite statement is created from \var{sql} using
\code{osi\_prepare\_statement} so that the statement won't interfere with
any other queries. The statement is added to the
\code{lazy-objects} list of the \code{statement-cache} and is
finalized when the transaction completes.  The \var{bindings} are marshaled
via \code{sqlite:marshal-bindings}. The resulting bindings record instance
is added to the \code{lazy-objects} list and applied
using \code{osi\_bind\_statement\_bindings}. A thunk is returned which, when
called, executes the statement using \code{osi\_step\_statement}. The thunk
returns one row of data or \code{\#f}.

This procedure may exit with reason \code{\#(db-error prepare
  \var{error} \var{sql})}, where \var{error} is a SQLite error pair.

\defineentry{execute-sql}
\begin{procedure}
  \code{(execute-sql \var{db} \var{sql} . \var{bindings})}
\end{procedure}
\returns{}
a list of rows where each row is a vector of data in column order as
specified in the \var{sql} statement

\code{execute-sql} should only be used for statements that do not need to be inside a transaction, such as a one-time query.

\var{sql} is prepared into a SQLite statement for use with \var{db}, executed via \code{sqlite:execute} with the specified \var{bindings}, and finalized.

This procedure may exit with reason \code{\#(db-error prepare
  \var{error} \var{sql})}, where \var{error} is a SQLite error pair.

\defineentry{columns}
\begin{procedure}
  \code{(columns \var{sql})}
\end{procedure}
\returns{}
a vector of column names in order as specified in the \var{sql} statement

\code{columns} should only be used from within a thunk \var{f}
provided to \code{db:transaction}.

\var{sql} is mapped to a SQLite statement using the
\code{statement-cache}. The statement columns are then retrieved
using \code{osi\_get\_statement\_columns}.

\defineentry{parse-sql}
\begin{procedure}\code{(parse-sql \var{x} \opt{\var{symbol->sql}})}\end{procedure}
\returns{} two values: a query string and a list of syntax objects for
the arguments

The \code{parse-sql} procedure is used by macro transformers to take
syntax object \var{x} and produce a query string and associated
arguments according to the patterns below.
When one of these patterns is matched, the \var{symbol->sql} procedure is
applied to the remaining symbols of the input before they are spliced into the
query string, as if by \code{\fixtilde(format "~a" (symbol->sql sym))}.
By default, \var{symbol->sql} is the identity function.

\begin{itemize}

\item \code{(insert \var{table} ([\var{column} \var{e\(\sb{1}\)}
    \var{e\(\sb{2}\)} \etc{}] \etc{}))}

  The \code{insert} form generates a SQL insert statement. The
  \var{table} and \var{column} patterns are SQL identifiers. Any
  \var{e} expression that is \code{(unquote \var{exp})} is converted
  to \code{?} in the query, and \var{exp} is added to the list of
  arguments. All other expressions are spliced into the query string.

\item \code{(update \var{table} ([\var{column} \var{e\(\sb{1}\)}
    \var{e\(\sb{2}\)} \etc{}] \etc{}) \var{where} \etc{})}

  The \code{update} form generates a SQL update statement. The
  \var{table} and \var{column} patterns are SQL identifiers. Any
  \var{e} or \var{where} expression that is \code{(unquote
    \var{exp})} is converted to \code{?} in the query, and \var{exp}
  is added to the list of arguments. All other expressions are spliced
  into the query string.

\item \code{(delete \var{table} \var{where} \etc{})}

  The \code{delete} form generates a SQL delete statement. The
  \var{table} pattern is a SQL identifier. Any \var{where} expression
  that is \code{(unquote \var{exp})} is converted to \code{?} in
  the query, and \var{exp} is added to the list of arguments. All
  other expressions are spliced into the query string.

\end{itemize}

\defineentry{database?}
\begin{procedure}
  \code{(database? \var{x})}
\end{procedure}
\returns{} a boolean

The \code{database?} procedure determines whether or not the datum
\var{x} is a database record instance.

\defineentry{database-create-time}
\begin{procedure}
  \code{(database-create-time \var{db})}
\end{procedure}
\returns{} a clock time in milliseconds

The \code{database-create-time} procedure returns the clock time from
\code{erlang:now} when database record instance \var{db} was created.

\defineentry{database-filename}
\begin{procedure}
  \code{(database-filename \var{db})}
\end{procedure}
\returns{} a string

The \code{database-filename} procedure returns the filename of
database record instance \var{db}.

\defineentry{database-count}
\begin{procedure}
  \code{(database-count)}
\end{procedure}
\returns{} the number of open databases

The \code{database-count} procedure returns the number of open
databases.
This is the procedure returned by \code{(foreign-handle-count\ 'databases)}.

\defineentry{print-databases}
\begin{procedure}
  \code{(print-databases \opt{\var{op}})}
\end{procedure}
\returns{} unspecified

The \code{print-databases} procedure prints information about all open
databases to textual output port \var{op}, which defaults to the
current output port.
This is the procedure returned by \code{(foreign-handle-print\ 'databases)}.

\defineentry{statement?}
\begin{procedure}
  \code{(statement? \var{x})}
\end{procedure}
\returns{} a boolean

The \code{statement?} procedure determines whether or not the datum
\var{x} is a statement record instance.

\defineentry{statement-create-time}
\begin{procedure}
  \code{(statement-create-time \var{stmt})}
\end{procedure}
\returns{} a clock time in milliseconds

The \code{statement-create-time} procedure returns the clock time from
\code{erlang:now} when statement record instance \var{stmt} was
created.

\defineentry{statement-database}
\begin{procedure}
  \code{(statement-database \var{stmt})}
\end{procedure}
\returns{} a string

The \code{statement-database} procedure returns the database record
instance of the statement record instance \var{stmt}.

\defineentry{statement-sql}
\begin{procedure}
  \code{(statement-sql \var{stmt})}
\end{procedure}
\returns{} a string

The \code{statement-sql} procedure returns the SQL string of the
statement record instance \var{stmt}.

\defineentry{statement-count}
\begin{procedure}
  \code{(statement-count)}
\end{procedure}
\returns{} the number of unfinalized statements

The \code{statement-count} procedure returns the number of unfinalized
statements.
This is the procedure returned by \code{(foreign-handle-count\ 'statements)}.

\defineentry{print-statements}
\begin{procedure}
  \code{(print-statements \opt{\var{op}})}
\end{procedure}
\returns{} unspecified

The \code{print-statements} procedure prints information about all
unfinalized statements to textual output port \var{op}, which defaults
to the current output port.
This is the procedure returned by \code{(foreign-handle-print\ 'statements)}.

\defineentry{bindings?}
\begin{procedure}
  \code{(bindings? \var{x})}
\end{procedure}
\returns{} a boolean

The \code{bindings?} procedure determines whether or not the datum
\var{x} is a marshaled bindings record instance.

\defineentry{bindings-count}
\begin{procedure}
  \code{(bindings-count)}
\end{procedure}
\returns{} the number of live marshaled bindings records

The \code{bindings-count} procedure returns the number of live marshaled
bindings records.
This is the procedure returned by \code{(foreign-handle-count\ 'bindings)}.

\defineentry{print-bindings}
\begin{procedure}
  \code{(print-bindings \opt{\var{op}})}
\end{procedure}
\returns{} unspecified

The \code{print-bindings} procedure prints information about all
live marshaled bindings records to textual output port \var{op}, which
defaults to the current output port.
This is the procedure returned by \code{(foreign-handle-print\ 'bindings)}.

\defineentry{sqlite:bind}
\begin{procedure}
  \code{(sqlite:bind \var{stmt} \var{bindings})}
\end{procedure}
\returns{} unspecified

The \code{sqlite:bind} procedure binds the variables in statement
record instance \var{stmt} with the list of \var{bindings}. It resets
the statement before binding the variables.

\defineentry{sqlite:bulk-execute}
\begin{procedure}
  \code{(sqlite:bulk-execute \var{stmts} \var{mbindings})}
\end{procedure}
\returns{} unspecified

The \code{sqlite:bulk-execute} procedure extracts the handles of the
vectors \var{stmts} and \var{mbindings} and calls
\code{osi\_bulk\_execute}.  \var{stmts} is a vector of statement
record instances, and \var{mbindings} is a vector of corresponding
marshaled bindings obtained via \code{sqlite:marshal-bindings}.

\defineentry{sqlite:clear-bindings}
\begin{procedure}
  \code{(sqlite:clear-bindings \var{stmt})}
\end{procedure}
\returns{} unspecified

The \code{sqlite:clear-bindings} procedure clears the variable
bindings in statement record instance \var{stmt}.

\defineentry{sqlite:close}
\begin{procedure}
  \code{(sqlite:close \var{db})}
\end{procedure}
\returns{} unspecified

The \code{sqlite:close} procedure closes the database associated with
database record instance \var{db}.

\defineentry{sqlite:columns}
\begin{procedure}
  \code{(sqlite:columns \var{stmt})}
\end{procedure}
\returns{} a vector of column names

The \code{sqlite:columns} procedure returns a vector of column names
for the statement record instance \var{stmt}.

\defineentry{sqlite:execute}
\begin{procedure}
  \code{(sqlite:execute \var{stmt} \var{bindings})}
\end{procedure}
\returns{} a list of rows where each row is a vector of data in column order

The \code{sqlite:execute} procedure binds any variables in statement
record instance \var{stmt} and then calls \code{(sqlite:step
  \var{stmt})} repeatedly to build the resulting list of rows.
If \var{bindings} is a marshaled bindings record instance,
then \code{sqlite:execute} calls \code{osi\_bind\_statement\_bindings}
to bind variables in the statement.
Otherwise it calls \code{sqlite:marshal-bindings}
before binding the variables and calls \code{sqlite:unmarshal-bindings}
before returning.
As a result, \code{sqlite:execute} accepts as \var{bindings} a list,
a vector, or a marshaled bindings record instance.
When the procedure exits, it resets the statement and clears the bindings.

\defineentry{sqlite:expanded-sql}
\begin{procedure}
  \code{(sqlite:expanded-sql \var{stmt})}
\end{procedure}
\returns{} a string

The \code{sqlite:expanded-sql} procedure returns the SQL string
expanded with the binding values for the statement record instance
\var{stmt}.

\defineentry{sqlite:finalize}
\begin{procedure}
  \code{(sqlite:finalize \var{stmt})}
\end{procedure}
\returns{} unspecified

The \code{sqlite:finalize} procedure finalizes the statement record
instance \var{stmt}.

\defineentry{sqlite:get-bindings}
\begin{procedure}
  \code{(sqlite:get-bindings \var{bindings})}
\end{procedure}
\returns{} a vector or \code{\#f}

The \code{sqlite:get-bindings} procedure returns a vector of the values
marshaled in the \var{bindings} record instance via \code{sqlite:marshal-bindings},
or \code{\#f} if the record has been unmarshaled by
\code{sqlite:unmarshal-bindings}.

\defineentry{sqlite:interrupt}
\begin{procedure}
  \code{(sqlite:interrupt \var{db})}
\end{procedure}
\returns{} a boolean

The \code{sqlite:interrupt} procedure interrupts any pending
operations on the database associated with database record instance
\var{db}. It returns \code{\#t} when the database is busy and
\code{\#f} otherwise.

\defineentry{sqlite:last-insert-rowid}
\begin{procedure}
  \code{(sqlite:last-insert-rowid \var{db})}
\end{procedure}
\returns{} unspecified

The \code{sqlite:last-insert-rowid} procedure returns the rowid of the
most recent successful insert into a rowid table or virtual table on
the database associated with database record instance \var{db}. It
returns 0 if no such insert has occurred.

\defineentry{sqlite:marshal-bindings}
\begin{procedure}
  \code{(sqlite:marshal-bindings \var{bindings})}
\end{procedure}
\returns{} a marshaled bindings record instance or \code{\#f}

The \code{sqlite:marshal-bindings} procedure returns a marshaled
bindings record instance for the provided list or vector of
\var{bindings}. The \code{sqlite:marshal-bindings} procedure registers
the marshaled bindings record with a foreign-handle guardian using the
type name \code{bindings}.

\defineentry{sqlite:open}
\begin{procedure}
  \code{(sqlite:open \var{filename} \var{flags})}
\end{procedure}
\returns{} a database record instance

The \code{sqlite:open} procedure opens the SQLite database in file
\var{filename} with \var{flags} specified by
\code{sqlite3\_open\_v2}~\cite{sqlite}. The constants
\code{SQLITE\_OPEN\_CREATE}, \code{SQLITE\_OPEN\_READONLY}, and
\code{SQLITE\_OPEN\_READWRITE} are exported from the \code{(swish db)}
library.
The \code{sqlite:open} procedure registers the database record
with a foreign-handle guardian using the type name \code{databases}.\index{database guardian}

\defineentry{sqlite:prepare}
\begin{procedure}
  \code{(sqlite:prepare \var{db} \var{sql})}
\end{procedure}
\returns{} a statement record instance

The \code{sqlite:prepare} procedure returns a statement record
instance for the \var{sql} statement in the database record instance
\var{db}.
The \code{sqlite:prepare} procedure registers the statement record
with a foreign-handle guardian using the type name \code{statements}.\index{statement guardian}

\defineentry{sqlite:sql}
\begin{procedure}
  \code{(sqlite:sql \var{stmt})}
\end{procedure}
\returns{} a string

The \code{sqlite:sql} procedure returns the unexpanded SQL string for
the statement record instance \var{stmt}.

\defineentry{sqlite:step}
\begin{procedure}
  \code{(sqlite:step \var{stmt})}
\end{procedure}
\returns{} a vector of data in column order or \code{\#f}

The \code{sqlite:step} procedure steps the statement record instance
\var{stmt} and returns the next row vector in column order or
\code{\#f} if there are no more rows.

\defineentry{sqlite:unmarshal-bindings}
\begin{procedure}
  \code{(sqlite:unmarshal-bindings \var{mbindings})}
\end{procedure}
\returns{} unspecified

The \code{sqlite:unmarshal-bindings} procedure deallocates the memory
associated with the marshaled bindings record instance
\var{mbindings}.
