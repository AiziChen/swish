% Copyright 2018 Beckman Coulter, Inc.
%
% Permission is hereby granted, free of charge, to any person
% obtaining a copy of this software and associated documentation files
% (the "Software"), to deal in the Software without restriction,
% including without limitation the rights to use, copy, modify, merge,
% publish, distribute, sublicense, and/or sell copies of the Software,
% and to permit persons to whom the Software is furnished to do so,
% subject to the following conditions:
%
% The above copyright notice and this permission notice shall be
% included in all copies or substantial portions of the Software.
%
% THE SOFTWARE IS PROVIDED "AS IS", WITHOUT WARRANTY OF ANY KIND,
% EXPRESS OR IMPLIED, INCLUDING BUT NOT LIMITED TO THE WARRANTIES OF
% MERCHANTABILITY, FITNESS FOR A PARTICULAR PURPOSE AND
% NONINFRINGEMENT. IN NO EVENT SHALL THE AUTHORS OR COPYRIGHT HOLDERS
% BE LIABLE FOR ANY CLAIM, DAMAGES OR OTHER LIABILITY, WHETHER IN AN
% ACTION OF CONTRACT, TORT OR OTHERWISE, ARISING FROM, OUT OF OR IN
% CONNECTION WITH THE SOFTWARE OR THE USE OR OTHER DEALINGS IN THE
% SOFTWARE.

\chapter {Introduction to Swish}\label{chap:intro-swish}

\section {Overview}

The Swish Concurrency Engine is a framework used to write
fault-tolerant programs with message-passing concurrency. It uses the
Chez Scheme~\cite{chez-scheme-users-guide} programming language and
embeds concepts from the Erlang~\cite{erlang} programming
language. Swish also provides a web server following the HTTP
protocol~\cite{RFC7230}.

Swish uses message-passing concurrency and fault isolation to
provide fault-tolerant software~\cite{armstrong-thesis,actors}. The
software is divided into lightweight processes that communicate via
asynchronous message passing but are otherwise isolated from each
other. Because processes share no mutable state, one process cannot
corrupt the state of another process---a problem that plagues software
using shared-state concurrency.

Exceptions are raised when the software detects an error and cannot
continue normal processing. If an exception is not caught by the
process that raised it, the process is terminated. An error logger
records process crashes and other software errors.

There are two mechanisms for detecting process termination,
\emph{links} and \emph{monitors}. Processes can be linked together so
that when one exits abnormally, the others are killed. A process can
monitor other processes and receive process-down messages that include
the termination reason.

A single event dispatcher receives events from the various processes
and sends them to all attached event handlers.  Event handlers filter
events based on their needs.

Swish is written in Chez Scheme
for two main reasons. First, it provides efficient first-class
continuations~\cite{one-shot,representing-control} needed to implement
lightweight processes with much less memory and CPU overhead than
operating system threads.  Second, Chez Scheme provides powerful
syntactic abstraction capabilities~\cite{syntactic-abstraction} needed
to make the code closely reflect the various aspects of the
design. For example, the message-passing system uses syntactic
abstraction to specify pattern matching succinctly.

I/O operations are performed asynchronously using C code (see
Chapter~\ref{chap:osi}), and they complete via Scheme callback
functions. Asynchronous I/O is used so that Swish can run in a single
thread without blocking for I/O. The results from asynchronous
operations are invoked synchronously by the Scheme code, allowing it
to control re-entrancy.

\section {Supervision Tree}\label{sec:default-supervision-tree}

\begin{figure}
  \center\includegraphics{swish/intro-sup-tree.pdf}
  \caption{\label{fig:intro-sup-tree}Supervision Tree}
\end{figure}

Calling \code{app:start} spawns a set of processes
organized in a supervision tree.
By default, Swish uses the supervision tree illustrated in
Figure~\ref{fig:intro-sup-tree}.
The application (see Chapter~\ref{chap:application}) is a single gen-server
that manages the lifetime of the program.
It links the top-level supervisor and shuts down the program when
requested or when the linked process dies.
The top-level supervisor, main-sup,
is configured one-for-all and no restarts so that a failure of any of
its children crashes the program. The event-mgr worker is the event
manager gen-server (see Chapter~\ref{chap:event-mgr}). The log-db
worker is a database gen-server (see Chapter~\ref{chap:log-db}) that
logs all events to the log database. The event-mgr-sentry worker is
used during shutdown to make sure the event manager stops sending
events to log-db before log-db shuts down. The statistics worker is a
system statistics gen-server (see Chapter~\ref{chap:stats}) that
periodically posts a \code{<statistics>} event.  The gatekeeper
worker is the gen-server described in Chapter~\ref{chap:gatekeeper}.

A web server can be added to the supervision tree by calling
\code{http:add-file-server} or \code{http:add-server} (see
Chapter~\ref{chap:http}) before calling \code{app:start}.
