% Copyright 2018 Beckman Coulter, Inc.
%
% Permission is hereby granted, free of charge, to any person
% obtaining a copy of this software and associated documentation files
% (the "Software"), to deal in the Software without restriction,
% including without limitation the rights to use, copy, modify, merge,
% publish, distribute, sublicense, and/or sell copies of the Software,
% and to permit persons to whom the Software is furnished to do so,
% subject to the following conditions:
%
% The above copyright notice and this permission notice shall be
% included in all copies or substantial portions of the Software.
%
% THE SOFTWARE IS PROVIDED "AS IS", WITHOUT WARRANTY OF ANY KIND,
% EXPRESS OR IMPLIED, INCLUDING BUT NOT LIMITED TO THE WARRANTIES OF
% MERCHANTABILITY, FITNESS FOR A PARTICULAR PURPOSE AND
% NONINFRINGEMENT. IN NO EVENT SHALL THE AUTHORS OR COPYRIGHT HOLDERS
% BE LIABLE FOR ANY CLAIM, DAMAGES OR OTHER LIABILITY, WHETHER IN AN
% ACTION OF CONTRACT, TORT OR OTHERWISE, ARISING FROM, OUT OF OR IN
% CONNECTION WITH THE SOFTWARE OR THE USE OR OTHER DEALINGS IN THE
% SOFTWARE.

\chapter {Supervisor}\label{chap:supervisor}

\section {Introduction}

In a fault tolerant system, faults must first be observed and then
acted upon. A \code{supervisor} monitors child processes for failure
and can be composed into a hierarchy to monitor for faults within
other supervisors.

The principles of supervisors and supervision hierarchies can be found
in Joe Armstrong's thesis~\cite{armstrong-thesis} or \emph{Programming
  Erlang---Software for a Concurrent
  World}~\cite{programming-erlang}. Documentation for Erlang's
\code{supervisor} is available online~\cite{supervisor-ref}. Source
code for the Erlang Open Telecom Platform can be found
online~\cite{erlang}. The source code for \code{supervisor} is part
of stdlib and can be found in /lib/stdlib/src/supervisor.erl.

\emph{Patterns for Fault Tolerant
  Software}~\cite{patterns-for-fault-tolerant-software} is a good
reference for understanding the mindset of creating fault tolerant
systems.

\section {Theory of Operation}

A \emph{supervisor} is a gen-server which is
responsible for starting, stopping, and monitoring its child
processes. A supervisor observes its children, and when a failure
occurs, restarts child processes.

A \emph{watcher} is a supervisor which is configured to only observe
the children. A watcher interface is provided for convenience.

A supervisor can be configured to restart individual children when
those children fail, or to restart all children when any child
fails. This is called the restart \var{strategy}. A strategy of
\code{one-for-one} indicates that when a child process terminates,
it should be restarted; only that child process is affected. A
strategy of \code{one-for-all} indicates that when a child process
terminates and should be restarted, all other child process are
terminated and then restarted.

A supervisor maintains a list of times of when a restart occurs. When
a child fails and is to be restarted, a timestamp is added to the
\var{restarts} list. A maximum restart frequency is represented as an
\var{intensity} and a \var{period} of time. If more than
\var{intensity} restarts occur in a \var{period} of time, the
supervisor terminates all child processes and then itself. This
prevents the possibility of an infinite cycle of child process
termination and restarts.

A supervisor is started with a list of child specifications. These
specifications are used to start child processes from within the
supervisor process during initialization.

Child specifications can be added to a supervisor at run time. These
dynamic children will not be automatically restarted if the supervisor
itself terminates and is restarted.

\paragraph*{state}\index{supervisor!state}
\code{(define-state-tuple <supervisor-state> strategy intensity
  period children restarts)}
\begin{itemize}
  \item \code{strategy} defines how the supervisor processes a child
    termination: \code{one-for-one} or \code{one-for-all}.
  \item \code{intensity} is the maximum restart intensity.
  \item \code{period} is the maximum restart period in milliseconds.
  \item \code{children} is a list of \code{<child>} tuples with
    the most recently started child first.
  \item \code{restarts} is an ordered list of times when restarts
    have occurred.
\end{itemize}

\code{(define-tuple <child> pid name thunk restart-type shutdown
  type)}\linebreak \code{pid} stores the child process or
\code{\#f}. The remaining fields are copied from the child
specification described below.

\genserver{supervisor}{init} The \code{init} procedure validates the
startup arguments and starts the initial child processes. Invalid
startup arguments cause the supervisor to fail to start. If any child
fails to start, all started children are terminated and the supervisor
fails to start.

This process traps exits so that it can detect child exits, as well as
the \code{EXIT} message from the parent process.

An invalid argument results in a specific error reason that includes
the invalid input.
\antipar
\begin{itemize}
  \item{\code{\#(invalid-strategy \var{strategy})}}
  \item{\code{\#(invalid-intensity \var{intensity})}}
  \item{\code{\#(invalid-period \var{period})}}
\end{itemize}

An invalid child specification during initialization will result in
\code{\#(error \#(start-specs \var{reason}))} where \var{reason} is
one of the reasons listed in the programming interface below.

\genserver{supervisor}{terminate} The \code{terminate} procedure
shuts each child process down in order (most recently added first).

\genserver{supervisor}{handle-call} The \code{handle-call} procedure
processes the following messages:

\antipar
\begin{itemize}
  \item{\code{\#(start-child \var{child-spec})}}: Validates the
    \var{child-spec}, starts the child, adds it to the state, and
    replies with \code{\#(ok \var{pid})} where \var{pid} is the new
    child process.

    If a child specification of the same name already exists,
    \code{\#(error already-present)} is returned. If the child
    process was already started \code{\#(error \#(already-started
      \var{pid}))} is returned.

    A successfully started child is linked to the supervisor, an event
    is fired to the event manager to log the start, and \code{\#(ok
      \var{pid})} is returned. If the \var{pid} already occurs in the
    children list, then \code{start-child} returns \code{\#(error
      \#(duplicate-process \var{pid}))}.

    If the child process start function returns \code{ignore}, the
    child specification is added to the supervisor, and the function
    returns \code{\#(ok \#f)}.

    If the child process start function returns \code{\#(error
      \var{reason})}, then \code{start-child} returns
    \code{\#(error \var{reason})}.

    If the child process start function exits with \var{reason},
    \code{\#(error \var{reason})} is returned.

    If the child process start function returns \var{other} values
    \code{\#(error \#(bad-return-value \var{other}))} is returned.

  \item{\code{\#(restart-child \var{name})}}: Finds a child by
    \var{name}, verifies that it is not currently running, then starts
    that child.

    If the child process is already running, \code{\#(error
      running)} is returned.  If the child specification does not
    exist, \code{\#(error not-found)} is returned.

    A successfully started child is linked to the supervisor, an event
    is fired to the event manager to log the start, and \code{\#(ok
      \var{pid})} is returned. If the \var{pid} already occurs in the
    children list, then \code{restart-child} returns \code{\#(error
      \#(duplicate-process \var{pid}))}.

    If the child process start function returns \code{ignore}, the
    child specification is added to the supervisor and the function
    returns \code{\#(ok \#f)}.

    If the child process start function returns \code{\#(error
      \var{reason})}, then \code{restart-child} returns
    \code{\#(error \var{reason})}.

    If the child process start function exits with \var{reason},
    \code{\#(error \var{reason})} is returned.

    If the child process start function returns \var{other} values
    \code{\#(error \#(bad-return-value \var{other}))} is returned.

  \item{\code{\#(delete-child \var{name})}}: Finds a child by
    \var{name}, verifies that it is not currently running, then
    removes the child specification from the state and returns
    \code{ok}.

    If the child process is running, \code{\#(error running)} is
    returned.  If the child specification does not exist,
    \code{\#(error not-found)} is returned.

  \item{\code{\#(terminate-child \var{name})}}: Finds a child by
    \var{name} and terminates it if it is running. The child \var{pid}
    is updated to \code{\#f} and returns \code{ok}.

    If the child specification does not exist, \code{\#(error
      not-found)} is returned.

  \item{\code{get-children}}: Returns the state's \code{children}
    field.
\end{itemize}

\genserver{supervisor}{handle-cast} The \code{handle-cast} procedure
does not process any messages.

\genserver{supervisor}{handle-info} The \code{handle-info} procedure
processes messages matching the following patterns:

\antipar
\begin{itemize}
  \item{\code{`(EXIT \var{pid} \var{reason})}}: Find \var{pid} in
    the children list and apply the restart strategy. An unknown
    \var{pid} is ignored.

    When the child specification \var{restart-type} is
    \code{permanent} or \code{transient} the current timestamp is
    prepended to the \var{restarts} list. The list is then pruned
    based on the \var{period}. If the resulting list length $<=$
    \var{intensity}, the supervisor continues. Otherwise, the
    supervisor terminates with reason \code{shutdown}.
\end{itemize}

Internally, the \code{(shutdown \var{pid} \var{x})} function kills
child processes and returns the exit reason. This function is used by
\code{terminate}, \code{terminate-child}, and during a failed
\code{init}. The following steps are necessary to defend against a
``naughty'' child which unlinks from the supervisor.

\antipar
\begin{itemize}
  \item Monitor \var{pid} to protect against a child process
    which may have called \code{unlink}.
  \item Unlink \var{pid} to stop receiving \code{EXIT} messages
    from \var{pid}.
  \item An \code{EXIT} message may already exist for \var{pid}. If
    it does, then wait for the \code{DOWN} message, and return the
    exit reason.
  \item If \var{x} $=$ \code{brutal-kill}, kill \var{pid} with
    reason \code{kill} and wait for the \code{DOWN} message to
    determine the exit reason.
  \item Otherwise, \var{x} is a \var{timeout}. kill \var{pid} with
    reason \code{shutdown} and wait for the \code{DOWN} message to
    determine the exit reason. If a \var{timeout} occurs, kill
    \var{pid} with reason \code{kill}, and wait for the
    \code{DOWN} message to determine the exit reason.
\end{itemize}

\section {Design Decisions}

Our initial implementation did not automatically link to child
processes, but this led to unexpected behavior when child processes
neglected to link to the supervisor.  Therefore, this implementation
links to all child processes.

\section {Programming Interface}

\code{supervisor:start\&link} and \code{supervisor:start-child}
use a child specification. A child specification\index{child specification}\label{page:child-spec}
is defined as:

\begin{grammar}
\prodrow{\var{child-spec}}{\code{\#(\var{name} \var{thunk} \var{restart-type} \var{shutdown} \var{type})}}
\end{grammar}

\var{name} is a symbol unique to the children within the supervisor.

\var{thunk} is a procedure that should spawn a process and link to the
supervisor process, then return \code{\#(ok \var{pid})} or
\code{\#(error \var{reason})} or \code{ignore}. Typically, the
thunk will call \code{gen-server:start\&link} which provides the
appropriate behavior and return value.

\var{restart-type} is a symbol with the following meaning:
\antipar
\begin{itemize}
  \item A \code{permanent} child process is aways restarted.
  \item A \code{temporary} child process is never restarted.
  \item A \code{transient} child process is only restarted if it
    terminates with an exit reason other than \code{normal} or
    \code{shutdown}.
  \item A \code{watch-only} child process is never restarted, and
    its child specification is removed from the supervisor when it
    terminates.
\end{itemize}

\var{shutdown} defines how a child process should be terminated.
\antipar
\begin{itemize}
\item \code{brutal-kill} indicates that the child process will be
  terminated using \code{(kill \var{pid} kill)}.

\item A fixnum $>$ 0 represents a timeout. The supervisor will use
  \code{(kill \var{pid} shutdown)} and wait for an exit signal. If
  no exit signal is received within the timeout, the child process
  will be terminated using \code{(kill \var{pid}
    kill)}. \code{infinity} can be used if and only if the
  \var{type} of the process is \code{supervisor}.
\end{itemize}

The \var{type} is useful for validating the \var{shutdown} parameter,
but is otherwise unused. It may be useful in conjunction with
\code{supervisor:get-children} to generate a tree of the running
supervision hierarchy.

\begin{grammar}
  \prodrow{\var{type}}{\code{supervisor}}
  \altrow{\code{worker}}
\end{grammar}

Invalid child specifications will result in specific error reasons
which include the invalid input.

\antipar
\begin{itemize}
  \item{\code{\#(invalid-name \var{name})}}
  \item{\code{\#(invalid-thunk \var{thunk})}}
  \item{\code{\#(invalid-restart-type \var{restart-type})}}
  \item{\code{\#(invalid-type \var{type})}}
  \item{\code{\#(invalid-shutdown \var{shutdown})}}
  \item{\code{\#(invalid-child-spec \var{spec})}}
\end{itemize}

\defineentry{supervisor:start\&link}
\begin{procedure}
  \code{(supervisor:start\&link \var{name} \var{strategy}
    \var{intensity} \var{period} \var{start-specs})}
\end{procedure}
\returns{}
\code{\#(ok \var{pid})} $|$
\code{\#(error \var{reason})}

The \code{supervisor:start\&link} procedure creates a new
\code{supervisor} gen-server using \code{gen-server:start\&link}.

\var{name} is the registered name of the process. For an anonymous
server, \code{\#f} may be specified.

\begin{grammar}
  \prodrow{\var{strategy}}{\code{one-for-one}}
  \altrow{\code{one-for-all}}
\end{grammar}

\begin{grammar}
  \prodrow{\var{intensity}}{a fixnum $>=$ 0}
\end{grammar}

\begin{grammar}
  \prodrow{\var{period}}{a fixnum $>$ 0}
\end{grammar}

\begin{grammar}
  \prodrow{\var{start-specs}}{(\var{child-spec} \etc)}
\end{grammar}

\defineentry{supervisor:start-child}
\begin{procedure}
  \code{(supervisor:start-child \var{supervisor} \var{child-spec})}
\end{procedure}
\returns{}
\code{\#(ok \var{pid})} $|$
\code{\#(error \var{reason})}

This procedure dynamically adds the given \var{child-spec} to the
\var{supervisor} which starts a child process.

The \code{supervisor:start-child} procedure calls
\code{(gen-server:call \var{supervisor} \#(start-child
  \var{child-spec}) infinity)}.

\defineentry{supervisor:restart-child}
\begin{procedure}
  \code{(supervisor:restart-child \var{supervisor} \var{name})}
\end{procedure}
\returns{}
\code{\#(ok \var{pid})} $|$
\code{\#(error \var{reason})}

This procedure restarts a child process identified by \var{name}. The
child specification must exist, and the child process must not be
running.

The \code{supervisor:restart-child} procedure calls
\code{(gen-server:call \var{supervisor} \#(restart-child \var{name})
  infinity)}.

\defineentry{supervisor:delete-child}
\begin{procedure}
  \code{(supervisor:delete-child \var{supervisor} \var{name})}
\end{procedure}
\returns{}
\code{ok} $|$
\code{\#(error \var{reason})}

This procedure deletes the child specification identified by
\var{name}. The child process must not be running.

The \code{supervisor:delete-child} procedure calls
\code{(gen-server:call \var{supervisor} \#(delete-child \var{name})
  infinity)}.

\defineentry{supervisor:terminate-child}
\begin{procedure}
  \code{(supervisor:terminate-child \var{supervisor} \var{name})}
\end{procedure}
\returns{}
\code{ok} $|$
\code{\#(error \var{reason})}

This procedure terminates the child process identified by
\var{name}. The child specification must exist, but the child process
does not need be running.

The \code{supervisor:terminate-child} procedure calls
\code{(gen-server:call \var{supervisor} \#(terminate-child
  \var{name}) infinity)}.

\defineentry{supervisor:get-children}
\begin{procedure}
  \code{(supervisor:get-children \var{supervisor})}
\end{procedure}
\returns{} a list of \code{<child>} tuples

This procedure returns the \var{supervisor} internal representation of
child specifications.

The \code{supervisor:get-children} procedure calls
\code{(gen-server:call \var{supervisor} get-children infinity)}.

\section {Published Events}

A supervisor can notify the event manager of the same events as a
gen-server, as well as the following events.

\begin{grammar}
  \prodrow{\var{event}}{\code{<supervisor-error>}}
  \altrow{\code{<child-start>}}
  \altrow{\code{<child-end>}}
\end{grammar}

\begin{pubevent}{<supervisor-error>}
  \argrow{timestamp}{the time the event occured}
  \argrow{supervisor}{the supervisor's process id}
  \argrow{error-context}{the context in which the event occured}
  \argrow{reason}{the reason for the error}
  \argrow{child-pid}{the child's process id}
  \argrow{child-name}{the child's name}
\end{pubevent}

This event is fired when the supervisor fails to start its children,
fails to restart its children, or when it has exceeded the maximum
restart frequency.

\begin{pubevent}{<child-start>}
  \argrow{timestamp}{the time the event occured}
  \argrow{supervisor}{the supervisor's process id}
  \argrow{pid}{the child's process id}
  \argrow{name}{the child's name}
  \argrow{restart-type}{the child's restart-type}
  \argrow{shutdown}{the child's shutdown}
  \argrow{type}{the child's type}
\end{pubevent}

This event is fired after the child start procedure has returned a
valid value.

\begin{pubevent}{<child-end>}
  \argrow{timestamp}{the time the event occured}
  \argrow{pid}{the child's process id}
  \argrow{killed}{1 indicates the supervisor terminated the child, 0 otherwise}
  \argrow{reason}{the reason the child has terminated}
\end{pubevent}

This event is fired after the supervisor terminates a child process,
and after the supervisor detects a failure in a child.

\section {Watcher Interface}

\defineentry{watcher:start\&link}
\begin{procedure}
\code{(watcher:start\&link \var{name})}
\end{procedure}

\returns{}
\code{\#(ok \var{pid})} $|$
\code{\#(error \var{reason})}

The \code{watcher:start\&link} procedure creates a supervisor with a
strategy of \code{one-for-one}, an intensity of 0, a period of 1,
and no children.

\var{name} is the registered name of the process. For an anonymous
server, \code{\#f} may be specified.

\defineentry{watcher:start-child}
\begin{procedure}
\code{(watcher:start-child \var{watcher} \var{name} \var{shutdown}
  \var{thunk})}
\end{procedure}

\returns{}
\code{\#(ok \var{pid})} $|$
\code{\#(error \var{reason})}

The \code{watcher:start-child} procedure calls
\code{(supervisor:start-child \var{watcher} \#(\var{name}
  \var{thunk} watch-only \var{shutdown} worker))}.

\defineentry{watcher:shutdown-children}
\begin{procedure}
\code{(watcher:shutdown-children \var{watcher})}
\end{procedure}

\returns{}
\code{ok}

The \code{watcher:shutdown-children} procedure terminates and
deletes each \code{watch-only} child in \var{watcher}.
