% Copyright 2023 Beckman Coulter, Inc.
%
% Permission is hereby granted, free of charge, to any person
% obtaining a copy of this software and associated documentation files
% (the "Software"), to deal in the Software without restriction,
% including without limitation the rights to use, copy, modify, merge,
% publish, distribute, sublicense, and/or sell copies of the Software,
% and to permit persons to whom the Software is furnished to do so,
% subject to the following conditions:
%
% The above copyright notice and this permission notice shall be
% included in all copies or substantial portions of the Software.
%
% THE SOFTWARE IS PROVIDED "AS IS", WITHOUT WARRANTY OF ANY KIND,
% EXPRESS OR IMPLIED, INCLUDING BUT NOT LIMITED TO THE WARRANTIES OF
% MERCHANTABILITY, FITNESS FOR A PARTICULAR PURPOSE AND
% NONINFRINGEMENT. IN NO EVENT SHALL THE AUTHORS OR COPYRIGHT HOLDERS
% BE LIABLE FOR ANY CLAIM, DAMAGES OR OTHER LIABILITY, WHETHER IN AN
% ACTION OF CONTRACT, TORT OR OTHERWISE, ARISING FROM, OUT OF OR IN
% CONNECTION WITH THE SOFTWARE OR THE USE OR OTHER DEALINGS IN THE
% SOFTWARE.

\chapter {Parallel}\label{chap:parallel}

\section{Introduction}

The parallel interface provides an API for spawning multiple worker
processes and collecting results in a fault-tolerant system. Because
Swish is single-threaded using lightweight processes, not all programs
benefit from multiplexing using this mechanism. Programs that are I/O
bound or spawn external OS processes benefit the most.

\section{Theory of Operation}

The \code{(swish parallel)} library implements a kernel capable of
spawning multiple processes, managing their lifetimes, and collecting
results. The processes involved are the caller, the kernel, and the
workers. The caller is the user code intending to multiplex the
work. The kernel spawns and manages the worker processes. Each worker
process is responsible for one unit of user-code work. These processes
are shown in Figure~\ref{fig:parallel}.

\begin{figure}
  \center\includegraphics{swish/parallel-theory.pdf}
  \caption{\label{fig:parallel}Parallel kernel}
\end{figure}

% Note: This description is very similar to a comment in the code. Try
% to keep them in sync.
The desired behavior here is a uni-directional link, which is not
native to Swish. To accomplish this, the caller starts the kernel
using \code{spawn\&link}, and the kernel enables
\code{process-trap-exit}. An exit signal generated by a worker
propagates to other workers, but does not signal the caller. An exit
signal from the caller will propagate to the workers, and the kernel
will complete normally.

The kernel is configured using \code{parallel:options}. The kernel
starts up to \code{start-limit} worker processes using
\code{spawn\&link} in some \code{order}\footnote{Because workers are
independent processes, start order does not guarantee execution
order. Control of start order provides a mechanism to help detect and
test for assumptions and errors in user code.}. As each worker
completes, a new one starts. When all the workers complete, the kernel
returns an API-specific value.

The kernel also supports a \code{timeout} for the entire parallel
operation. When the timeout expires, all remaining workers are killed
with the reason \code{timeout}, and an exception is thrown in the
caller process.

Because the caller is not killed by the kernel, it acts as an observer
of the kernel's behavior. As such, when an \code{event-mgr} process
exists, the caller publishes \code{<child-start>} and
\code{<child-end>} events described in
Section~\ref{supervisor:events}. Because the potential number of
workers may be high, the worker processes are not logged.

\section {Programming Interface}

Kernel options can be defined using
\code{(parallel:options [\var{option} \var{value}] \etc)}.
The following options may be used:
\defineentry{parallel:options}
\phantomsection % make pageref go to correct page for this label
\label{parallel:options}

\begin{tabular}{lp{5em}p{.65\textwidth}}
  option & default & description \\ \hline

  \code{start-limit}
  & \code{(most-positive-fixnum)}
  & a positive fixnum; the number of workers allowed to start and run
  concurrently \\

  \code{order}
  & \code{random}
  & a symbol, one of \code{random}, \code{left}, or \code{right} \\

  \code{timeout}
  & \code{infinity}
  & \code{infinity} or a nonnegative fixnum; the number of
  milliseconds the workers are allowed to execute before failing the
  entire operation \\
\end{tabular}

\defineentry{parallel}
\begin{syntax}
  \code{(parallel () \var{e} \etc)}\\
  \code{(parallel ([\var{option} \var{value}] \etc) \var{e} \etc)}\\
  \code{(parallel ,\var{options} \var{e} \etc)}
\end{syntax}

The \code{parallel} construct invokes each expression \var{e}, \etc{}
concurrently and returns a list of the resulting values. Expressions
may run concurrently unless \code{start-limit} is 1.

The \code{(parallel ([\var{option} \var{value}] \etc) \var{e} \etc)}
form constructs a copy of the default options, overriding each
specified \var{option} with the specified \var{value}.

The \code{(parallel ,\var{options} \var{e} \etc)} form allows you to
specify an expression for the \code{parallel:options} object described
in Section~\ref{parallel:options}.

\defineentry{parallel"!}
\begin{syntax}
  \code{(parallel! () \var{e} \etc)}\\
  \code{(parallel! ([\var{option} \var{value}] \etc) \var{e} \etc)}\\
  \code{(parallel! ,\var{options} \var{e} \etc)}
\end{syntax}

The \code{parallel!} construct behaves as \code{parallel} but returns
an unspecified value.

\defineentry{parallel:execute}
\begin{procedure}
  \code{(parallel:execute \opt{\var{options}} \var{thunks})}
\end{procedure}
\returns{} a list of results

The \code{parallel:execute} procedure takes a list of \var{thunks}
(procedures of no arguments), each of which returns a single
value. \code{parallel:execute} invokes the thunks concurrently and
returns a list of the resulting values. Each \var{procedure} may run
concurrently unless \code{start-limit} is 1.

The optional \var{options} argument is defined using
\code{parallel:options} described in Section~\ref{parallel:options}.

\defineentry{parallel:execute"!}
\begin{procedure}
  \code{(parallel:execute! \opt{\var{options}} \var{thunks})}
\end{procedure}
\returns{} unspecified

The \code{parallel:execute!} procedure behaves as
\code{parallel:execute} but returns an unspecified value.

\defineentry{parallel:for-each}
\begin{procedure}
  \code{(parallel:for-each \opt{\var{options}} \var{procedure} \var{list$_1$} \var{list$_2$} \etc)}
\end{procedure}
\returns{} unspecified

The \code{parallel:for-each} procedure works like the
\code{parallel:map} procedure except that it does not accumulate and
return a list of values. Each application of \var{procedure} may run
concurrently unless \code{start-limit} is 1.

The optional \var{options} argument is defined using
\code{parallel:options} described in Section~\ref{parallel:options}.

\defineentry{parallel:map}
\begin{procedure}
  \code{(parallel:map \opt{\var{options}} \var{procedure} \var{list$_1$} \var{list$_2$} \etc)}
\end{procedure}
\returns{} a list of results

The \code{parallel:map} procedure works like Chez Scheme's \code{map}
procedure. Each application of \var{procedure} may run concurrently
unless \code{start-limit} is 1.

The optional \var{options} argument is defined using
\code{parallel:options} described in Section~\ref{parallel:options}.

\defineentry{parallel:vector-map}
\begin{procedure}
  \code{(parallel:vector-map \opt{\var{options}} \var{procedure} \var{vector$_1$} \var{vector$_2$} \etc)}
\end{procedure}
\returns{} a vector of results

The \code{parallel:vector-map} procedure works like Chez Scheme's
\code{vector-map} procedure. Each application of \var{procedure} may
run concurrently unless \code{start-limit} is 1.

The optional \var{options} argument is defined using
\code{parallel:options} described in Section~\ref{parallel:options}.
