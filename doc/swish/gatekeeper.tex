% Copyright 2018 Beckman Coulter, Inc.
%
% Permission is hereby granted, free of charge, to any person
% obtaining a copy of this software and associated documentation files
% (the "Software"), to deal in the Software without restriction,
% including without limitation the rights to use, copy, modify, merge,
% publish, distribute, sublicense, and/or sell copies of the Software,
% and to permit persons to whom the Software is furnished to do so,
% subject to the following conditions:
%
% The above copyright notice and this permission notice shall be
% included in all copies or substantial portions of the Software.
%
% THE SOFTWARE IS PROVIDED "AS IS", WITHOUT WARRANTY OF ANY KIND,
% EXPRESS OR IMPLIED, INCLUDING BUT NOT LIMITED TO THE WARRANTIES OF
% MERCHANTABILITY, FITNESS FOR A PARTICULAR PURPOSE AND
% NONINFRINGEMENT. IN NO EVENT SHALL THE AUTHORS OR COPYRIGHT HOLDERS
% BE LIABLE FOR ANY CLAIM, DAMAGES OR OTHER LIABILITY, WHETHER IN AN
% ACTION OF CONTRACT, TORT OR OTHERWISE, ARISING FROM, OUT OF OR IN
% CONNECTION WITH THE SOFTWARE OR THE USE OR OTHER DEALINGS IN THE
% SOFTWARE.

\chapter {Gatekeeper}\label{chap:gatekeeper}

\section {Introduction}

The gatekeeper\index{gatekeeper} is a single gen-server named
\code{gatekeeper} that manages shared resources using mutexes.
Before a process uses a shared resource, it asks the gatekeeper to
enter the corresponding mutex. When the process no longer needs the
resource or terminates, it tells the gatekeeper to leave the mutex. A
process may enter the same mutex multiple times, and it needs to leave
the mutex the same number of times.  The gatekeeper breaks deadlocks
by raising an exception in one of the processes waiting for a mutex
involved in a cyclic dependency chain.

The gatekeeper hooks system primitives \code{\$cp0},
\code{\$np-compile}, \code{pretty-print}, and \code{sc-expand} because
they are not safe to be called from two processes at the same time
(see the discussion of the global winders list in
Section~\ref{sec:erlang-theory}). The \code{\$cp0} procedure uses
resource \code{\$cp0}, the \code{\$np-compile} procedure uses resource
\code{\$np-compile}, and so forth.

\section {Theory of Operation}

\paragraph* {state}\index{gatekeeper!state} The gatekeeper state is a
list of \code{<mutex>}\index{gatekeeper!<mutex>@\code{<mutex>}}
records, each of which has the following fields:\antipar

\begin{itemize}
\item \var{resource}: resource compared for equality using \code{eq?}
\item \var{process}: process that owns \var{resource}
\item \var{monitor}: monitor of \var{process}
\item \var{count}: number of times \var{process} has entered this
  mutex
\item \var{waiters}: ordered list of \var{from} arguments from
  \code{handle-call} for processes that are waiting to enter this
  mutex
\end{itemize}

\genserver{gatekeeper}{init} The gatekeeper \code{init} procedure
hooks the system primitives listed in the introduction so that they
use \code{with-gatekeeper-mutex} with a timeout of one minute, and it
sets the \code{current-expand} parameter to the hooked
\code{sc-expand} procedure. The process traps exits so that
\code{terminate} can unhook the system primitives when the process is
shut down. It returns an empty list of \code{<mutex>} records.

\genserver{gatekeeper}{terminate} The gatekeeper \code{terminate}
procedure unhooks the system primitives listed in the introduction and
sets the \code{current-expand} parameter to the unhooked
\code{sc-expand} procedure.

\genserver{gatekeeper}{handle-call} The gatekeeper
\code{handle-call} procedure handles the following messages:\antipar

\begin{itemize}
\item \code{\#(enter \var{resource})}: Find \var{mutex} $\in$
  \var{state} where \var{mutex}.\var{resource} = \var{resource}.

  If no such \var{mutex} exists, no-reply with \code{(enter-mutex
    \var{resource} \var{from} '() \var{state})}.

  If \var{mutex}.\var{process} = \var{from}.\var{process}, increment
  \var{mutex}.\var{count}, and reply \code{ok} with the updated
  state.

  If \code{(deadlock? \textrm{\var{from}.\var{process}} \var{mutex}
    \var{state})}, reply \code{\#(deadlock \var{resource})} with
  \var{state}.

  Otherwise, add \var{from} to the end of \var{mutex}.\var{waiters},
  and no-reply with the updated state.

\item \code{\#(leave \var{resource})}: Find \var{mutex} $\in$
  \var{state} where \var{mutex}.\var{resource} = \var{resource} and
  \var{mutex}.\var{process} = \var{from}.\var{process}.

  If no such \var{mutex} exists, reply \code{\#(unowned-resource
    \var{resource})} with \var{state}.

  If \var{mutex}.\var{count} $>$ 1, decrement \var{mutex}.\var{count},
  and reply \code{ok} with the updated state.

  Otherwise, reply \code{ok} with \code{(leave-mutex \var{mutex}
    \var{state})}.
\end{itemize}

\genserver{gatekeeper}{handle-cast} The gatekeeper
\code{handle-cast} procedure raises an exception on all messages.

\genserver{gatekeeper}{handle-info} The gatekeeper
\code{handle-info} procedure handles the following message:\antipar

\begin{itemize}
\item \code{\#(DOWN \var{monitor} \_ \_)}: Find \var{mutex} $\in$
  \var{state} where \var{mutex}.\var{monitor} =
  \var{monitor}. No-reply with \code{(leave-mutex \var{mutex}
    \var{state})}.
\end{itemize}

\index{gatekeeper!enter-mutex@\code{enter-mutex}}
\begin{procedure}
  \code{(enter-mutex \var{resource} \var{from} \var{waiters} \var{state})}
\end{procedure}
\returns{} updated state

The \code{enter-mutex} procedure calls \code{(gen-server:reply
  \var{from} 'ok)} to reply to the caller waiting to enter the
mutex. It adds a \code{<mutex>} record with
\var{resource} = \var{resource}, \var{process} = \var{from}.\var{process},
\var{monitor} = \code{(monitor \var{process})}, \var{count} = 1, and
\var{waiters} = \var{waiters} to \var{state}.

\index{gatekeeper!leave-mutex@\code{leave-mutex}}
\begin{procedure}
  \code{(leave-mutex \var{mutex} \var{state})}
\end{procedure}
\returns{} updated state

The \code{leave-mutex} procedure calls \code{(demonitor\&flush
  \textrm{\var{mutex}.\var{monitor}})}. If
\var{mutex}.\var{waiters} = \code{()}, it returns \code{(remq
  \var{mutex} \var{state})}. Otherwise, it returns
\code{(enter-mutex \textrm{\var{mutex}.\var{resource}} (car
  \textrm{\var{mutex}.\var{waiters}}) (cdr \textrm{\var{mutex}.\var{waiters}})
  (remq \var{mutex} \var{state}))}.

\index{gatekeeper!deadlock?@\code{deadlock?}}
\begin{procedure}
  \code{(deadlock? \var{process} \var{mutex} \var{state})}
\end{procedure}
\returns{} a boolean

The \code{deadlock?} procedure returns \code{\#t} if \var{process}
would deadlock waiting for \var{mutex}. Let \var{owner} =
\var{mutex}.\var{process}. If \var{owner} = \var{process}, return
\code{\#t}. Otherwise, find the mutex \var{waiting} $\in$
\var{state} where \code{\#(\var{owner} \_)} $\in$
\var{waiting}.\var{waiters}. If no such \var{waiting} exists, return
\code{\#f}. Otherwise, return \code{(deadlock? \var{process}
  \var{waiting} \var{state})}.

\section {Programming Interface}

\defineentry{gatekeeper:start\&link}
\begin{procedure}
  \code{(gatekeeper:start\&link)}
\end{procedure}
\returns{}
\code{\#(ok \var{pid})} $|$
\code{\#(error \var{reason})}

The \code{gatekeeper:start\&link} procedure calls
\code{(gen-server:start\&link 'gatekeeper)}.

\defineentry{gatekeeper:enter}
\begin{procedure}
  \code{(gatekeeper:enter \var{resource} \var{timeout})}
\end{procedure}
\returns{} \code{ok}

The \code{gatekeeper:enter} procedure calls \code{(gen-server:call
  'gatekeeper \#(enter \var{resource}) \var{timeout})} to enter the
mutex for \var{resource}. If it returns $e \ne \code{ok}$, it raises
exception $e$.

\defineentry{gatekeeper:leave}
\begin{procedure}
  \code{(gatekeeper:leave \var{resource})}
\end{procedure}
\returns{} \code{ok}

The \code{gatekeeper:leave} procedure calls \code{(gen-server:call
  'gatekeeper \#(leave \var{resource}))} to leave the mutex for
\var{resource}. If it returns $e \ne \code{ok}$, it raises exception
$e$.

\defineentry{with-gatekeeper-mutex}
\begin{syntax}
  \code{(with-gatekeeper-mutex \var{resource} \var{timeout} \var{body\(\sb{1}\)} \var{body\(\sb{2}\)} \etc)}
\end{syntax}
\expandsto{}
\code{(\$with-gatekeeper-mutex '\var{resource} \var{timeout} (lambda ()
  \var{body\(\sb{1}\)} \var{body\(\sb{2}\)} \etc))}

The \code{with-gatekeeper-mutex} form executes the body expressions in a
dynamic context where the calling process owns \var{resource}, which
must be an identifier. The \var{timeout} expression specifies how long
the caller is willing to wait to enter the mutex for \var{resource} as
defined by \code{gen-server:call}.  The internal
\code{\$with-gatekeeper-mutex} procedure is defined as follows:\antipar
\begin{samepage}\begin{alltt}
(define (\$with-gatekeeper-mutex \var{resource} \var{timeout} \var{body})
  (dynamic-wind
    (lambda () (gatekeeper:enter \var{resource} \var{timeout}))
    \var{body}
    (lambda () (gatekeeper:leave \var{resource}))))
\end{alltt}\end{samepage}
