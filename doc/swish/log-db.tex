% Copyright 2018 Beckman Coulter, Inc.
%
% Permission is hereby granted, free of charge, to any person
% obtaining a copy of this software and associated documentation files
% (the "Software"), to deal in the Software without restriction,
% including without limitation the rights to use, copy, modify, merge,
% publish, distribute, sublicense, and/or sell copies of the Software,
% and to permit persons to whom the Software is furnished to do so,
% subject to the following conditions:
%
% The above copyright notice and this permission notice shall be
% included in all copies or substantial portions of the Software.
%
% THE SOFTWARE IS PROVIDED "AS IS", WITHOUT WARRANTY OF ANY KIND,
% EXPRESS OR IMPLIED, INCLUDING BUT NOT LIMITED TO THE WARRANTIES OF
% MERCHANTABILITY, FITNESS FOR A PARTICULAR PURPOSE AND
% NONINFRINGEMENT. IN NO EVENT SHALL THE AUTHORS OR COPYRIGHT HOLDERS
% BE LIABLE FOR ANY CLAIM, DAMAGES OR OTHER LIABILITY, WHETHER IN AN
% ACTION OF CONTRACT, TORT OR OTHERWISE, ARISING FROM, OUT OF OR IN
% CONNECTION WITH THE SOFTWARE OR THE USE OR OTHER DEALINGS IN THE
% SOFTWARE.

\chapter {Log Database}\label{chap:log-db}

\section {Introduction}

The log database is a single gen-server named \code{log-db} that
uses the database interface (see Chapter~\ref{chap:db}) to log system
events (see Chapter~\ref{chap:event-mgr}).

\section {Theory of Operation}

\subsection {Initialization}

The \code{log-db} gen-server handles startup and setup through two
separate procedures. Startup uses the \code{db:start\&link}
procedure to connect to the the SQLite log database specified by the
\code{(log-file)} parameter. It creates the file if it does not
exist, but otherwise startup does not modify the database.

Setup makes sure the schema of the log database has been created and
is up-to-date. A unique symbol identifying the schema version is
stored in a table named version. This allows the software to upgrade
between known schema versions and to exit with an error when it
encounters an unsupported database version. These schema updates
happen within a database transaction so that if there is an error, the
changes are rolled back.

Setup calls \code{event-mgr:set-log-handler} after updating the
schema. This registers the \code{log-db} to log system events. It
also calls \code{event-mgr:flush-buffer}. This causes the event
manager to stop buffering startup events and the \code{log-db} to
log the events that were buffered.

Setup sends a \code{<system-attributes>} event so that \code{log-db}
receives and logs it. Finally, setup calls \code{db:expire-cache} to
release the schema definition queries.

Once the \code{log-db} gen-server has been setup, it continues to
receive events from the system event manager. It converts events that
it recognizes into insertions to the log database. Events that it does
not recognize are ignored.

The tables are pruned using insert triggers to hold 90 days of
information. To keep the insert operations fast, the timestamp columns
are indexed, and the pruning deletes no more than 10 rows per insert.
See \hyperref[make-swish-event-logger]{\code{make-swish-event-logger}}.

\subsection {Extensions}

An application typically produces events beyond those that are part of
Swish and may wish to log them in the same log database file where the
Swish events are logged. The \code{log-db} design allows for this
type of extension.

The \code{log-db:setup} procedure takes a list of
\code{<event-logger>} tuples. Each logger represents an extension
to the log database schema and contains two procedures, \code{setup}
and \code{log}. The \code{log-db:setup} procedure calls the
\code{setup} procedure of logger to make sure that its portion of
the schema has been created and is up-to-date. Then, when
\code{log-db} receives an event, it calls the \code{log} procedure
of each logger. If the event is recognized by that portion of the
schema, the \code{log} procedure inserts or updates data in the log
database. Otherwise, the procedure ignores that event.

Additionally, the version table does not store a single schema
version. Instead, it stores schema versions associated with names. The
\code{setup} procedure of an \code{<event-logger>} uses an unique
name for its portion of the schema and the \code{log-db:version}
procedure to retrieve and set its version.

The schema and logging for Swish events is implemented as an
\code{<event-logger>} defined by \code{swish-event-logger} and
using the schema version name \code{swish}. An application that
wishes to use this logging must provide \code{swish-event-logger} in
the list to \code{log-db:setup}. If the application wishes to log
Swish events in a different structure, it can omit the
\code{swish-event-logger} and provide its own logger with its own
schema. However, doing so makes the application more brittle with
respect to changes in the Swish implementation.

\section {Programming Interface}

\begin{tupledef}{<event-logger>}
  \argrow{setup}{procedure of no arguments that makes sure this
    portion of the schema is created and up-to-date}
  \argrow{log}{procedure of one argument, an event, that logs the
    event if it recognizes it and otherwise ignores it}
\end{tupledef}

\defineentry{log-db:start\&link}
\begin{procedure}
  \code{(log-db:start\&link \opt{\var{db-options}})}
\end{procedure}
\returns{}
\code{\#(ok \var{pid})} $|$
\code{\#(error \var{error})}

The \code{log-db:start\&link} procedure creates a new \code{db}
gen-server named \code{log-db} using \code{db:start\&link}. It
uses the value of the \code{(log-file)} parameter as the path to the
SQLite database and specifies \code{create} mode.
The optional \var{db-options} must be an object created by
\hyperref[db:options]{\code{db:options}}.

\defineentry{log-db:setup}
\begin{procedure}
  \code{(log-db:setup \var{loggers})}
\end{procedure}
\returns{}
\code{ignore} $|$
\code{\#(error \var{error})}

The argument \var{loggers} is a list of \code{<event-logger>}
tuples. The \code{log-db:setup} makes sure the \code{log-db} is
setup to run by doing the following in order.

\begin{enumerate}
  \item Initialize or upgrade the database schema from within a
    \code{db:transaction} call. It does this by calling the
    \code{setup} procedure of each logger.
  \item Register a procedure with \code{event-mgr:set-log-handler} to
    have the \code{log-db} gen-server log events it recognizes. When
    this procedure receives an event, it calls the \code{log}
    procedure of each logger.
  \item Call \code{event-mgr:flush-buffer} to stop buffering system
    events and apply the log handler to the events already buffered.
  \item Send a \code{<system-attributes>} event.
\end{enumerate}

If everything succeeds, the procedure returns \code{ignore}. If
either the \code{db:transaction} or
\code{event-mgr:set-log-handler} indicate an error, the procedure
returns that error.

\defineentry{log-db:version}
\begin{procedure}
  \code{(log-db:version \var{name} \opt{\var{version}})}
\end{procedure}

\begin{argtbl}
  \argrow{name}{symbol identifying the schema}
  \argrow{version}{string specifying the version of the schema}
\end{argtbl}

When called with one argument, \code{log-db:version} retrieves the
version associated with \var{name} from the database and returns it as
a string. It returns \code{\#f} if no version associated with
\var{name} is stored in the database.

When called with two arguments, it stores \var{version} as the version
associated with \var{name} in the database.

\defineentry{log-db:get-instance-id}
\begin{procedure}
  \code{(log-db:get-instance-id)}
\end{procedure}

\returns{} a string

\code{log-db:setup} associates a globally unique identifier with the
database file. The \code{log-db:get-instance-id} function caches and
returns that identifier.

\defineentry{make-swish-event-logger}
\begin{property}
  \code{(make-swish-event-logger \opt{\var{prune-max-days} \var{prune-limit}})}
  \label{make-swish-event-logger}
\end{property}

The \code{make-swish-event-logger} procedure returns
an \code{<event-logger>} tuple that defines the schema
for Swish events.
It uses the name \code{swish} to store its schema version.
The optional \var{prune-max-days} and \var{prune-limit} arguments are passed to
\code{create-prune-on-insert-trigger} when initializing the Swish event log
tables.
The default \var{prune-max-days} is 90.
The default \var{prune-limit} is 10.

\defineentry{swish-event-logger}
\begin{property}
  \code{swish-event-logger}
\end{property}

The \code{swish-event-logger} is an \code{<event-logger>} tuple
created by calling \code{(make-swish-event-logger)}.

% TODO deprecate swish-event-logger binding?

\defineentry{create-table}
\begin{syntax}\begin{alltt}
(create-table \var{name}
  (\var{field} \var{type} . \var{inline})
  \etc{})\strut\end{alltt}
\end{syntax}
\expandsto{} \code{(execute "create table if not exists \etc{}")}

The \code{create-table} syntax describes the schema of a single
table and expands into a call to \code{execute} to create the table
if no table with that name already exists. The name of the table,
\var{name}, and of each field, \var{field}, are converted from Scheme
to SQL identifiers by replacing hyphen characters with underscores and
eliminating any non-alphanumeric and non-underscore characters. The
SQL definition of each field is produced by joining the converted
field name, the \var{type} and any additional \var{inline} arguments
into a space separated string.

\defineentry{define-simple-events}
\begin{syntax}\begin{alltt}
(define-simple-events \var{create} \var{handle}
  (\var{name} \var{clause} \etc{})
  \etc{})\strut\end{alltt}
\end{syntax}
\expandsto{} A definition of the \var{create} and \var{handle}
procedures

The \code{define-simple-events} syntax is used to log tuple types
by inserting a row into a table with the same name and the same
fields. Each \var{name} is a tuple type. Each \var{clause} is a valid
\code{create-table} clause for one of the fields in that tuple
type.

It defines \var{create} as a procedure of 0 arguments that consists of
a \code{(create-table \var{name} \var{clause} \etc{})} for each
tuple in the \code{define-simple-events}. This means that the name
of the tuple type and each field are converted to SQL names by the
\code{create-table} syntax.

It defines \var{handle} as a procedure of 1 argument, an event. If the
event is one of the tuple types in the \code{define-simple-events}, it
calls \code{db:log} with an insert statement applying \code{coerce} to
each value. If the event is unrecognized, it returns \code{\#f}.

\defineentry{coerce}
\begin{procedure}
  \code{(coerce \var{x})}
\end{procedure}
\returns{} a Scheme object

The argument \var{x} is a Scheme object mapped to a SQLite value.

\begin{tabular}{lp{0.75\textwidth}}
  \var{type} & transformation\\ \hline
  \var{string} & \var{string} \\
  \var{bytevector} & \var{bytevector} \\
  \var{number} & \var{number, if it fits in 64 bits} \\
  \var{symbol} & \code{symbol->string} \\
  \var{date} & {\code{format-rfc2822}} \\
  \var{JSON object} & \code{json:object->string} \\
  \var{process} & {\code{global-process-id}} \\
  \var{condition}
  & a string representing a JSON object with the following fields:\\
  & \code{message} containing \code{(exit-reason->english \var{x})} and\\
  & \code{stacks} containing \code{(map stack->json (exit-reason->stacks \var{x}))} \\
  \var{continuation-condition} & a string containing \code{\#(error
    \var{reason} \var{stack})} where the \var{stack} is obtained from
  \code{dump-stack} \\
  \hline
\end{tabular}

\code{coerce} passes \code{\#f} through unmodified which SQLite
interprets as NULL.  Other values are converted to string using
\code{write}.

\defineentry{create-prune-on-insert-trigger}
\begin{procedure}
  \code{(create-prune-on-insert-trigger \var{table} \var{column} \var{max-days} \var{limit})}
\end{procedure}
\returns{} unspecified

The \code{create-prune-on-insert-trigger} procedure should be called
only within a thunk \var{f} provided to \code{db:transaction}.
It creates a temporary trigger that prunes, after an insert, up to
\var{limit} rows of the specified \var{table} where the \code{erlang:now}
timestamp in \var{column} is older than \var{max-days}.
\var{max-days} must be a nonnegative fixnum, and
\var{limit} must be a positive fixnum.
To keep insert operations fast, \var{column} should be indexed.

\defineentry{stack->json}
\begin{procedure}
  \code{(stack->json \var{k} \opt{\var{max-depth}})}
\end{procedure}
\returns{} a JSON object

The \code{stack->json} procedure renders the stack of continuation \var{k}
as a JSON object by calling \hyperlink{walk-stack}{\code{walk-stack}}.
The return value may contain the following keys:

\begin{tabular}{lp{4.6in}}
  \code{type} & \code{"stack"} \\
  \code{depth} & the depth of the stack \\
  \code{truncated} & if present, the \var{max-depth} at which the stack dump was truncated \\
  \code{frames} & if present, a list of JSON objects representing stack frames
\end{tabular}

A stack frame may contain the following keys:

\begin{tabular}{lp{4.6in}}
  \code{type} & \code{"stack-frame"} \\
  \code{depth} & the depth of this frame \\
  \code{source} & if present, a source object for the return point \\
  \code{procedure-source} & if present, a source object for the procedure containing the return point \\
  \code{free} & if present, a list of JSON objects representing free variables
\end{tabular}

A source object \var{x} with source file descriptor \var{sfd} is
represented by a JSON object containing the following keys:

\begin{tabular}{lp{4.6in}}
  \code{bfp} & \code{(source-object-bfp \var{x})} \\
  \code{efp} & \code{(source-object-efp \var{x})} \\
  \code{path} & \code{(source-file-descriptor-path \var{sfd})} \\
  \code{checksum} & \code{(source-file-descriptor-checksum \var{sfd})}
\end{tabular}

A free variable with value \var{val} is represented by a JSON object
containing the following keys:

\begin{tabular}{lp{4.6in}}
  \code{name} & a string containing the variable name or its index \\
  \code{value} & the result of \code{\fixtilde(format "~s" \var{val})} \\
\end{tabular}

\defineentry{json-stack->string}
\begin{procedure}
  \code{(json-stack->string \opt{\var{op}} \var{x})}
\end{procedure}
\returns{} see below

The two argument form of \code{json-stack->string} prints the
stack represented by JSON object \var{x} to the textual output port \var{op}.
The single argument form of \code{json-stack->string} prints the stack
represented by JSON object \var{x} to a string output port and returns
the resulting string.
In either case, the printed form resembles that generated by \code{dump-stack}
except that source locations are given as file offsets rather than line and character
numbers.

\section {Published Events}

\begin{pubevent}{<system-attributes>}
  \argrow{timestamp}{timestamp from \code{erlang:now}}
  \argrow{date}{date from \code{current-date}}
  \argrow{software-info}{JSON object from \code{software-info}}
  \argrow{machine-type}{\code{(symbol->string (machine-type))}}
  \argrow{computer-name}{computer name from \code{osi\_get\_hostname}}
  \argrow{os-system}{\code{(<uname> system (get-uname))}}
  \argrow{os-release}{\code{(<uname> release (get-uname))}}
  \argrow{os-version}{\code{(<uname> version (get-uname))}}
  \argrow{os-machine}{\code{(<uname> machine (get-uname))}}
\end{pubevent}

The \code{<system-attributes>} event is sent exactly once, when
\code{log-db:setup} is called.
