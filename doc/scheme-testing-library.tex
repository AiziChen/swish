% Copyright 2018 Beckman Coulter, Inc.
%
% Permission is hereby granted, free of charge, to any person
% obtaining a copy of this software and associated documentation files
% (the "Software"), to deal in the Software without restriction,
% including without limitation the rights to use, copy, modify, merge,
% publish, distribute, sublicense, and/or sell copies of the Software,
% and to permit persons to whom the Software is furnished to do so,
% subject to the following conditions:
%
% The above copyright notice and this permission notice shall be
% included in all copies or substantial portions of the Software.
%
% THE SOFTWARE IS PROVIDED "AS IS", WITHOUT WARRANTY OF ANY KIND,
% EXPRESS OR IMPLIED, INCLUDING BUT NOT LIMITED TO THE WARRANTIES OF
% MERCHANTABILITY, FITNESS FOR A PARTICULAR PURPOSE AND
% NONINFRINGEMENT. IN NO EVENT SHALL THE AUTHORS OR COPYRIGHT HOLDERS
% BE LIABLE FOR ANY CLAIM, DAMAGES OR OTHER LIABILITY, WHETHER IN AN
% ACTION OF CONTRACT, TORT OR OTHERWISE, ARISING FROM, OUT OF OR IN
% CONNECTION WITH THE SOFTWARE OR THE USE OR OTHER DEALINGS IN THE
% SOFTWARE.

\documentclass[letterpaper,11pt,twoside,final]{article}
\usepackage{swish}

\begin{document}

\title {Scheme Testing Library}
\author {Leif Danielson \and Bob Burger}
\date {\copyright\ 2018 Beckman Coulter, Inc.
  Licensed under the \href{https://opensource.org/licenses/MIT}{MIT License}.}
\maketitle

\section* {Introduction}

The Scheme Testing Library \code{(swish mat)} provides methods to
define, iterate through, and run test cases, and to log the
results. Test cases are called \emph{mats} and consist of a name, a
set of tags, and a test procedure.  The set of mats is stored in
reverse order in a single, global list.  The list of tags allows the
user to group tests or mark them.  For example, tags can be used to
note that a test was created for a particular change request.

The Scheme Testing Library uses only the \code{(chezscheme)}
library so that it can be used to test any Scheme code.

A \emph{mat} is an entry in the global list of the form
\code{(\var{name} \var{tags} . \var{test})}. The \var{test} is a
procedure of no arguments.

\section* {Programming Interface}

% ----------------------------------------------------------------------------
\begin{syntax}
  \code{(mat \var{name} (\var{tag} \etc) $e_1$ $e_2$ \etc)}
\end{syntax}
\expandsto{} \code{(add-mat '\var{name} '(\var{tag} \etc)
  (lambda () $e_1$ $e_2$ \etc))}

The \code{mat} macro creates a mat with the given \var{name},
\var{tag}s, and test procedure $e_1$ $e_2$ \etc\ using the
\code{add-mat} procedure.

% ----------------------------------------------------------------------------
\begin{procedure}
  \code{(add-mat \var{name} \var{tags} \var{test})}
\end{procedure}
\returns{} unspecified

The \code{add-mat} procedure adds a mat to the front of the global
list. \var{name} is a symbol, \var{tags} is a list, and \var{test} is
a procedure of no argument.

If \var{name} is already used, an exception is raised.

% ----------------------------------------------------------------------------
\begin{procedure}
  \code{(run-mat \var{name} \var{reporter})}
\end{procedure}
\returns{} see below

The \code{run-mat} procedure runs the mat of the given \var{name} by
executing its test procedure with an altered exception handler. If the
test procedure completes without raising an exception, the mat result
is \code{pass}. If the test procedure raises exception \var{e}, the
mat result is \code{(fail~.~\var{e})}.

After the mat completes, the \code{run-mat} procedure tail calls
\code{(\var{reporter} \var{name} \var{tags} \var{result} \var{statistics})}.

If no mat with the given \var{name} exists, an exception is raised.

% ----------------------------------------------------------------------------
\begin{syntax}
  \code{(run-mats)}
\end{syntax}
\begin{syntax}
  \code{(run-mats \var{name$_1$} \var{name$_2$} \etc)}
\end{syntax}
\returns{} unspecified

The \code{run-mats} macro runs each mat specified by symbols
\var{name$_1$} \var{name$_2$} \etc.  When no names are supplied, all
mats are executed.  After each mat is executed, its result, name, and
exception if it failed are displayed.  When the mats are finished, a
summary of the number run, passed, and failed is displayed.

% ----------------------------------------------------------------------------
\begin{procedure}
  \code{(run-mats-to-file \var{filename})}
\end{procedure}
\returns{} unspecified

The \code{run-mats-to-file} procedure executes all mats and writes
the results into the file specified by the string \var{filename}. If
the files exists, its contents are overwritten. The file format is a
sequence of s-expressions readable with \code{load-results} and
\code{summarize}.

% ----------------------------------------------------------------------------
\begin{procedure}
  \code{(for-each-mat \var{procedure})}
\end{procedure}
\returns{} unspecified

The \code{for-each-mat} procedure calls \code{(\var{procedure}
  \var{name} \var{tags})} for each mat, in no particular order.

% ----------------------------------------------------------------------------
\begin{procedure}
  \code{(load-results \var{filename})}
\end{procedure}
\returns{} a list of expressions

The \code{load-results} procedure reads the contents of the file
specified by string \var{filename} and returns the list of
s-expressions contained within.

% ----------------------------------------------------------------------------
\begin{procedure}
  \code{(summarize \var{files})}
\end{procedure}
\returns{} two values: the number of passing mats and the number of
failing mats

The \code{summarize} procedure reads the contents of each file in
\var{files}, a list of string filenames, and returns the number of
passing mats and the number of failing mats. An error is raised if any
entry is malformed.

\end{document}
